\documentclass[11pt, oneside]{article}   	% use "amsart" instead of "article" for AMSLaTeX format


% \usepackage{draftwatermark}
% \SetWatermarkText{Draft}
% \SetWatermarkScale{5}
% \SetWatermarkLightness {0.9} 
% \SetWatermarkColor[rgb]{0.7,0,0}


\usepackage{geometry}                		% See geometry.pdf to learn the layout options. There are lots.
\geometry{letterpaper}                   		% ... or a4paper or a5paper or ... 
%\geometry{landscape}                		% Activate for for rotated page geometry
%\usepackage[parfill]{parskip}    		% Activate to begin paragraphs with an empty line rather than an indent
\usepackage{graphicx}				% Use pdf, png, jpg, or eps� with pdflatex; use eps in DVI mode
								% TeX will automatically convert eps --> pdf in pdflat						
								% TeX will automatically convert eps --> pdf in pdflatex		
\usepackage{amssymb}
\usepackage{mathrsfs}
\usepackage{hyperref}
\usepackage{url}
\usepackage{subcaption}
\usepackage{authblk}
\usepackage{amsmath}
\usepackage{mathtools}
\usepackage{graphicx}
\usepackage[export]{adjustbox}
\usepackage{fixltx2e}
\usepackage{hyperref}
\usepackage{alltt}
\usepackage{color}
\usepackage[utf8]{inputenc}
\usepackage[english]{babel}
\usepackage{float}
\usepackage{bigints}
\usepackage{braket}
\usepackage{siunitx}

%
% so you can do e.g., \begin{bmatrix}[r] (or [c] or [l])
%

\makeatletter
\renewcommand*\env@matrix[1][c]{\hskip -\arraycolsep
  \let\@ifnextchar\new@ifnextchar
  \array{*\c@MaxMatrixCols #1}}
\makeatother

\newcommand{\argmax}{\operatornamewithlimits{argmax}}
\newcommand{\argmin}{\operatornamewithlimits{argmin}}

\begin{document}

\section*{What Does The Series $\sum\limits_{n = 1}^\infty (\frac{a}{b})^n $ Converge To?}

\noindent
Well,  $\sum\limits_{n = 1}^\infty (\frac{a}{b})^n  = \frac{a}{b \:-\: a}$ where $a,b \in \mathbb{N}$, $b \neq 0$ and $a < b$ (so $b - a \neq 0$). OK, but why?

\bigskip
\noindent
 Here's one way to think about it:  
      
\begin{equation*}
\begin{array}{lllll}
S
&=& \sum\limits_{n = 1}^\infty (\frac{a}{b})^n                                                                                                                              &\qquad  \mathrel{\#} \text{define $S$}                                           \\ 
\vspace{2.0mm}
&=& (\frac{a}{b})^1 + (\frac{a}{b})^2 + (\frac{a}{b})^3 + \cdots                                                                                                    &\qquad  \mathrel{\#} \text{expand series}                                      \\
\vspace{2.0mm}
&\Rightarrow& (\frac{b}{a}) \cdot S = (\frac{b}{a}) \cdot \Big [(\frac{a}{b})^1 + (\frac{a}{b})^2 + (\frac{a}{b})^3 + \cdots \Big ]   &\qquad  \mathrel{\#} \text{multiply both sides by $\frac{b}{a}$}      \\
\vspace{2.0mm}
&\Rightarrow&  (\frac{b}{a}) \cdot S = 1 +(\frac{a}{b})^1 + (\frac{a}{b})^2 + (\frac{a}{b})^3 + \cdots                                          &\qquad  \mathrel{\#} \text{multiply through on right side}                \\
\vspace{2.0mm}
&\Rightarrow&  (\frac{b}{a}) \cdot S = 1 + S                                                                                                                              &\qquad  \mathrel{\#} \text{definition of $S$}                                     \\
\vspace{2.0mm}
&\Rightarrow&  (\frac{b}{a}) \cdot S - S = 1                                                                                                                               &\qquad  \mathrel{\#} \text{subtract $S$ from both sides}                 \\
\vspace{2.0mm}
&\Rightarrow&  (\frac{b}{a}) \cdot S - (\frac{a}{a}) \cdot S = 1                                                                                                   &\qquad  \mathrel{\#} \text{multiply $S$ by $1 = \frac{a}{a}$}            \\
\vspace{2.0mm}
&\Rightarrow&  S \cdot \Big [\frac{b}{a}  - \frac{a}{a} \Big ] = 1                                                                                                 &\qquad  \mathrel{\#} \text{factor out $S$}                                          \\
\vspace{2.0mm}
&\Rightarrow& S \cdot \Big [ \frac{b \:-\: a} {a} \Big ]= 1                                                                                                           &\qquad  \mathrel{\#} \text{simplify}                                                      \\ 
% \vspace{2.0mm}
&\Rightarrow& S  = \frac{a}{b \:-\: a}                                                                                                                                        &\qquad  \mathrel{\#} \text{multiply both sides by $ \frac{a}{b \:-\: a} $}
\end{array}
\end{equation*}

\bigskip
\noindent
So $S = \sum\limits_{n = 1}^\infty (\frac{a}{b})^n  =  \frac{a}{b \:-\: a}$ where $a,b \in \mathbb{N}$, $b \neq 0$ and $a < b$.  

\bigskip
\noindent
For example, if we  let $a =1$ and $b =2$ then $\sum\limits_{n = 1}^\infty (\frac{1}{2})^n  =  \frac{1}{2 \:-\: 1} = 1$. Similarly, if $a =1$ and $b = 3$ then 
$\sum\limits_{n = 1}^\infty (\frac{1}{3})^n  =  \frac{1}{3 \:-\: 1} = \frac{1}{2}$. 

\bigskip
\noindent
\section*{Acknowledgements}

Thanks to Dave Neary for pointing out that this series diverges if $a \geq b$.

\end{document} 

