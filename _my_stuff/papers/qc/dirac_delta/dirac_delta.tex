\documentclass{article}

% \usepackage{draftwatermark}
% \SetWatermarkText{Draft}
% \SetWatermarkScale{5}
% \SetWatermarkLightness {0.9} 
% \SetWatermarkColor[rgb]{0.7,0,0}


\usepackage{geometry}                		% See geometry.pdf to learn the layout options. There are lots.
\geometry{letterpaper}                   		% ... or a4paper or a5paper or ... 
%\geometry{landscape}                		% Activate for for rotated page geometry
%\usepackage[parfill]{parskip}    		% Activate to begin paragraphs with an empty line rather than an indent
\usepackage{graphicx}				% Use pdf, png, jpg, or eps� with pdflatex; use eps in DVI mode
								% TeX will automatically convert eps --> pdf in pdflat						
								% TeX will automatically convert eps --> pdf in pdflatex		
\usepackage{amssymb}
\usepackage{mathrsfs}
\usepackage{hyperref}
\usepackage{url}
\usepackage{subcaption}
\usepackage{authblk}
\usepackage{amsmath}
\usepackage{mathtools}
\usepackage{graphicx}
\usepackage[export]{adjustbox}
\usepackage{fixltx2e}
\usepackage{hyperref}
\usepackage{alltt}
\usepackage{color}
\usepackage[utf8]{inputenc}
\usepackage[english]{babel}
\usepackage{float}
\usepackage{bigints}
\usepackage{braket}
\usepackage{siunitx}

\usepackage{tikz}
\usepackage{verbatim}
\usetikzlibrary{plotmarks}
\usepackage{pgfplots}
\usepackage{amsmath}
 \usepackage{relsize}


\title{A Few Notes On The Dirac Delta Function}
\author{David Meyer \\ dmm@\{1-4-5.net,uoregon.edu,...\}}

\date{Last update: \today}							% Activate to display a given date or no date

\begin{document}
\maketitle

\section{Introduction}
These notes began life as some thoughts on the Dirac Delta Function and evolved into notes on several related topics including  Laplace Transforms. The 
Dirac Delta function has all kinds of crazy and interesting properties. More TBD.

\section{The Dirac Delta Function}
The Dirac Delta Function is defined as shown in Figure \ref{fig:delta}. In the limit ($\mathlarger{\epsilon} \to 0$) the
 Dirac Delta function is written $\delta_a(t)$ or sometimes $\delta(t - a)$. As we will see in a moment, the $\delta_{a,\epsilon}(t)$ form of the delta function
 is useful when we want to use the Mean Value Theorem for Integrals \cite{wiki:meam_value_theorem_for_integrals} to evaluate integrals involving the delta function.
 
\bigskip

\begin{figure}[H]
  \centering
  \begin{tikzpicture}[scale=2.0]
     \draw [<->] (0,3) -- (0,0) -- (4,0);                                                                      % draw axes
     \draw (3,2) node[rectangle] {                                                                           % draw function to the left
         $\delta_{a,\epsilon}(t) =  
           \begin{cases} 
             \frac{1}{\mathlarger {\epsilon}} & a \leq t \leq a + \epsilon \\
             0                                               & \text{otherwise}
           \end{cases}$
       }; 
     \draw [dashed] (1,0) node[below] {$a$} -- (1,1);
     \draw (0,1) node[left] {$\frac{1}{\mathlarger{\epsilon}}$} ;
     \draw [dashed] (2,0) node[below] {$a + \epsilon$} -- (2,1);
     \coordinate (y) at (0,1);\fill [red] (y) circle (1pt)  (0,1) circle (1 pt);
     \draw [thick,red] (0,0) -- (1,0);
     \draw [thick,red] (2,0) -- (4,0);
     \draw [thick,red] (1,1) -- (2,1);
     \draw [dashed]  (0,1) -- (1,1);
     \draw[] (4,0) node [ label=below:{$t$}] {};
     \draw[] (0,3) node [ label=above:{$\delta_{a,\epsilon}(t)$}] {};
  \end{tikzpicture}
  \caption{The $\delta_{a,\epsilon}(t)$ function}
  \label{fig:delta}
\end{figure}

\bigskip
\noindent
So $\delta_{a,\epsilon}(t)$ is defined to be

\begin{equation*}
\delta_{a,\epsilon}(t) =  
 \begin{cases} 
      \frac{1}{\mathlarger {\epsilon}} & a \leq t \leq a + \epsilon \\
      0                                               & \text{otherwise}
   \end{cases}
\label{eqn:delta}
\end{equation*}

\noindent
\bigskip
and has the constraint that


\begin{equation*}
  \int_{0}^{\infty} \delta_{a,\epsilon}(t) =  1
\end{equation*}

\bigskip
\bigskip
\noindent
\bigskip
That is, $\delta_{a,\epsilon}(t)$ is in some sense a probability density.

\bigskip
\noindent
In the limit the Dirac Delta Function looks like

\begin{equation*}
\lim_{\epsilon \to 0} \delta_{a,\epsilon}(t) =
\delta_{a}(t) =  
 \begin{cases} 
     \infty & t = a\\
      0     & t \neq a
   \end{cases}
\end{equation*}

\bigskip
\noindent
or sometimes

\begin{equation*}
\delta (t - a) =  
  \begin{cases} 
     0 & t \neq a\\
      \infty     & t = a
  \end{cases}
\end{equation*}

\bigskip
\noindent
$\delta_a(t)$ also has the constraint that 


\begin{equation*}
  \int_{0}^{\infty} \delta_{a}(t) =  1
\end{equation*}

\bigskip
\noindent
and so is also a probability density. $\delta_a(t)$  is shown in Figure \ref{fig:delta_limit}.

\begin{figure}[H]
  \centering
  \begin{tikzpicture}[scale=2.0]
     \draw [<->] (0,2) -- (0,0) -- (4,0);                                                                      % draw axes
     \draw (3,1) node[rectangle] {                                                                           % draw function to the left
         $\delta_{a}(t) =  
           \begin{cases} 
              \infty & t = a  \\
               0      & \text{otherwise}
           \end{cases}$
       }; 
     \draw (1,0) node[below] {$a$};
     \draw [dashed,red] (1,0) -- (1,2);
     \draw[] (4,0) node [ label=below:{$t$}] {};
     \draw[] (0,2) node [ label=above:{$\delta_{a}(t)$}] {};
  \end{tikzpicture}
  \caption{The $\delta_{a}(t)$ function}
  \label{fig:delta_limit}
\end{figure}

\subsection{Integrals Involving $\delta_{a,\epsilon}(t)$}
$\delta_{a,\epsilon}(t)$ has all kinds of interesting properties. One of them involves the integral of the product $\delta_{a,\epsilon}(t)$ with some function $g(t)$.  
Here we would like to evaluate integrals of the form

\bigskip
\begin{equation}
  \int_{0}^{\infty} \delta_{a,\epsilon}(t) g(t) dt
  \label{eqn:delta_integral}
\end{equation}

\bigskip
\noindent
where $g(t)$ is continuous on the interval $[a, a+\epsilon]$. 

\bigskip
\noindent
The Mean Value Theorem for Integrals  \cite{wiki:meam_value_theorem_for_integrals}  tells us that

\begin{equation}
  \int_{a}^{b} g(t) dt = (b -a) g(c)
  \label{eqn:mvti}
\end{equation}

\bigskip
\noindent
where the point $c$ lies in the interval $[a, a+\epsilon]$. Now, since we know that $\delta_{a,\epsilon}(t)$ is zero everywhere except on the interval
$[a, a+\epsilon]$ we can rewrite the improper integral in Equation \ref{eqn:delta_integral} as the proper integral

\begin{equation*}
  \int_{a}^{a + \epsilon} \delta_{a,\epsilon}(t) g(t) dt
\end{equation*}

\bigskip
\noindent
Now we can notice that $ \delta_{a,\epsilon}(t) = \frac{1}{\mathlarger{\epsilon}}$ on the interval $[a, a+\epsilon]$ so we can rewrite our integral as 

\bigskip
\begin{equation*}
  \int_{a}^{a + \epsilon} \frac{1}{\mathlarger{\epsilon}} g(t) dt =  \frac{1}{\mathlarger{\epsilon}} \int_{a}^{a + \epsilon} g(t) dt
\end{equation*}

\bigskip
\noindent
Now we can use Equation \ref{eqn:mvti}, the Mean Value Theorem for Integrals\footnote{This is where the $\delta_{a,\epsilon}(t)$ form of the delta function comes in handy.},
by setting $b = a + \epsilon$ and $a = a$ so that $b - a = \epsilon$. Then by the Mean Value Theorem for Integrals


\bigskip
\begin{equation*}
 \frac{1}{\mathlarger{\epsilon}} \int_{a}^{a + \epsilon} g(t) dt =  \frac{1}{\mathlarger{\epsilon}} \underbrace{ \Big [(a + \epsilon) - a \Big]}_{b -a} g(c) = \frac{1}{\mathlarger{\epsilon}} \epsilon g(c) = g(c)
\end{equation*}

\bigskip
\noindent
where $c \in [a, a + \epsilon]$. Finally, if we look at the limit as $\epsilon \to 0$ we see that $\lim_{\epsilon \to 0} c = a$ so that 

\bigskip
\begin{equation}
  \int_{0}^{\infty} \delta_{a}(t) g(t) dt = g(a)
\end{equation}

\bigskip
\noindent
Essentially $\delta_{a}(t)$ pulls out the value of $g$ at $a$, that is, $g(a)$.

\bigskip
\noindent
Another way to get this result is to notice that the integrand of 

\bigskip
\begin{equation*}
  \int_{0}^{\infty} \delta (t-a) g(t) dt
\end{equation*}

\bigskip
\noindent
is zero everywhere except where $t = a$, so we can rewrite our integral as
 $\int_{0}^{\infty} \delta (t-a) g(a) dt = g(a) \int_{0}^{\infty} \delta (t-a) dt$ (since $g(a)$ doesn't depend on $t$). 
 Then since by definition $ \int_{0}^{\infty} \delta (t-a) dt = 1$ we get

\bigskip
\begin{equation*}
  \int_{0}^{\infty} \delta (t-a) g(t) dt = g(a)
\end{equation*}
\section*{Acknowledgements}

\newpage
\bibliographystyle{plain}
\bibliography{/Users/dmm/papers/bib/qc}
\end{document}