\documentclass{article}

% \usepackage{draftwatermark}
% \SetWatermarkText{Draft}
% \SetWatermarkScale{5}
% \SetWatermarkLightness {0.9} 
% \SetWatermarkColor[rgb]{0.7,0,0}


\usepackage{geometry}                		% See geometry.pdf to learn the layout options. There are lots.
\geometry{letterpaper}                   		% ... or a4paper or a5paper or ... 
%\geometry{landscape}                		% Activate for for rotated page geometry
%\usepackage[parfill]{parskip}    		% Activate to begin paragraphs with an empty line rather than an indent
\usepackage{graphicx}				% Use pdf, png, jpg, or eps� with pdflatex; use eps in DVI mode
								% TeX will automatically convert eps --> pdf in pdflat						
								% TeX will automatically convert eps --> pdf in pdflatex		
\usepackage{amssymb}
\usepackage{mathrsfs}
\usepackage{hyperref}
\usepackage{url}
\usepackage{subcaption}
\usepackage{authblk}
\usepackage{amsmath}
\usepackage{mathtools}
\usepackage{graphicx}
\usepackage[export]{adjustbox}
\usepackage{fixltx2e}
\usepackage{hyperref}
\usepackage{alltt}
\usepackage{color}
\usepackage[utf8]{inputenc}
\usepackage[english]{babel}
\usepackage{float}
\usepackage{bigints}
\usepackage{braket}
\usepackage{siunitx}

\usepackage{tikz}
\usepackage{verbatim}
\usetikzlibrary{plotmarks}
\usepackage{pgfplots}
\usepackage{amsmath}
 \usepackage{relsize}


\title{A Few Notes On The Dirac Delta Function}
\author{David Meyer \\ dmm@\{1-4-5.net,uoregon.edu,...\}}

\date{Last update: \today}							% Activate to display a given date or no date

\begin{document}
\maketitle

\section{Introduction}
These notes began life as some thoughts on the Dirac Delta Function and evolved into notes on several related topics including  Laplace Transforms. The 
Dirac Delta function has all kinds of crazy and interesting properties. More TBD.

\section{The Dirac Delta Function}
The Dirac Delta Function is defined as shown in Figure \ref{fig:delta}. In the limit ($\mathlarger{\epsilon} \to 0$) the
 Dirac Delta function is written $\delta_a(t)$ or sometimes $\delta(t - a)$. As we will see in a moment, the $\delta_{a,\epsilon}(t)$ form of the delta function
 is useful when we want to use the Mean Value Theorem for Integrals \cite{wiki:meam_value_theorem_for_integrals} to evaluate integrals involving the delta function.
 
\bigskip

\begin{figure}[H]
  \centering
  \begin{tikzpicture}[scale=2.0]
     \draw [<->] (0,3) -- (0,0) -- (4,0);                                                                      % draw axes
     \draw (3,2) node[rectangle] {                                                                           % draw function to the left
         $\delta_{a,\epsilon}(t) =  
           \begin{cases} 
             \frac{1}{\mathlarger {\epsilon}} & a \leq t \leq a + \epsilon \\
             0                                               & \text{otherwise}
           \end{cases}$
       }; 
     \draw [dashed] (1,0) node[below] {$a$} -- (1,1);
     \draw (0,1) node[left] {$\frac{1}{\mathlarger{\epsilon}}$} ;
     \draw [dashed] (2,0) node[below] {$a + \epsilon$} -- (2,1);
     \coordinate (y) at (0,1);\fill [red] (y) circle (1pt)  (0,1) circle (1 pt);
     \draw [thick,red] (0,0) -- (1,0);
     \draw [thick,red] (2,0) -- (4,0);
     \draw [thick,red] (1,1) -- (2,1);
     \draw [dashed]  (0,1) -- (1,1);
     \draw[] (4,0) node [ label=below:{$t$}] {};
     \draw[] (0,3) node [ label=above:{$\delta_{a,\epsilon}(t)$}] {};
  \end{tikzpicture}
  \caption{The $\delta_{a,\epsilon}(t)$ function}
  \label{fig:delta}
\end{figure}

\bigskip
\noindent
So $\delta_{a,\epsilon}(t)$ is defined to be

\begin{equation*}
\delta_{a,\epsilon}(t) =  
 \begin{cases} 
      \frac{1}{\mathlarger {\epsilon}} & a \leq t \leq a + \epsilon \\
      0                                               & \text{otherwise}
   \end{cases}
\label{eqn:delta}
\end{equation*}

\noindent
\bigskip
and has the constraint that


\begin{equation*}
  \int_{0}^{\infty} \delta_{a,\epsilon}(t) =  1
\end{equation*}

\bigskip
\bigskip
\noindent
\bigskip
That is, $\delta_{a,\epsilon}(t)$ is in some sense a probability density.


\section*{Acknowledgements}

\newpage
\bibliographystyle{plain}
\bibliography{/Users/dmm/papers/bib/qc}
\end{document}