\documentclass[11pt, oneside]{article}   	% use "amsart" instead of "article" for AMSLaTeX format


\usepackage{geometry}                		% See geometry.pdf to learn the layout options. There are lots.
\geometry{letterpaper}                   		% ... or a4paper or a5paper or ... 
%\geometry{landscape}                		% Activate for for rotated page geometryhttps://www.washingtonpost.com/world/europe/amid-impeachment-probe-gordon-sondland-is-overseeing-a-renovation-of-his-residence-that-has-cost-1-million-in-taxpayer-money/2019/10/16/d0eece92-ef86-11e9-bb7e-d2026ee0c199_story.html?tid=sm_tw
%\usepackage[parfill]{parskip}    		% Activate to begin paragraphs with an empty line rather than an indent
\usepackage{graphicx}				% Use pdf, png, jpg, or eps� with pdflatex; use eps in DVI mode
								% TeX will automatically convert eps --> pdf in pdflat						\label{thm:integral_domain}

								% TeX will automatically convert eps --> pdf in pdflatex		
\usepackage{amssymb}
\usepackage{amsmath}
\usepackage{amsthm}
\usepackage{mathrsfs}
\usepackage[hyphens,spaces,obeyspaces]{url}
\usepackage{url}
\usepackage{hyperref}
\usepackage{subcaption}
\usepackage{authblk}
\usepackage{mathtools}
\usepackage{graphicx}
\usepackage[export]{adjustbox}
\usepackage{fixltx2e}
\usepackage{hyperref}
\usepackage{alltt}
\usepackage{color}
\usepackage[utf8]{inputenc}
\usepackage[english]{babel}
\usepackage{float}
\usepackage{bigints}
\usepackage{braket}
\usepackage{siunitx}
\usepackage{mathtools}


\usepackage{tikz}
\usepackage{verbatim}
\usetikzlibrary{plotmarks}
\usepackage{pgfplots}
\usepackage{amsmath}
\usepackage{relsize}
 \usepackage[hyphenbreaks]{breakurl}
 
 \newtheorem{thm}{Theorem}[section]
% \newtheorem{defn}[thm]{Definition}
\theoremstyle{definition}
\newtheorem{definition}{Definition}[section]
\newtheorem{proposition}{Proposition}[section]
\newtheorem{lemma}{Lemma}[section]
\newtheorem{example}{Example}[section]

\newcommand{\argmax}{\operatornamewithlimits{argmax}}
\newcommand{\argmin}{\operatornamewithlimits{argmin}}





\title{Why Is There No Difference In Our Ages?}
\author{David Meyer \\ dmm@1-4-5.net}

\date{Last update: \today}							% Activate to display a given date or no date

\begin{document}
\maketitle

\bigskip
\noindent
Suppose your age is $x_t$ and my age is $y_t$ at some time $t$. Then in $n$ years your age is $x_t + n$ and my age is $y_t + n$. What this implies
is that as time goes by (measured by $n$), the difference in our ages vanishes!

\bigskip
\noindent
Why? Consider that 

\medskip
\bigskip
\begin{equation*}
\centering
{\mathlarger {\lim\limits_{n \to \infty} \Bigg [ \frac{x_t + n}{y_t + n} \Bigg ]  = 1}}
\end{equation*}

\bigskip
\noindent
which is another way of saying the same thing. This result is reassuring since it pretty much models our experience. 

\bigskip
\noindent
More generally, consider the function $f_a(n) = a + n$ where $a,n \in \mathbb{N}$. Then

\medskip
\bigskip
\begin{equation}
\centering
\mathlarger {\lim\limits_{n \to \infty} \Bigg [ \frac{f_a(n)}{f_b(n)} \Bigg ]  = 1}
\label{eqn:lim}
\end{equation}

\bigskip
\noindent
for $a,b \in \mathbb{N}$. We also write Equation \ref{eqn:lim} in the following alternate notation:

\bigskip
\begin{equation*}
f_a(n) \thicksim f_b(n)
 \end{equation*}

\bigskip
\noindent
That is, the $\thicksim$ symbol means that the ratio of its two arguments tends towards $1$ as its arguments tend toward $\infty$.


\end{document} 

