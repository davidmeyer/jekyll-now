\documentclass[11pt, oneside]{article}   	% use "amsart" instead of "article" for AMSLaTeX format


% \usepackage{draftwatermark}
% \SetWatermarkText{Draft}
% \SetWatermarkScale{5}
% \SetWatermarkLightness {0.9} 
% \SetWatermarkColor[rgb]{0.7,0,0}

\usepackage{geometry}                		% See geometry.pdf to learn the layout options. There are lots.
\geometry{letterpaper}                   		% ... or a4paper or a5paper or ... 
%\geometry{landscape}                		% Activate for for rotated page geometry
%\usepackage[parfill]{parskip}    		% Activate to begin paragraphs with an empty line rather than an indent
\usepackage{graphicx}				% Use pdf, png, jpg, or eps� with pdflatex; use eps in DVI mode
								% TeX will automatically convert eps --> pdf in pdflat						
								% TeX will automatically convert eps --> pdf in pdflatex		
\usepackage{amssymb}
\usepackage{mathrsfs}
\usepackage{hyperref}
\usepackage{url}
\usepackage{subcaption}
\usepackage{authblk}
\usepackage{amsmath}
\usepackage{mathtools}
\usepackage{graphicx}
\usepackage[export]{adjustbox}
\usepackage{fixltx2e}
\usepackage{hyperref}
\usepackage{alltt}
\usepackage{color}
\usepackage[utf8]{inputenc}
\usepackage[english]{babel}
\usepackage{float}
\usepackage{bigints}
\usepackage{braket}
\usepackage{siunitx}

%
% so you can do e.g., \begin{bmatrix}[r] (or [c] or [l])
%

\makeatletter
\renewcommand*\env@matrix[1][c]{\hskip -\arraycolsep
  \let\@ifnextchar\new@ifnextchar
  \array{*\c@MaxMatrixCols #1}}
\makeatother

\newcommand{\argmax}{\operatornamewithlimits{argmax}}
\newcommand{\argmin}{\operatornamewithlimits{argmin}}

\title{Notes on Change of Basis}
\author{David Meyer \\ dmm@\{1-4-5.net,uoregon.edu\}}

\date{Last update: \today}							% Activate to display a given date or no date



\begin{document}
\maketitle


\section{Introduction}
This document contains a few of my notes made when I was reviewing change of basis in linear algebra.

\section{Definitions}
Consider a set $B \subseteq V$ where $V$ is a $n$-dimensional vector space. $B$ is a \emph{basis} of $V$ if every element of 
$V$ may be written in a unique way as a (finite) linear combination of elements of $B$. The coefficients of this linear combination 
are referred to as components or \emph{coordinates} on $B$ of the vector. The coordinates of a given vector $x \in V$ expressed in 
basis $B$ is denoted as $[x]_B$.

\section{The Basis of a Vector Space}
\label{sub:basis_vectors}
A basis $B$ of a vector space $V$ over a field $\mathbb{F}$ (e.g. $\mathbb{Q}$, $\mathbb{R}$ or $\mathbb{C}$) is a linearly independent subset of $V$
that spans $V$. This means that a subset $B$ of $V$ is a basis if the following two conditions hold:

\begin{itemize}
\item Linear Independence
\begin{itemize}
\item for every finite subset of  $\{b_1, \cdots, b_n\}$ of $B$ and $\{a_1, \cdots, a_n\} $ of $\mathbb{F}$ if  \\
$a_1b_1 +  \hdots + a_nb_n = 0$ then $a_1 = \hdots = a_n = 0$
\end{itemize}
\item $B$ \emph{spans} $V$
\begin{itemize}
\item for every vector $\mathbf{v} \in V$ it is possible to choose $\{a_1, \cdots,  a_n\}$ in $\mathbb{F}$ and \\
$\{b_1,  \cdots, b_n\}$ in $B$ such that $\mathbf{v} = a_1b_1 + \cdots +  a_nb_n$. 
\end{itemize}
\end{itemize}

\bigskip
\noindent
The elements of $B$  are called basis vectors and importantly both $B$ and $V$ are ordered sets.


\bigskip
\noindent
The summary is that the elements of a basis $B$ are linearly independent and every element of $V$ can be represented as a linear combination 
of the elements of $B$ with coefficients from the field $\mathbb{F}$. 

\section{The Standard Basis}
The standard basis (also called natural basis) for a Euclidean space is the set of unit vectors pointing in the direction of the axes of a Cartesian coordinate system.
For example, the standard basis for three-dimensional space is formed by the vectors

\begin{equation*}
\mathbf{e}_{x} = (1,0,0),\quad \mathbf{e}_{y} = (0,1,0), \quad \mathbf{e}_{z} = (0,0,1)
\end{equation*}

\bigskip
\noindent
These vectors are a basis in the sense that any other vector can be expressed uniquely as a linear combination of these. 
For example, every vector $\mathbf{v}$ in three-dimensional space can be written uniquely as

\begin{equation*}
\mathbf{v} = v_{x} \mathbf {e} _{x} + v_{y}\mathbf {e} _{y} + v_{z}\mathbf {e}_{z}
\end{equation*}

\bigskip
\noindent
where  $v_x, v_y, v_z \in \mathbb{F}$  are the scalar "coordinates" of $\mathbf{v}$ in the standard basis. The \emph{coordinate vector} for $\mathbf{v}$ is denoted
$[\mathbf{v}]_B$ for basis $B$. Here  $[\mathbf{v]}_{STD} = \begin{bmatrix} v_x \\ v_y \\ v_z \end{bmatrix}$.

\bigskip
\noindent
In general,  the standard basis (which is sometimes called the computational basis)  for the $n$-dimensional Euclidean space consists of the ordered set
of $n$ distinct vectors

\begin{equation*}
\{\mathbf {e} _{i}: 1 \leq i \leq n\}
\end{equation*}

\bigskip
\noindent
where $\mathbf{e}_i$ is the $i^\text{th}$ basis vector which has a 1 in the $i^\text{th}$ coordinate and 0's elsewhere.\footnote{This is also called a "one-hot" 
encoding in machine learning parlance, where the $i$'s might be the classes in a classification problem.}
That is, the $\mathbf{e}_i$ are the $n \times 1$ column vectors 

\begin{flalign*}
\mathbf{e}_{1} = 
\begin{bmatrix} 
1 \\
0 \\
\vdots \\
0 \\
0
\end{bmatrix}, 
\mathbf{e}_{2} = 
\begin{bmatrix} 
0 \\
1 \\
\vdots \\
0 \\
0
\end{bmatrix}, 
\hdots, 
\mathbf{e}_{n-1} =
\begin{bmatrix} 
0 \\
0 \\
\vdots \\
1 \\
0
\end{bmatrix}, 
\mathbf{e}_{n} =
\begin{bmatrix} 
0 \\
0 \\
\vdots \\
0 \\
1
\end{bmatrix}
\end{flalign*}


\bigskip
\noindent
Standard bases can be defined for other vector spaces such as polynomials and matrices. In both cases, the standard basis consists of the elements of the 
vector space such that all coefficients but one are 0 and the non-zero one is 1. For polynomials, the standard basis thus consists of the monomials and is 
commonly called monomial basis. 

\bigskip
\noindent
For example, the standard basis for $\mathcal{P}^2$, the set of polynomials of degree $\leq 2$ over some field $\mathbb{F}$ is the ordered set

\begin{equation*}
B_{\mathcal{P}^2} = \left \{ \begin{bmatrix} x^0 \\ 0 \\0 \end{bmatrix}, \begin{bmatrix} 0 \\ x^1 \\0 \end{bmatrix}, \begin{bmatrix} 0 \\ 0 \\ x^2 \end{bmatrix} \right \} \\
=  \left \{ \begin{bmatrix} 1 \\ 0 \\0 \end{bmatrix}, \begin{bmatrix} 0 \\ x \\0 \end{bmatrix}, \begin{bmatrix} 0 \\ 0 \\ x^2 \end{bmatrix} \right \}
\end{equation*} 

\bigskip
\noindent
By the definition of basis any polynomial $a +bx +cx^2 \in \mathcal{P}^2$ can be written as a linear combination of vectors from $B_{\mathcal{P}^2}$. That is 


\begin{equation*}
a +bx +cx^2 = a  \begin{bmatrix} 1 \\ 0 \\0 \end{bmatrix} + b \begin{bmatrix} 0 \\ x \\0 \end{bmatrix} + c \begin{bmatrix} 0 \\ 0 \\ x^2 \end{bmatrix} \\
\end{equation*}

\noindent
for $a,b,c \in \mathbb{F}$. 

\bigskip
\noindent
We can also see this in matrix form by loading the vectors of $B_{\mathcal{P}^2}$ into a matrix. We call this matrix
$P_{\mathcal{P}^2} = \begin{bmatrix} 1 & 0 & 0 \\ 0 & x & 0 \\ 0 & 0 & x^2 \end{bmatrix}$. In matrix form:

\bigskip
\begin{equation*}
\begin{bmatrix} 1 & 0 & 0 \\ 0 & x & 0 \\ 0 & 0 & x^2 \end{bmatrix} \begin{bmatrix} a \\ b \\ c \end{bmatrix} =
\begin{bmatrix} [l]
a \cdot 1 &+& b \cdot 0 &+& c \cdot 0 \\
a \cdot 0 &+& b \cdot x &+& c \cdot 0 \\
a \cdot 0 &+& b \cdot 0 &+& c \cdot x^2 
\end{bmatrix}
= a  \begin{bmatrix} 1 \\ 0 \\0 \end{bmatrix} + b \begin{bmatrix} 0 \\ x \\0 \end{bmatrix} + c \begin{bmatrix} 0 \\ 0 \\ x^2 \end{bmatrix} \\
= a +bx +cx^2 
\end{equation*}



\bigskip
\noindent
This example shows an interesting property of vector spaces: once a basis is chosen for a $n$-dimensional vector space $V$ we get an
isomorphism between $V$ and $\mathbb{F}^n$. For polynomials $a +bx +cx^2 \in \mathcal{P}^2$ over a field $\mathbb{F}$, we get the
isomorphism  $\mathcal{P}^2 \cong \mathbb{F}^3$. Here the mapping is

\bigskip
\begin{equation*}
 a +bx +cx^2 \longmapsto \begin{bmatrix} a \\ b \\ c \end{bmatrix} \text{and} \\
 \begin{bmatrix} a \\ b \\ c \end{bmatrix} \longmapsto a +bx +cx^2
\end{equation*}

\bigskip
\noindent
where $a +bx +cx^2 \in  \mathcal{P}^2$, $\begin{bmatrix} a \\ b \\ c \end{bmatrix}  \in \mathbb{F}^3$, and $\mathbb{F}$ could be any field.


\bigskip
\noindent
For matrices $\mathcal {M}_{m \times n}$ , the standard basis consists of the $m \times n$-matrices with exactly one 
non-zero entry, which is 1.  For example, the standard basis for $2 \times 2$ over $\mathbb{R}$ matrices is formed by the 4 matrices

\bigskip
\begin{equation*}
\mathbf{e}_{11} = \begin{bmatrix}1&0\\0&0 \end{bmatrix}, \quad \mathbf{e}_{12} = \begin{bmatrix} 0&1\\0&0 \end{bmatrix},\quad
\mathbf{e}_{21} = \begin{bmatrix}0&0\\1&0\end{bmatrix},\quad \mathbf{e}_{22} = \begin{bmatrix}0&0\\0&1\end{bmatrix}
\end{equation*}

\bigskip
\section{Examples}
Consider the bases $B = \Bigg \{\begin{bmatrix}  1 \\ 1\end{bmatrix}, \begin{bmatrix} 2 \\ 0 \end{bmatrix} \Bigg \}$ and
$C = \Bigg \{\begin{bmatrix}  0 \\ 1\end{bmatrix}, \begin{bmatrix} 1 \\ 2 \end{bmatrix} \Bigg \}$ of the vector space $\mathbb{R}^2$. Suppose 
further that we know that $[x]_B = \begin{bmatrix}  3 \\ 1 \end{bmatrix}$.

\bigskip
\noindent
First question: How do we find $[x]_C$? The brute force method is to convert 

\begin{equation*}
[x]_B \rightarrow [x]_{STD} \rightarrow [x]_C
\end{equation*}

\bigskip
\noindent
To do this, first we multiply $[x]_B$ by the corresponding basis elements:

\begin{equation*}
[x]_B = \begin{bmatrix} 3 \\ 1 \end{bmatrix} \implies x =  [x]_{STD}   = 3 \cdot  \begin{bmatrix}  1 \\ 1\end{bmatrix} + 1 \cdot  \begin{bmatrix}  2 \\ 0\end{bmatrix} 
= \begin{bmatrix}  3 \\ 3 \end{bmatrix}  + \begin{bmatrix}  2 \\ 0\end{bmatrix} = \begin{bmatrix}  5 \\ 3\end{bmatrix} 
\end{equation*}

\bigskip
\noindent
So $[x]_{STD} = \begin{bmatrix}  5 \\ 3\end{bmatrix}$.  Next we have to solve 

\begin{equation*}
[x]_C = \begin{bmatrix} x_1 \\ x_2 \end{bmatrix} \implies x_1 \cdot \begin{bmatrix} 0 \\ 1 \end{bmatrix} + x_2 \cdot \begin{bmatrix} 1 \\  2 \end{bmatrix} =  \begin{bmatrix}  5 \\ 3\end{bmatrix}
\end{equation*} 

\bigskip
\noindent
So  

\begin{equation}
x_1 \cdot \begin{bmatrix} 0 \\ 1 \end{bmatrix} + x_2 \cdot \begin{bmatrix} 1 \\  2 \end{bmatrix}  =  \begin{bmatrix}  5 \\ 3\end{bmatrix}
\label{eqn:solve}
\end{equation} 

\bigskip
\noindent
First, recall that if $A = \begin{bmatrix} a & b \\ c & d \end{bmatrix}$ and $\det(A) \neq 0$ 
then\footnote{Recall also that a $n \times n$ matrix $A$ is invertible ($A^{-1}$ exists) iff $\det(A) \neq 0$.}

\bigskip
\begin{equation}
A^{-1} = \frac{1}{\det(A)} \begin{bmatrix}[r] d & -b \\ -c & a \end{bmatrix}
\label{eqn:a_inverse}
\end{equation}

\noindent
and

\begin{equation}
\det(A) = ad - bc
\label{eqn:det}
\end{equation}

\bigskip
\noindent
\textbf{Aside}: The set of $2 \times 2$ invertible matrices over the real numbers with the group operation being matrix multiplication is called $GL(2,\mathbb{R})$, 
the \emph{General Linear Group}  over $\mathbb{R}$. That is,  

\begin{flalign*}
GL(2,\mathbb{R}) 
& = \big \{2 \times 2 \text{ matrices $A$ over $\mathbb{R}$}  \mid  \det(A) \neq 0 \big \}  \\
&=  \Bigg \{\begin{bmatrix} a & b \\ c & d \end{bmatrix} \mid   a,b,c,d \in \mathbb{R} \text{ and } ad-bc \neq 0 \Bigg \}
\end{flalign*}

\bigskip
\noindent
Next, rewriting Equation \ref{eqn:solve} in matrix form we get


\begin{equation*}
\begin{array}{rcll}
\begin{bmatrix} 0 & 1 \\ 1 & 2  \end{bmatrix}   \begin{bmatrix} x_1 \\  x_2 \end{bmatrix}  
&=& \begin{bmatrix}  5 \\ 3\end{bmatrix}
     & \qquad \mathrel{\#}  \text{solve for} \begin{bmatrix} x_1 \\ x_2 \end{bmatrix} \text{in matrix form} \\
\begin{bmatrix} x_1 \\ x_2 \end{bmatrix}
&=& \frac{1} {\det \Bigg (\begin{bmatrix} 0 & 1 \\ 1 & 2  \end{bmatrix} \Bigg )} \begin{bmatrix}[r] 2 & -1 \\ -1 & 0 \end{bmatrix} \begin{bmatrix} 5 \\ 3 \end{bmatrix} 
     & \qquad \mathrel{\#}  \text{multiply both sides by} \begin{bmatrix} 0 & 1 \\ 1 & 2  \end{bmatrix}^{-1}, \text{ Equation } \ref{eqn:a_inverse}  \\
&=& \frac{1}{-1} \begin{bmatrix}[r] 2 & -1 \\ -1 & 0 \end{bmatrix} \begin{bmatrix} 5 \\ 3 \end{bmatrix} 
     & \qquad \mathrel{\#}  \det \Bigg (\begin{bmatrix} 0 & 1 \\ 1 & 2  \end{bmatrix} \Bigg ) =  0 \cdot 2 -1 \cdot 1 = -1, \text{ Equation } \ref{eqn:det} \\
&=& \begin{bmatrix}[r] -2 & 1 \\ 1 & 0 \end{bmatrix} \begin{bmatrix} 5 \\ 3 \end{bmatrix} 
     & \qquad \mathrel{\#} \text{scalar multiplication} \\
&=& \begin{bmatrix}[r] -7 \\ 5 \end{bmatrix} 
     & \qquad \mathrel{\#} \begin{bmatrix} x_1 \\ x_2 \end{bmatrix} = \begin{bmatrix}[r] -7 \\ 5 \end{bmatrix} 
\end{array}
\end{equation*}


\bigskip
\noindent
Note that since we need to compute $A^{-1}$ the determinate of $A$ cannot be zero.  The set of
$n \times n$ invertible matrices $S$ are just those for which $ A \in S \implies \det(A) \neq 0$. 

\bigskip
\noindent
Here  the matrix $\begin{bmatrix} 0 & 1 \\ 1 & 2  \end{bmatrix}$ is called $P_C$. Similarly, the matrix $\begin{bmatrix} 1 & 2 \\ 1 & 0  \end{bmatrix}$ is called
$P_B$. So we have

\bigskip
\begin{equation*}
P_B [\mathbf{e}_1]_B = \begin{bmatrix} 1 \\ 1 \end{bmatrix}, P_B[\mathbf{e}_2]_B = \begin{bmatrix} 2 \\ 0 \end{bmatrix},
 P_C[\mathbf{e}_1]_C = \begin{bmatrix} 0 \\ 1 \end{bmatrix}, \text{ and } P_C[\mathbf{e}_2]_C= \begin{bmatrix} 1 \\ 2 \end{bmatrix}
 \end{equation*}
 
 \bigskip
 \noindent
where  $\mathbf{e}_i$ is the $i^{\text{th}}$ element of the (ordered) basis. What we can observe here is that in general  
$P_B[x]_B = [x]_{STD}$  and  $P_C [x]_C = [x]_{STD}$ and so $P_{B}[x]_B =  P_{C}[x]_C$. 


\bigskip
\noindent
Note that the method above converts $[x]_B$ to $[x]_{STD}$ then to $[x]_C$. In general you wouldn't want to do this since we can compute the conversion matrix
$P_{B \rightarrow C}$ directly without having to go through the standard basis. 

\bigskip
\noindent
So how do we compute $P_{B \rightarrow C}$?  We saw above that  $P_{B}[x]_B =  P_{C}[x]_C$, so if we multiply 
both sides by $P^{-1}_C$ we get  $[x]_C = P^{-1}_{C} P_{B} [x]_B$, which tells us that $P_{B \rightarrow C} = P^{-1}_{C} P_{B}$.


\bigskip
\noindent
Applying this result to our example above

\begin{equation*}
\begin{array}{rcll}
[x]_C
&=& \underbrace{ \frac{1}{-1}  \begin{bmatrix}[r] 2 & -1 \\ -1 & 0 \end{bmatrix}}_{P^{-1}_C}
        \underbrace{\begin{bmatrix} 1 & 2  \\ 0 & 1 \end{bmatrix}}_{P_B}
        \underbrace{\begin{bmatrix} 3 \\1 \end{bmatrix}}_{[x]_B}
     & \qquad \qquad \mathrel{\#}  [x]_C =  P^{-1}_{C} P_{B} [x]_B = P_{B \rightarrow C}  [x]_B \\
&=&  \begin{bmatrix}[r] -2 & 1 \\ 1 & 0 \end{bmatrix}  \begin{bmatrix} 1 & 2  \\ 0 & 1 \end{bmatrix} \begin{bmatrix} 3 \\1 \end{bmatrix}
     & \qquad\qquad \mathrel{\#}  \text{multiply by -1 }  \\
&=&  \begin{bmatrix}[r] -1 & -4 \\ 1 & 2 \end{bmatrix} \begin{bmatrix} 3 \\1 \end{bmatrix}
     & \qquad \qquad \mathrel{\#}  P_{B \rightarrow C} = \frac{1}{-1}  \begin{bmatrix}[r] 2 & -1 \\ -1 & 0 \end{bmatrix} \begin{bmatrix} 1 & 2  \\ 0 & 1 \end{bmatrix} =  
         \begin{bmatrix}[r] -1 & -4 \\ 1 & 2 \end{bmatrix}\\
&=&  \begin{bmatrix}[r] -7 \\ 5 \end{bmatrix}
     & \qquad\qquad \mathrel{\#}  [x]_C = P_{B \rightarrow C} [x]_B = \begin{bmatrix}[r] -1 & -4 \\ 1 & 2 \end{bmatrix} \begin{bmatrix} 3 \\1 \end{bmatrix} =  \begin{bmatrix}[r] -7 \\ 5 \end{bmatrix}

\end{array}
\end{equation*}

\section{Acknowledgements}

\newpage
\bibliographystyle{plain}
\bibliography{/Users/dmm/papers/bib/qc}



\end{document} 
