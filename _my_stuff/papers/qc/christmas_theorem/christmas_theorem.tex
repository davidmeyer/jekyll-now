\documentclass[11pt, oneside]{article}   	% use "amsart" instead of "article" for AMSLaTeX format


% \usepackage{draftwatermark}
% \SetWatermarkText{Draft}
% \SetWatermarkScale{5}
% \SetWatermarkLightness {0.9} 
% \SetWatermarkColor[rgb]{0.7,0,0}


\usepackage{geometry}                		% See geometry.pdf to learn the layout options. There are lots.
\geometry{letterpaper}                   		% ... or a4paper or a5paper or ... 
%\geometry{landscape}                		% Activate for for rotated page geometryhttps://www.washingtonpost.com/world/europe/amid-impeachment-probe-gordon-sondland-is-overseeing-a-renovation-of-his-residence-that-has-cost-1-million-in-taxpayer-money/2019/10/16/d0eece92-ef86-11e9-bb7e-d2026ee0c199_story.html?tid=sm_tw
%\usepackage[parfill]{parskip}    		% Activate to begin paragraphs with an empty line rather than an indent
\usepackage{graphicx}				% Use pdf, png, jpg, or eps� with pdflatex; use eps in DVI mode
								% TeX will automatically convert eps --> pdf in pdflat						\label{thm:integral_domain}

								% TeX will automatically convert eps --> pdf in pdflatex		
\usepackage{amssymb}
\usepackage{amsmath}
\usepackage{amsthm}
\usepackage{mathrsfs}
\usepackage[hyphens,spaces,obeyspaces]{url}
\usepackage{url}
\usepackage{hyperref}
\usepackage{subcaption}
\usepackage{authblk}
\usepackage{mathtools}
\usepackage{graphicx}
\usepackage[export]{adjustbox}
\usepackage{fixltx2e}
\usepackage{hyperref}
\usepackage{alltt}
\usepackage{color}
\usepackage[utf8]{inputenc}
\usepackage[english]{babel}
\usepackage{float}
\usepackage{bigints}
\usepackage{braket}
\usepackage{siunitx}
\usepackage{mathtools}
\usepackage{xcolor}



\usepackage[hyphenbreaks]{breakurl}

\newtheorem{thm}{Theorem}[section]
% \newtheorem{defn}[thm]{Definition}
\theoremstyle{definition}
\newtheorem{definition}{Definition}[section]
\newtheorem{proposition}{Proposition}[section]
\newtheorem{lemma}{Lemma}[section]
\newtheorem{example}{Example}[section]




\newcommand{\veq}{\mathrel{\rotatebox{90}{$=$}}}
\DeclareMathOperator{\bda}{\Big \downarrow}


\DeclareMathOperator{\E}{\mathbb{E}}
\newcommand{\argmax}{\operatornamewithlimits{argmax}}
\newcommand{\argmin}{\operatornamewithlimits{argmin}}

\title{Fermat's Christmas Theorem}
\author{David Meyer \\ dmm@\{1-4-5.net,uoregon.edu\}}

\date{Last update: December 26, 2020}


\begin{document}
\maketitle

\section*{Fermat's Christmas Theorem}

\noindent
Fermat's Christmas Theorem \cite{wiki:christmas_theorem} is a beautiful and simply stated theorem. It is called Fermat's Christmas Theorem because Fermat announced 
a proof of the theorem in a letter to Mersenne dated December 25, 1640. 

\bigskip
\noindent
The prime numbers $p$ for which the theorem is true are called Pythagorean primes. See \cite{wiki:pythagorean_primes} for more on Pythagorean primes.

\bigskip
\noindent
Fermat's Christmas Theorem (aka Fermat's theorem on sums of two squares) states that an odd prime number $p$ can be expressed as

\bigskip
\begin{center}
\scalebox{2.00} {$p = r^{2} + s^{2}$}
\end{center}
\bigskip

\noindent
where $r,s \in \mathbb{N}$, if and only if $p \equiv 1 \textrm{ (mod  $4$)}$.

\bigskip
\noindent
For example, the primes 5, 13, 17, 29, 37 and 41 are all congruent to 1 modulo 4 and can be expressed as sums of two squares in the following ways:

\begin{center}
\begin{equation*}
\begin{array}{rcll} 
5   &=& 1^{2} + 2^{2}   \\
13 &=& 2^{2} + 3^{2}   \\
17 &=&1^{2}  + 4^{2}    \\
29 &=& 2^{2} + 5^{2}    \\
37 &=& 1^{2} + 6^{2}    \\
41 &=& 4^{2} + 5^{2}
\end{array}
\end{equation*}
\end{center}

\bigskip
\noindent
A variety of proofs of Fermat's Christmas Theorem can be found in \cite{wiki:christmas_theorem_proofs}.


\bibliographystyle{plain}
\bibliography{/Users/dmm/papers/bib/qc}



\end{document} 
