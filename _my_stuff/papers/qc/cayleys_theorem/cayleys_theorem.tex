\documentclass[11pt, oneside]{article}   	% use "amsart" instead of "article" for AMSLaTeX format


% \usepackage{draftwatermark}
% \SetWatermarkText{Draft}
% \SetWatermarkScale{5}
% \SetWatermarkLightness {0.9} 
% \SetWatermarkColor[rgb]{0.7,0,0}


\usepackage{geometry}                		% See geometry.pdf to learn the layout options. There are lots.
\geometry{letterpaper}                   		% ... or a4paper or a5paper or ... 
%\geometry{landscape}                		% Activate for for rotated page geometry
%\usepackage[parfill]{parskip}    		% Activate to begin paragraphs with an empty line rather than an indent
\usepackage{graphicx}				% Use pdf, png, jpg, or eps� with pdflatex; use eps in DVI mode
								% TeX will automatically convert eps --> pdf in pdflat						
								% TeX will automatically convert eps --> pdf in pdflatex		
\usepackage{amssymb}
\usepackage{amsmath}
\usepackage{amsthm}
\usepackage{mathrsfs}
\usepackage{hyperref}
\usepackage{url}
\usepackage{subcaption}
\usepackage{authblk}
\usepackage{mathtools}
\usepackage{graphicx}
\usepackage[export]{adjustbox}
\usepackage{fixltx2e}
\usepackage{hyperref}
\usepackage{alltt}
\usepackage{color}
\usepackage[utf8]{inputenc}
\usepackage[english]{babel}
\usepackage{float}
\usepackage{bigints}
\usepackage{braket}
\usepackage{siunitx}

\theoremstyle{definition}
\newtheorem{thm}{Theorem}[section]
% \newtheorem{defn}[thm]{Definition}
\newtheorem{definition}{Definition}[section]




\newcommand{\veq}{\mathrel{\rotatebox{90}{$=$}}}
\DeclareMathOperator{\bda}{\Big \downarrow}


\DeclareMathOperator{\E}{\mathbb{E}}
\newcommand{\argmax}{\operatornamewithlimits{argmax}}
\newcommand{\argmin}{\operatornamewithlimits{argmin}}

\title{A Few Notes on Cayley's Theorem}
\author{David Meyer \\ dmm@\{1-4-5.net,uoregon.edu\}}

\date{Last update: \today}							% Activate to display a given date or no date



\begin{document}
\maketitle

\section{Introduction}
Group Theory is the study of symmetry.  Cayley's Theorem is a fundamental theorem in Group Theory, and the topic of these notes.

\bigskip
\noindent
Before diving into Cayley's Theorem, a couple of notes:
\begin{itemize}
\item The Symmetric Group $Sym(G)$ or sometimes $S_n$, where $n = |G|$ ($G$ is finite), is the set of all bijections from $G$ to itself with function composition as 
the group operation.  That is, $Sym(G) = S_n = S_{|G|} = \{f: G \rightarrow G \mid f \text{ is an bijection}\}$.
\item We use the symbol $\simeq$ (or sometimes $\cong$) to mean that the groups $G$ and $H$ are \emph{isomorphic}. That is, 
$G \simeq H \implies \exists f \mid f: G \rightarrow H$ where $f$ is a bijection and a homomorphism. See Equation \ref{eqn:homomorphism}.
 \item To show that $f$ is one-to-one, show that $f(x) = f(y) \implies x = y$.
 \item To show that $f$ is onto, pick an arbitrary $h \in H$ and show that $\exists g \in G \mid f(g) = h$.
\end{itemize}

\bigskip
\noindent
Recall also that if we have two groups $(G,*)$ and $(H,\cdot)$ we say that $(G,*)$  is \emph{isomorphic} to $(H,\cdot)$ if there exists a bijection $f:G \rightarrow H$
which satisfies the \emph{homomorphism} property:

\bigskip
\begin{equation}
f(x * y) = f(x) \cdot f(y) \;\;\; \forall  x,y \in G
\label{eqn:homomorphism}
\end{equation}

\bigskip
\noindent
That is, $f$ is a bijection (one-to-one and onto) and $f$ is also a homomorphism.


\bigskip
\noindent
Any bijective function $f$ which satisfies Equation \ref{eqn:homomorphism} is called a \emph{group} isomorphism from $G$ to $H$. The basic idea of
$(G,*)$ being isomorphic to $(H,\cdot)$ is that $(G,*)$ and $(H,\cdot)$ are "algebraically equivalent". That is, there is a one-to-one correspondence 
between elements of $G$ and elements of $H$ where the outcomes of operations on elements of $G$ are matched with the outcomes of the 
corresponding operations on the corresponding elements of $H$. 


\section{Cayley's Theorem}

\begin{thm}
\textbf{Cayley's Theorem: }If $G$ is a group then there exists a subgroup $H$ of $Sym(G)$ such that $G$ is isomorphic to $H$.
\label{thm:caley}
\end{thm}

\noindent
\textbf{Proof:} Suppose that $G$ is a group.  Then to prove Cayley's Theorem we need to find a subgroup $H$ of $Sym(G)$ and a bijective homomorphism 
$f: G \rightarrow H$. My roadmap for the proof looks like

\bigskip
\begin{enumerate}
\item Define $\phi_{a}: G \rightarrow G$ for each $a \in G$ and show that $\phi_{a}$ is a bijection 
\item Define $H = \{\phi_{a} \mid a \in G\}$ and show that $H$ is a subgroup of $Sym(G)$
\item Define $f:G \rightarrow H$ and show that $f$ is both a bijection and a homomorphism
\end{enumerate}

\bigskip
\noindent
BTW, a nice thing about the proof of Cayley's theorem is that it is a \emph{constructive} proof: the statement of the theorem is that a certain group $H$ exists. 
In the course of the proof of the theorem one can actually show not only that such an $H$ exists but also how to actually find it. We'll see an example of this
below (Section \ref{subsub:klein}).


\subsection{Define $\phi_{a}: G \rightarrow G$ for each $a \in G$ and show that $\phi_{a}$ is a bijection}
To start, for each fixed element $a \in G$ define $\phi_{a}: G \rightarrow G$ by the map $x \mapsto ax$. That is 

\begin{equation}
\phi_{a}(x) = ax \quad \forall x \in G
\label{eqn:phi}
\end{equation}


\bigskip
\noindent
Luckily it turns out that each $\phi_{a}$ is a bijection.  To see this we need to show that $\phi_a$ is one-to-one and onto. 
First,  consider that $\phi_a$  is one-to-one since

\begin{equation}
\begin{array}{llll}
\phi_{a}(x)  
&=&                 \phi_{a}(y)                        &\quad  \mathrel{\#} \text{to show $\phi_a$ is 1-to-1 show } \phi_a(x) = \phi_a(y) \Rightarrow x = y      \\                                                                                                                     
&\Rightarrow& ax = ay                            &\quad  \mathrel{\#} \text{definition of $\phi_a(x)$ (Equation \ref{eqn:phi})}                                                                                     \\
&\Rightarrow& a^{-1}(ax) = a^{-1}(ay)     &\quad  \mathrel{\#} \text{multiply by $a^{-1}$};  a \in G \text{ \& $G$ a group} \Rightarrow a^{-1} \in G \\
&\Rightarrow&  (a^{-1}a)x = (a^{-1}a)y    &\quad  \mathrel{\#} \text{multiplication is associative}                                                                             \\
&\Rightarrow&  x = y                                &\quad  \mathrel{\#} a^{-1}a = 1 
\end{array}
\label{eqn:1-to-1}
\end{equation}

\bigskip
\noindent
So $\phi_{a}$ is one-to-one.

\bigskip
\noindent
\textbf{Aside on cancellation laws:} Note that in (\ref{eqn:1-to-1}) we used the fact that $a \in H$ and that $H$ is a group so $a^{-1} \in H$.
Here we have  $a^{-1}a = 1$, which essentially gives us a \emph{cancellation law}\footnote{Note that having a cancellation law is equivalent to saying 
there are no \emph{zero divisors}.}; in  (\ref{eqn:1-to-1}) this allows us to "cancel" the $a$ on both sides. Now, what if we don't have 
access to multiplicative inverses? We might be faced with this situation if we have a ring, where we don't in general have multiplicative 
inverses\footnote{A ring with multiplicative inverses is called a division ring (or skew field).  Example: the quaternions.}. So if we don't  have multiplicative 
inverses how do we go about showing that something is one-to-one? 

\bigskip
\noindent
One  approach is to factor out $a$ and note that by assumption, $a \neq 0$
so something else must be. For example

\begin{equation*}
\begin{array}{llll}
\phi_{a}(x)  
&=&                 \phi_{a}(y)                         &\quad  \mathrel{\#} \text{to show $\phi_a$ is 1-to-1 show that } \phi_a(x) = \phi_a(y) \Rightarrow x = y \\
&\Rightarrow& ax = ay                            &\quad  \mathrel{\#} \text{definition of $\phi_a(x)$ (\text{Equation \ref{eqn:phi})}}            \\
&\Rightarrow& ax-ay = 0                         &\quad  \mathrel{\#} \text{subtract $ay$ from both sides} \\
&\Rightarrow&  a (x-y) = 0                      &\quad  \mathrel{\#} \textit{factor out a}                            \\
&\Rightarrow&  x - y     = 0                       &\quad  \mathrel{\#} a \neq 0 \text{ by assumption so } x - y = 0  \\
&\Rightarrow&  x = y                               &\quad  \mathrel{\#} \text{so $\phi_{a}$ is one-to-one}
\end{array}
\end{equation*}

\bigskip
\noindent
Getting back to showing that $\phi_{a}$ is a bijection, we next need to show that $\phi_{a}$ is onto. To do this pick an arbitrary $y \in G$ (here $G$ is the range).  
Then $a^{-1}y \in G$ (here $G$ is the domain) and so $\phi_{a}(a^{-1}y) = a(a^{-1}y)$. Since multiplication is associative we have 
$\phi_{a}(a^{-1}y) = a(a^{-1}y) = (aa^{-1})y = y$. So $\phi_{a}$ is onto and hence $\phi_{a}$ is a bijection.
 
 
\subsection{Define $H = \{\phi_{a} \mid a \in G\}$ and show that $H$ is a subgroup of $Sym(G)$}
Now we can define $H = \{\phi_a \mid a \in G\}$. Since each element of $H$ is a bijection from $G$ to $G$ and since $Sym(G)$ is the set of all
bijections from $G$ to $G$ we know that $H \subseteq G$. To show that $H$ is a subgroup of $Sym(G)$ we also need to show that $H$ is closed under function 
composition and  inversion.  

\bigskip
\noindent
To show closure under function composition we need to show that $\alpha, \beta \in H \Rightarrow \alpha \circ \beta \in H$.  To see this consider 
$\alpha, \beta \in H$. Then there exists $a \in G$ such that $\alpha = \phi_{a}$. Similarly there exists $b \in G$ such that $\beta = \phi_b$. So we 
know that

\begin{equation}
\alpha \circ \beta = \phi_a \circ \phi_b
\label{eqn:ab}
\end{equation}
 
\bigskip
\noindent
and so for any $x \in G$ we have

\bigskip
\begin{equation}
\begin{array}{rcll}
(\alpha \circ \beta)(x)  
&=& (\phi_a \circ \phi_b)(x)                &\quad  \mathrel{\#}  \alpha \circ \beta = \phi_a \circ \phi_b \text{ (Equation \ref{eqn:ab})} \\
&=& \phi_{a}(\phi_{b}(x))                    &\quad  \mathrel{\#} \text{definition of function composition}          \\
&=& \phi_{a}(bx)                                 &\quad  \mathrel{\#} \phi_b(x) = bx \text{ (definition of $\phi_b$)}  \\
&=& a(bx)                                           &\quad  \mathrel{\#} \phi_a(x) = ax \text{ (definition of $\phi_a$)}  \\
&=& (ab)x                                           &\quad  \mathrel{\#} \text{multiplication is associative}                 \\
&=& \phi_{ab}(x)                                 &\quad  \mathrel{\#} \phi_{g}(x) =gx  \text{ where $g=ab$}
\end{array}
\label{eqn:function_composition}
\end{equation}

\bigskip
\noindent
So $\alpha \circ \beta = \phi_a \circ \phi_b = \phi_{ab}$. Since $ab \in G$ ($G$ is closed under multiplication) we know that $\phi_{ab} \in H$. Now
we have $\phi_{ab} \in H$ and $\alpha \circ \beta = \phi_{ab}$ which together imply that  $\alpha \circ \beta \in H$. So $H$ is closed under function composition.

\bigskip
\noindent
To show that $H$ is closed under inversion we need to show that $\alpha \in H \Rightarrow \alpha^{-1} \in H$.
To see this consider $\alpha \in H$.  Then there exists an $a \in G$ such that $\alpha = \phi_a$. 
Since $a \in G$ and since $G$ is a group, $a^{-1} \in G$ and so $\phi_{a^{-1}} \in H$. Note further that for any $x \in G$

\bigskip
\begin{equation}
\begin{array}{rcll}
(\phi_{a^{-1}} \circ \phi_a)(x)
&=& \phi_{a^{-1}}(\phi_a(x))        &\quad  \mathrel{\#} \text{definition: $(f \circ g)(x) = f(g(x))$}        \\
&=& \phi_{a^{-1}}(ax)                   &\quad  \mathrel{\#} \phi_a(x) = ax                                                \\
&=& a^{-1}(ax)                             &\quad  \mathrel{\#} \text{definition: } \phi_{a^{-1}}(x) = a^{-1}(x)  \\
&=& (a^{-1}a)x                             &\quad  \mathrel{\#} \text{multiplication is associative}                  \\
&=& x                                           &\quad  \mathrel{\#} a^{-1}a = 1
\end{array}
\label{eqn:comp1}
\end{equation}


\bigskip
\noindent
and  

\bigskip
\begin{equation}
\begin{array}{rcll}
(\phi_{a} \circ \phi_{a^{-1}})(x)
&=& \phi_{a}(\phi_{a^{-1}}(x))      &\quad  \mathrel{\#} \text{definition: $(f \circ g)(x) = f(g(x))$}  \\
&=& \phi_{a}(a^{-1}x)                   &\quad  \mathrel{\#} \phi_{a^{-1}}(x) = a^{-1}x                          \\
&=& a(a^{-1}x)                             &\quad  \mathrel{\#} \text{definition: }  \phi_{a}(x) = ax             \\
&=& (a a^{-1})x                            &\quad  \mathrel{\#} \text{multiplication is associative}             \\
&=& x                                           &\quad  \mathrel{\#} aa^{-1} = 1
\end{array}
\label{eqn:comp2}
\end{equation}

\bigskip
\noindent
Recall that if a function $f$ is a bijection we know $(f^{-1} \circ f)(x) = (f \circ f^{-1})(x) = x$.  From (\ref{eqn:comp1}) 
and  (\ref{eqn:comp2}) we see that  $\phi_{a^{-1}}$ is the inverse of $\phi_a$. More specifically $\phi_{a^{-1}} = \phi^{-1}_{a}$.
Since $\alpha = \phi_a$,  $\alpha^{-1} = \phi^{-1}_a = \phi_{a^{-1}}  \in H$. So $H$ is closed under inversion.



\subsection{Define $f:G \rightarrow H$ and show that $f$ is a homomorphic bijection}
We still need to show a homomorphic bijection $f$ from $G$ to $H$. One way to do this is to define $f(g) = \phi_{g}$ for all $g \in G$. 
Then to show that $f$ is a bijection we need to show that $f$ is both  one-to-one and onto.

\bigskip
\noindent
To see that $f$ is one-to-one consider

\begin{equation*}
\begin{array}{rcll}
f(a)
&=& f(b)                                                  &\quad  \mathrel{\#} \text{to show $f$ is 1-to-1 show that } f(a) = f(b) \Rightarrow a = b \\
&\Rightarrow& \phi_a(x) = \phi_b(x)       &\quad  \mathrel{\#} \text{definition of $f(g): f(g) = \phi_g \text{ for all } g \in G$} \\
&\Rightarrow& \phi_a(a) = \phi_b(a)       &\quad  \mathrel{\#} \text{evaluate at $a \in G$} \\
&\Rightarrow& aa = ba                           &\quad  \mathrel{\#} \text{definition of $\phi_{g}: \phi_{g}(x) = gx  \text{ for all } g \in G$} \\
&\Rightarrow& (aa)a^{-1} = (ba)a^{-1}    &\quad  \mathrel{\#} \text{multiply by $a^{-1}$}; a \in G \text{ and $G$ a group} \Rightarrow a^{-1} \in G \\
&\Rightarrow& a(aa^{-1}) = b(aa^{-1})    &\quad  \mathrel{\#} \text{multiplication is associative}            \\
&\Rightarrow& a = b                               &\quad  \mathrel{\#} aa^{-1} = 1    \\
\end{array}
\end{equation*}

\bigskip
\noindent
So $f$ is one-to-one. 

\bigskip
\noindent
To show that $f$ is onto, choose a $\alpha \in H$. Then there exists an $a \in G$ such that $\alpha = \phi_a$. However we know that 
$f(a) = \phi_a$ and  $\phi_a = \alpha$ so we know that  $f(a) = \alpha$. So $f$ in onto and since we saw that $f$ is one-to-one, $f$ is a bijection.


\bigskip
\noindent
Finally, to show that $f$ is also a homomorphism we want to show that $f(ab) = f(a) \circ f(b)$. To see this consider that for any $a,b \in G$ we have 

\bigskip
\begin{equation*}
\begin{array}{rcll}
f(ab)
&=& \phi_{ab}                          &\quad  \mathrel{\#} \text{definition of $f$}                                           \\
&=& \phi_{a} \circ \phi_{b}       &\quad  \mathrel{\#} \text{Equation  \ref{eqn:function_composition}}   \\
&=& f(a) \circ f(b)                     &\quad  \mathrel{\#} \text{definition of $f$}
\end{array}
\end{equation*}

\bigskip
\noindent
So  $f$ is a homomorphism. 

\bigskip
\noindent
This completes the proof of  Cayley's Theorem.


\section{Examples}

% \newpage

\subsection{($\mathbb{Z}_4,+) \rightarrow (G,\cdot)$}
Let $(\mathbb{Z}_4,+)$ be the set $\mathbb{Z}_4 = \{0,1,2,3\}$ with addition modulo 4 and let $(G,\cdot)$ be the set $G = \{1,-1,i,-i\}$ (the fourth roots of unity) 
with the usual multiplication on $\mathbb{C}$. Then $(\mathbb{Z}_4,+) \simeq (G,\cdot)$. To see that $\mathbb{Z}_4$ is isomorphic to $G$, let 
$f: \mathbb{Z}_4 \rightarrow G$ be the bijection

\begin{equation*}
\begin{array}{lccr}
& 0 &\longrightarrow&  1 \\
& 1 &\longrightarrow&  i \\
& 2 &\longrightarrow&  -1 \\
& 3 &\longrightarrow&  -i \\
\end{array}
\end{equation*}

\bigskip
\noindent
Here are the Cayley tables for $\mathbb{Z}_4$ and $G$:

\bigskip
\begin{table}[ht]
\begin{minipage}{0.5\linewidth} 
\centering
\begin{tabular}{c | c c c c c} 
+ & 0 & 1  & 2 & 3 \\
\cline{1-5}
0 & 0 & 1 & 2 & 3 \\
1 & 1 & 2 & 3 & 0 \\
2 & 2 & 3 & 0 & 1 \\
3 & 3 & 0 & 1 & 2
\end{tabular}
\caption{$\mathbb{Z}_4$} 
\label{tab:z4}
\end{minipage}
%
 \begin{minipage}{0.5\linewidth} 
\centering
\begin{tabular}{r | r r r r} 
$\cdot$ & $1$ & $i$  & $-1$ & $-i$ \\
\cline{1-5}
$1$   & $1$   & $i$     & $-1$   & $-i$  \\
$i$    & $i$    & $-1$   & $-i$    & $1$  \\
$-1$  & $-1$  & $-i$    & $1$    & $i$    \\
$-i$   &  $-i$  & $1$     & $i$     & $-1$
\end{tabular}
\caption{$G$} 
\label{tab:G}
\end{minipage}
\end{table}

\bigskip
\noindent
To show that $f$ is an isomorphism we need to show that $f$ is a homomorphism, that is, that 
$f(x+y) = f(x) \cdot f(y)$.  Since there are only $n^2 = 4^2 = 16$ values for $f(x+y)$  we can just 
enumerate them:

\begin{equation*}
\begin{array}{lllll}
f(0+0) &= f(0) &= 1  &= 1 \cdot 1     &=  f(0) \cdot f(0) \\
f(0+1) &= f(1) &= i   &= 1 \cdot i      &=  f(0) \cdot f(1) \\
f(0+2) &= f(2) &= -1 &= 1 \cdot -1   &=  f(0) \cdot f(2)  \\
f(0+3) &= f(3) &= -i  &= 1 \cdot  -i    &=  f(0) \cdot f(3) \\
f(1+0) &= f(1) &= i   &=  i \cdot 1      &=  f(1) \cdot f(0) \\
f(1+1) &= f(2) &= -1 &=  i \cdot i       &=  f(1) \cdot f(1)  \\
f(1+2) &= f(3) &= -i  &=  i \cdot  -1    &=  f(1) \cdot f(2) \\
f(1+3) &= f(0) &= 1  &=  i \cdot -i      &=  f(1) \cdot f(3) \\
f(2+0) &= f(2) &= -1 &=  -1 \cdot 1    &=  f(2) \cdot f(0)  \\
&& \qquad \vdots                                                           \\
f(3+3) &= f(2) &= -1 &=  -i \cdot -i   &=  f(3) \cdot f(3)  \\
\end{array}
\end{equation*}

\bigskip
\noindent
So the bijection $f: \mathbb{Z}_4 \rightarrow G$ above is a homomorphism and hence $f$ is a group isomorphism.

% \newpage 

\subsection{The Klein 4-group}
\label{subsub:klein}

The Klein 4-group is the group $K = \{e,a,b,c\}$ where $e$ is the identity element and the group operation
is defined by the Cayley table below (Table \ref{tab:klein}).

\bigskip
\begin{table}[H]
\centering
\begin{tabular}{c | c c c c c} 
& $e$ & $a$ & $b$ & $c$ \\
\cline{1-5}
$e$ & $e$ & $a$ & $b$ & $c$ \\
$a$ & $a$ & $e$ & $c$ & $b$ \\
$b$ & $b$ & $c$ & $e$ & $a$ \\
$c$ & $c$ & $b$ & $a$ & $e$
\end{tabular}
\caption{The Klein 4-group Operation} 
\label{tab:klein}
\end{table}

\bigskip
\noindent
Here $K$ is \emph{not} isomorphic to $\mathbb{Z}_{4}$. To see this notice that there are 24 bijections from $\mathbb{Z}_4$ and $K$:
$|K| = |\mathbb{Z}_4|  = 4$ so  there are $n! = 4! = 24$ possible bijections from $\mathbb{Z}_4$ to $K$. Since we need 
$f(0) = e$ that leaves $3! = 6$  bijections that could be homomorphisms.  For example, consider the bijection

\begin{equation*}
\begin{array}{ccr}
0 &\longrightarrow&  e \\
1 &\longrightarrow&  a \\
2 &\longrightarrow&  c \\
3 &\longrightarrow&  b 
\end{array}
\end{equation*}

\bigskip
\noindent
This bijection is not a homomorphism since $f(1+3) = f(4) = f(0) = e$ while $f(1) \cdot f(3) = ab = c$, so $f(1+3) \neq f(1) \cdot f(3)$. 

\bigskip
\noindent
One way to see that $\mathbb{Z}_4$ is not isomorphic to $K$ is to recognize that every element of $K$ satisfies the equation $x \cdot x =e$  
(a key property of the Klein 4-group). However not every element of $\mathbb{Z}_4$ satisfies the equation $x + x = 0$. 

\bigskip
\noindent
This gives a clue as to how to prove,  by contradiction,   that $\mathbb{Z}_4$ is not isomorphic to $K$. Specifically,
suppose that $\mathbb{Z}_4$ is isomorphic to $K$. Then there exists a bijection 
$f: \mathbb{Z}_4 \rightarrow K$ such that $f(x + y) = f(x) \cdot f(y)$ for all $x,y \in \mathbb{Z}_4$. Well,
we know by definition that $f(0) = e$ and since $f$ is one-to-one we also know that $f(1) \neq e$. Since 
$f$ is a homomorphism we also know that

\begin{equation*}
f(1+1) = f(1) \cdot f(1)
\end{equation*}


\bigskip
\noindent
However, since $f(1) \in K$ and all elements of $K$ satisfy $x \cdot x = e$ we can conclude that $f(1) \cdot f(1) = e$, so $f(1+1) = f(2) = e$. Now we
have $f(0) = e$ \emph{and}  $f(2) = e$ which is a contradiction since we assumed that  $f$ was one-to-one. So the original assumption that 
$\mathbb{Z}_4$ is isomorphic to $K$ is false.

\bigskip
\noindent
Ok, but Cayley's Theorem says there is a subgroup $H$ of $S_4$ which is isomorphic to $K$. How to find $H$? Since as noted above Cayley's Theorem is constructive,
we should be able to follow the approach used in the proof to find $H$. Here we let $H =\{\phi_{e}, \phi_{a}, \phi_{b}, \phi_{c}\}$ where, for all $x \in K$

\bigskip
\begin{equation*}
\begin{array}{llll}
\phi_{e}(x) = ex   &&&\quad  \mathrel{\#}  \phi_{e}(x) = x  \\     
\phi_{a}(x) = ax  \\    
\phi_{b}(x) = bx  \\       
\phi_{c}(x) = cx  \\                                        
\end{array}
\end{equation*}

\bigskip
\noindent
Now we can rewrite the Cayley table for the Klein 4-group (Table \ref{tab:klein}) as 
\bigskip
\begin{equation*}
\begin{array}{llll}
\phi_{e} = \begin{pmatrix} e & a & b & c \\ e & a & b & c \end{pmatrix}  \\
\phi_{a} = \begin{pmatrix} e & a & b & c \\ a & e & c & b \end{pmatrix}  \\
\phi_{b} = \begin{pmatrix} e & a & b & c \\ b & c & e & a \end{pmatrix}  \\
\phi_{c} = \begin{pmatrix} e & a & b & c \\ c & b & a & e \end{pmatrix}                                   
\end{array}
\end{equation*}

\bigskip
\noindent
Now, if we relabel $K$ by the bijection

\begin{equation*}
\begin{array}{l c c c c c c c c c c c c}
 e      & a       & b      & c \\
 \bda  & \bda & \bda & \bda \\
 1     & 2       & 3      & 4
\end{array}
\end{equation*}

\bigskip
\noindent
we can represent $K$ in cyclic notation:

\bigskip
\begin{equation*}
\begin{array}{llll}
\phi_{e} = \begin{pmatrix} 1 & 2 & 3 & 4  \\ 1 & 2 & 3 & 4  \end{pmatrix}  = (1)         \\
\phi_{a} = \begin{pmatrix} 1 & 2 & 3 & 4  \\ 2 & 1 & 4 & 3 \end{pmatrix}   = (12)(34) \\
\phi_{b} = \begin{pmatrix} 1 & 2 & 3 & 4  \\  3 & 4 & 1 & 2 \end{pmatrix}  = (13)(24)  \\
\phi_{c} = \begin{pmatrix} 1 & 2 & 3 & 4  \\  4 & 3 & 2 & 1 \end{pmatrix}   = (14)(23)                                 
\end{array} 
\end{equation*}

\bigskip
\noindent
Now we can see that $K \simeq H$ where $H = \{(1), (12)(34), (13)(24), (14)(23)\}$. That is, 
$f:K\rightarrow H$ is the bijection

\bigskip
\begin{equation*}
\begin{array}{c c c c c}
f: &  1   & 2        & 3        & 4 \\
   & \bda & \bda     & \bda     & \bda \\
   & (1)  & (12)(34) & (13)(24) &  (14)(23)
\end{array}
\end{equation*}

\bigskip
\noindent
The Klein 4-group $K$ has many other interesting properties, including

\begin{itemize}
\item $K$ is the smallest non-cyclic group
\item $K$ is the underlying group of the four-element field
\item $K$ is the symmetry group of a non-square rectangle 
\item $K$ is the group of bitwise exclusive or operations on two-bit binary values
\item $K = \mathbb{Z}_2 \times \mathbb{Z}_2$, the direct product of two copies of the cyclic group of order 2
\end{itemize}


\section{Acknowledgements}

\newpage
\bibliographystyle{plain}
\bibliography{/Users/dmm/papers/bib/qc}
\end{document} 
