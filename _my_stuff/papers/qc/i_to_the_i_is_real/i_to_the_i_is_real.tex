\documentclass[11pt, oneside]{article}   	% use "amsart" instead of "article" for AMSLaTeX format


% \usepackage{draftwatermark}
% \SetWatermarkText{Draft}
% \SetWatermarkScale{5}
% \SetWatermarkLightness {0.9} 
% \SetWatermarkColor[rgb]{0.7,0,0}


\usepackage{geometry}                		% See geometry.pdf to learn the layout options. There are lots.
\geometry{letterpaper}                   		% ... or a4paper or a5paper or ... 
%\geometry{landscape}                		% Activate for for rotated page geometry
%\usepackage[parfill]{parskip}    		% Activate to begin paragraphs with an empty line rather than an indent
\usepackage{graphicx}				% Use pdf, png, jpg, or eps� with pdflatex; use eps in DVI mode
								% TeX will automatically convert eps --> pdf in pdflat						
								% TeX will automatically convert eps --> pdf in pdflatex		
\usepackage{amssymb}
\usepackage{mathrsfs}
\usepackage{hyperref}
\usepackage{url}
\usepackage{subcaption}
\usepackage{authblk}
\usepackage{amsmath}
\usepackage{mathtools}
\usepackage{graphicx}
\usepackage[export]{adjustbox}
\usepackage{fixltx2e}
\usepackage{hyperref}
\usepackage{alltt}
\usepackage{color}
\usepackage[utf8]{inputenc}
\usepackage[english]{babel}
\usepackage{float}
\usepackage{bigints}
\usepackage{braket}
\usepackage{siunitx}

%
% so you can do e.g., \begin{bmatrix}[r] (or [c] or [l])
%

\makeatletter
\renewcommand*\env@matrix[1][c]{\hskip -\arraycolsep
  \let\@ifnextchar\new@ifnextchar
  \array{*\c@MaxMatrixCols #1}}
\makeatother

\newcommand{\argmax}{\operatornamewithlimits{argmax}}
\newcommand{\argmin}{\operatornamewithlimits{argmin}}

\begin{document}

\title {Is $i^i$ a real number?}
\author{David Meyer \\ dmm@\{1-4-5.net,uoregon.edu\}}

\date{Last update: July 25, 2018}							% Activate to display a given date or no date
\maketitle


\begin{equation*}
\begin{array}{lllll}
e^{ix}
&=& \cos x + i \sin x                                                                                                   &\qquad \qquad  \mathrel{\#} \text{Euler's formula}                                                      \\ 
&\Rightarrow& e^{i \frac{\pi}{2}}  = \cos \frac{\pi}{2} + i \sin \frac{\pi}{2}                    &\qquad  \qquad  \mathrel{\#} \text{set $x = \frac{\pi}{2}$}                                            \\
&\Rightarrow& e^{i \frac{\pi}{2}}  = 0 +  i \cdot 1                                                       &\qquad  \qquad  \mathrel{\#} \cos \frac{\pi}{2} = 0 \text{ and} \sin \frac{\pi}{2} = 1       \\
&\Rightarrow& e^{i \frac{\pi}{2}}  =  i                                                                         &\qquad  \qquad  \mathrel{\#} \text{simplify}                                                                  \\
&\Rightarrow& (e^{i \frac{\pi}{2}})^i =  i^i                                                                   &\qquad  \qquad  \mathrel{\#} \text{raise both sides to $i$}                                           \\
&\Rightarrow& e^{\frac{i^2 \pi}{2}} =  i^i                                                                    &\qquad  \qquad  \mathrel{\#}  (x^m)^n = x^{mn}                                                           \\
&\Rightarrow& e^{- \frac {\pi}{2}} =  i^i                                                                      &\qquad  \qquad  \mathrel{\#} i^2 = -1                                                                            \\
&\Rightarrow& e^{- \frac {\pi}{2}} \in \mathbb{R} \Rightarrow i^i \in \mathbb{R}       &\qquad  \qquad  \mathrel{\#} \text{$i^i$ is a real number}                   
                                                        
                                                        
                                                      
\end{array}
\end{equation*}


\end{document} 

