\documentclass[11pt, oneside]{article}   	% use "amsart" instead of "article" for AMSLaTeX format


% \usepackage{draftwatermark}
% \SetWatermarkText{Draft}
% \SetWatermarkScale{5}
% \SetWatermarkLightness {0.9} 
% \SetWatermarkColor[rgb]{0.7,0,0}


\usepackage{geometry}                		% See geometry.pdf to learn the layout options. There are lots.
\geometry{letterpaper}                   		% ... or a4paper or a5paper or ... 
%\geometry{landscape}                		% Activate for for rotated page geometry
%\usepackage[parfill]{parskip}    		% Activate to begin paragraphs with an empty line rather than an indent
\usepackage{graphicx}				% Use pdf, png, jpg, or eps� with pdflatex; use eps in DVI mode
								% TeX will automatically convert eps --> pdf in pdflat						
								% TeX will automatically convert eps --> pdf in pdflatex		
\usepackage{amssymb}
\usepackage{amsmath}
\usepackage{amsthm}
\usepackage{mathrsfs}
\usepackage{hyperref}
\usepackage{url}
\usepackage{subcaption}
\usepackage{authblk}
\usepackage{mathtools}
\usepackage{graphicx}
\usepackage[export]{adjustbox}
\usepackage{fixltx2e}
\usepackage{hyperref}
\usepackage{alltt}
\usepackage{color}
\usepackage[utf8]{inputenc}
\usepackage[english]{babel}
\usepackage{float}
\usepackage{bigints}
\usepackage{braket}
\usepackage{siunitx}


\newtheorem{thm}{Theorem}[section]
% \newtheorem{defn}[thm]{Definition}
\theoremstyle{definition}
\newtheorem{definition}{Definition}[section]
\newtheorem{lemma}{Lemma}[section]



\newcommand{\veq}{\mathrel{\rotatebox{90}{$=$}}}
\DeclareMathOperator{\bda}{\Big \downarrow}


\DeclareMathOperator{\E}{\mathbb{E}}
\newcommand{\argmax}{\operatornamewithlimits{argmax}}
\newcommand{\argmin}{\operatornamewithlimits{argmin}}

\title{A Few Notes on Group, Ring and Field Theory (WIP)}
\author{David Meyer \\ dmm@\{1-4-5.net,uoregon.edu\}}

\date{Last update: September 8, 2016}							% Activate to display a given date or no date



\begin{document}
\maketitle

\section{Introduction}
Group Theory is the study of symmetry. Symmetry takes many different forms. Here we'll focus on the basic theory and use that to motivate a 
few examples

\section{Definitions}
The pair $(G,\circ)$ with the following four properties is called a group:

\begin{enumerate}
\item \textbf{Closure:} $x, y \in G \implies x \circ y \in G$  \\
In words: A group is closed under the group operation $\circ$. $\circ$ is sometimes called "group multiplication" (or even "multiplication") even 
though it might be some other operation (such as addition).

\item \textbf{Associativity:} $x,y,z \in G \implies  (x \circ y) \circ z = x \circ (y \circ z)$

\item \textbf{Identity:} $\exists e \in G \;  \text{s.t.} \;  \forall x \in G \; x \circ e = e \circ x = x$

\item \textbf{Inverse:} $\forall x \in G \;  \exists x^{-1} \in G  \;  \text{s.t.} \;  x \circ x^{-1} = x^{-1} \circ x = e$
\end{enumerate}

\bigskip
\noindent
The point here is that a group is a device that measures symmetry. 

\bigskip
\noindent
Before doing an example, we need to define the \emph{order} of a group
element. Specifically, the order of a group element\footnote{Not to be confused with the order of a group, which for finite $G = \mid G \mid$.}
is the smallest integer $n$, if it exists, such that  $x^n = e$. Note that the relationship between the order of an element and the order of a 
group is that the order of the subgroup $\langle x \rangle$ generated by $x$ is $n$. That is, $x^n = e \implies | \langle x \rangle | = n$.

\bigskip
\noindent
Now, consider the  symmetries of a triangle shown in Figure \ref{fig:symmetries_of_a_triangle}.  This forms a group called $S_3$ where
the group operation $\circ$ is function composition. Here we name the vertices of the triangle with "labels" from the set $\{1,2,3\}$. In this 
case our group has six elements, as shown using cyclic notation in Table \ref{tab:s3}.

\bigskip
\begin{figure}
\center{\includegraphics[scale=0.7] {images/s3.png}}
\caption{Symmetries of a Triangle}
\label{fig:symmetries_of_a_triangle}
\end{figure}


\bigskip
\begin{table}[H]
\centering
\begin{tabular}{c | |c| c | c | c }
Element & Type & Action & Inverse & Order \\
\hline
e         &  No Op       & No Op                                         & e & 1               \\
(1 2)    & Reflection  & Fix 3, swap 1 \& 2                        & (1 2) & 2         \\
(2 3)    & Reflection  & Fix 1, swap 2 \& 3                        & (2 3) & 2         \\
(1 3)    & Reflection  &  Fix 2, swap 1 \& 3                       & (1 3) & 2         \\
(1 2 3) & Rotation     & Rotate CW by $\frac{2 \phi}{3}$    & (1 3 2) & 3      \\
(1 3 2) & Rotation     & Rotate CCW by $\frac{2 \phi}{3}$  & (1 2 3) & 3
\end{tabular}
\caption{The Group $S_3$}
\label{tab:s3}
\end{table}

\noindent
A group is called \emph{Abelian} or \emph{commutative} if the group operation $\circ$ commutes. That is, $\forall x,y \in G$  $x \circ y = y \circ x$. Note
that in our example above, $S_3$ is not Abelian. You can see this if you consider: Let $x = (1 2)$ and $y = (2 3)$. Then $x \circ y \neq y \circ x$ since

\begin{flalign*}
(1 2) \circ (2 3) &= (1 2 3) \\
(2 3) \circ (1 2) &= (1 3 2) 
\end{flalign*}

\noindent
So $S_3$ is not Abelian. 

\bigskip
\noindent
One other example:  Let $X$ be any set and $G =\{\text{bijections } f: X \rightarrow X \}$ with group operation function composition. Then $(G, \circ)$ is a group. To show
this, we need to show the following four things:

\begin{enumerate}
\item \textbf{Closure under $\circ$:} The composition of bijections is a bijection
\item \textbf{Associativity:} $(f \circ g) \circ h = f \circ (g \circ h)$
\item \textbf{Identity: } $I(x) = x$
\item \textbf{Inverse:} $y = f(x) \Leftrightarrow f^{-1}(y) = x$
\end{enumerate}


\bigskip
\noindent
One other definition we'll need here: A non-empty subset $H \subseteq G$ is a \emph{subgroup}, denoted $H \leq  G$, iff 

\begin{enumerate}
\item \textbf{Closure:} $x,y \in H  \implies x \circ y \in H$
\item \textbf{Inverses:} $x \in H \implies x^{-1} \in H$
\item \textbf{Identity:}  $e \in H$  (implies $H$ non-empty)
\end{enumerate}

\section{Subgroups, Equivalence Relations, and Cosets}

Let $H \leq G$. $H$ induces an equivalence relation $\sim$ on $G$ as follows: For a group $G$ and a subgroup $H \leq G$

\begin{equation}
x \sim y \; \text{iff} \; x = yh \;\text{for some}\;  h \in H
\label{eqn:er}
\end{equation}

\bigskip
\noindent
The equivalence relation $\sim$ implies $H$ partitions $G$. In particular, fix some $x \in G$. Then

\bigskip
\begin{equation*}
E_x = \{y \in G \mid x \sim y \}
\end{equation*}

\bigskip
\noindent
$E_x$ is sometimes denoted $[x]$.  In this case the notation $[x] = \{y \in G \mid x \sim y\} \implies x^{-1}y \in H$ or $y \in xH$.

\bigskip
\noindent
Note that the $E_x$'s form a disjoint union: $\underset{x \in G}{\cup}  E_x = G$ and either $E_x = E_y$ or $E_x \cap \E_y = \emptyset$. This
will become important later when we consider Lagrange's Theorem (Theorem \ref{thm:lagrange}).


\bigskip
\noindent
More generally, 

\begin{equation*}
\begin{array}{rcll}
E_x
&=& \{y \in G \mid x \sim y\}                                               &\quad  \mathrel{\#} \text{definition of $E_x$}   \\
&=& \{y \in G \mid y = xh \text{ for some } h \in H\}            &\quad  \mathrel{\#} \text{definition of $\sim$}  \\
&=& \{xh \mid h \in H\}                                                        &\quad  \mathrel{\#} \text{simplify}                    \\
&=& xH                                                                               &\quad  \mathrel{\#} xH \text{ is the left \emph{coset} of H for x}                    \\
\end{array}
\end{equation*}

\bigskip
\noindent
Summary: $E_x = [x] = \{xh \mid h \in H\} = xH$. $xH$ is called a \emph{left coset} of $H$ for $x$.

\bigskip
\noindent
Somewhat surprisingly, $H$ and all of its cosets have the same cardinality. That is,  $|H| = |xH|\;  \forall x \in G$.  The proof
of this is pretty straightforward. Here we just need to construct a bijection $H \rightarrow xH$. We can do this by fixing a $x \in G$ and considering 
$l_x: H \rightarrow xH$ by the map $h \mapsto xh$. Then we need to show that $l_x$ is one-to-one and onto.

\begin{itemize}
\item \textbf{one-to-one} \\
For one-to-one we want to show that  $l_x(h_1) = l_x(h_2) \implies h_1 = h_2$. In this case we can cancel on the left: 
$l_x(h_1) = l_x(h_2) \implies xh_1 = xh_2 \implies h_1 = h_2$.
So one-to-one.
\item \textbf{onto} \\
Consider some $z \in xH$. Then $z = xh$ for some $h \in H$, so $l_x(x) = z$ and $l_x$ is onto.
\end{itemize}

\noindent
So $l_x: H \rightarrow xH$ is a bijection and therefore $H$ and all of its cosets $xH$ have the same cardinality, namely, $|H|$.


\bigskip
\noindent
A couple useful definitions:

\begin{itemize}
\item $| G |$ is the \emph{order} of $G$. If $G$ is finite then $|G| = \text{\# of elements in } G$.
\item $| x |$ is the \emph{order} of $x$.  $|x|$ is the smallest positive integer $n$, if it exists, such that $x^n = e$.
\item $[G:H]$ is the \emph{index} of $H$ in $G$. $[G:H] =  \text{\# of cosets for $H$ (left or right) in $G$}$.
\end{itemize}

\bigskip
\noindent
With this machinery we can state and prove Lagrange's Theorem:

\bigskip
\noindent
\begin{thm}
\textbf{Lagrange's Theorem:}  Let  $G$ be a finite group and let $H$ be a subgroup of $G$. Then $|H|$ divides $|G|$. 
\label{thm:lagrange}
\end{thm}

\bigskip
\noindent
\textbf{Proof:} Recall that the left cosets of H in G form a disjoint union of G. We also know that $|H| = |xH|$. Now suppose that there are $k$ cosets. 
Then $|G| = k \cdot |H|$. Since $|G|$, $|H|$, and $k$ are integers $|H|$ divides $|G|$. $\square$

\bigskip
\noindent
Note that if $G$ is finite, $[G:H] = \frac{|G|}{|H|}$. A nice corollary of Lagrange's Theorem is that if $x \in G$, then $|x|$ divides $|G|$.

\section{Normal and Quotient Subgroups}
Let $G$ be any group. Then a subgroup $N \subseteq G$ is called \emph{normal}, written $N  \lhd G$, iff $\forall x \in G \; xN = Nx$. Equivalently,
$\forall x \in G \; N = xNx^{-1}$ (multiply on the right by $x^{-1}$). Also equivalently $\forall x \in G \;  \forall n \in N \; x n x^{-1} \in N$. The last expression
is called conjugation by $x$.


\section{Group Actions}
The action of a group is a formal way of interpreting the manner in which the elements of the group correspond to (usually linear) transformations of some space 
in a way that preserves the structure of that space. Common examples of spaces that groups act on are sets, vector spaces, and topological spaces. 
Actions of groups on vector spaces are called representations of the group (this is the subject of representation theory).

\bigskip
\noindent
\textbf{Definition:} An \emph{action} of the group $G$ on the set $X$ is a map $G \times X \rightarrow X$ by $(g,x) \mapsto \phi(g)x$ such that
\begin{itemize}
\item $\phi(e) x = x$
\item $\phi(g_1g_2) = \phi(g_1) \big [ \phi(g_2)x \big ]$
\end{itemize}

\bigskip
\noindent
Examples of actions include 
\begin{itemize}
\item $G = (S_n$, $\circ$) and $X = \{1, \hdots, n\}$. Then for permutation $\sigma$, $\phi(\sigma)i = \sigma(i)$. \\
\item $G = (\mathbb{Z},+)$ and $X = \mathbb{R}$. Then the translations $\phi(n)x = x + n$ are an action.             \\
\item $(S^1, +)$ and $X = \mathbb{C}$. Then the rotations $\phi(e^{i\theta})z = e^{i\theta}z$ are an action.
\end{itemize}


\bigskip
\noindent
Another definition of group action is that we say that the group  $G$ acts on the set $X$ if there is a homomorphism 
$\phi: G \rightarrow \text{Sym}(X)$.


\subsection{Group Actions and Equivalence Relations}
Perhaps surprisingly, it turns out that $x \sim y \Leftrightarrow \exists g \in G \text{ s.t. } \phi(g)x = y$, where $\phi$ is as 
defined above. So all the machinery developed for the equivalence relations $E_x$ apply to actions, except that in the 
context of group actions what we called equivalence classes above are called orbits. Orbits are denoted $O_x$ and
$O_x = E_x$.  In particular

\begin{flalign*}
O_x 
&= \{y \in X \mid x \sim y\} \\
&= \{ y \in X \mid \phi(g)x = y \} \\
&= \{\phi(g)x  \mid g \in G\}
\end{flalign*}

\section{Group Homomorphism}
Let $(G,\diamond)$ and $(H, \circ)$ be groups. Common usage is to use $G$ to refer to $(G,\diamond)$. Similarly, $H$ will refer to $(H, \circ)$. Then a 
mapping $\phi: G \rightarrow H$ is called a \emph{homomorphism} iff

\begin{equation*}
\phi(x \diamond y) = \phi(x) \circ \phi(y) \; \forall x,y \in G
\end{equation*}

\bigskip
\noindent
Essentially, a homomorphism $\phi: G \rightarrow H$ is a way of exploring the structure of $H$ by varying $G$ using structure preserving
transformations. That is, $\phi$ preserves the group operation.


\bigskip
\noindent
\textbf{Example:} Define a map

\begin{equation*}
\phi : G \rightarrow H
\end{equation*}

\bigskip
\noindent
where $G = \mathbb{Z}$  and $H = \mathbb{Z}_2 = \mathbb{Z}/2\mathbb{Z}$ is the standard group of order two. Then
define $\phi: \mathbb{Z} \rightarrow \mathbb{Z}_2$  by the rule

\begin{equation*}
  \phi(x) =
    \begin{cases}
      0 & \text{if $x$ is even} \\
      1 & \text{if $x$ is odd}\\
    \end{cases}       
\end{equation*}

\bigskip
\noindent
It is easy to check that $\phi$ is a homomorphism. Suppose that $x$ and $y$ are two integers. Then there are four cases:
\begin{itemize}
 \item $x$ and $y$ are both even
 
 In this case $\phi(x+y) = 0$ (even + even = even). Here $\phi(x)+\phi(y) = 0+0 = 0$ so $\phi(x+y) = \phi(x) + \phi(y)  = 0 + 0 = 0$. 
 
 \item $x$ and $y$ are both odd
  
In this case $\phi(x+y) = 0$ (odd + odd = even).  Here $\phi(x) + \phi(y) = 1 + 1 = 2 \mod 2 = 0$, so  $\phi(x+y) = \phi(x) + \phi(y)  = 1 + 1 = 2 \mod 2 = 0$.

 \item $x$ is even and $y$ is odd or  $x$ is odd and  $y$ is even
 
In this case one is even and the other is odd and $x + y$ is odd.  Here $\phi(x+y) = 1$ and $\phi(x)+\phi(y) = 1+0 = 1$ so $\phi(x+y) = \phi(x)+\phi(y)$.
 \end{itemize}
 
 \bigskip
 \noindent
 Thus $\phi$ is a homomorphism. Note that in this example  $\diamond = +$ (normal addition in $\mathbb{Z}$) and $\circ = +$ (addition mod 2 in $\mathbb{Z}_2$).



\bigskip
\noindent
\textbf{Example:} Let $G = (\mathbb{R}, +)$ and $H = (\mathbb{R}_{> 0}, \cdot)$ and let $\phi(x) = e^x$. To see that $\phi$ is a homomorphism
consider $\phi(x + y)$:

\begin{equation*}
\begin{array}{rcll}
\phi(x + y) 
&=& e^{(x + y)}                               &\quad \mathrel{\#} \text{definition of $\phi$}   \\
&=& e^x \cdot e^y                          &\quad \mathrel{\#} a^{(b + c)} = a^b \cdot a^c \\
&=& \phi(x) \cdot \phi(y)                     &\quad \mathrel{\#} \text{definition of $\phi$}                    \\
\phi(x + y)  &=&   \phi(x) \cdot \phi(y)   &\quad \mathrel{\#} \text{therefore $\phi$ is a homomorphism}
\end{array}
\end{equation*}

\bigskip
\noindent
Note that in this example $\diamond = +$ (normal addition) and $\circ = \cdot$ (normal multiplication). 

\bigskip
\noindent
Frequently the group operation is the same
for $G$ and $H$ (like, for example, if $\phi: G \rightarrow G$). So in many cases we see the homomorphism property expressed as follows:

\begin{equation*}
\phi(xy) = \phi(x)\phi(y) \; \forall x,y \in G
\end{equation*}

\bigskip
\noindent
\textbf{Example:}  Let $G$  be a group with $N \lhd G$ ($N$ is a normal subgroup of $G$). Define the \emph{quotient homomorphism} $q$ as 

\begin{equation*}
q: G \rightarrow  G/N 
\end{equation*}

\bigskip
\noindent
and let $x \mapsto  xN$ by $q(x) = xN$. To show that $q$ is a homomorphism, let $x,y \in G$ and consider $q(xy)$:


\begin{equation*}
\begin{array}{rcll}
q(xy) 
&=& xyN                     &\quad \mathrel{\#} \text{definition of $q$}   \\
&=& xyNN                  &\quad \mathrel{\#} NN =N \text{ for $N$ a subgroup in $G$} \\
&=& x (yN) N              &\quad \mathrel{\#} \text{group multiplication is associative} \\
&=& x (Ny) N              &\quad \mathrel{\#} \text{$N$ normal so  $xN = Nx$} \\
&=& (xN)(yN)              &\quad \mathrel{\#} \text{group multiplication is associative} \\
&=& q(x) q(y)              &\quad \mathrel{\#} \text{definition of $q$}          \\
q(xy) &=& q(x) q(y)     &\quad \mathrel{\#} \text{therefore $q$ is a homomorphism} 
\end{array}
\end{equation*}

  % \newpage

\bigskip
\begin{lemma}
Let $G$ and $H$ be groups where $e$ is the identity in $G$ and $f$ is the identity in $H$ and define a homomorphism $\phi: G \rightarrow H$. Then
\begin{enumerate}
\item $\phi(e) = f$. That is,  $\phi$ maps the identity in G to the identity in H. In particular
   \begin{itemize}
     \item $\phi(0) = 0$
     \item $\phi(1) = 1$
    \end{itemize}
\item $\phi(a^{-1}) = (\phi(a))^{-1}$. That is, $\phi$ maps inverses to inverses.
\item If $K$ is subgroup of $G$, then $\phi(K)$ is a subgroup of $H$.
\end{enumerate}
\end{lemma}

\bigskip
\noindent
\textbf{Proof:} To show 1. above let $a = \phi(e)$. Then

\begin{equation*}
\begin{array}{rcll}
a 
&=& \phi(e)                                                                  &\quad \mathrel{\#}  \text{by assumption}   \\
&=& \phi(e \cdot e)                                                       &\quad \mathrel{\#} x \cdot e = e \cdot x = x \text{ so } e = e \cdot e   \\
&=& \phi(e) \cdot \phi(e)                                              &\quad \mathrel{\#} \text{$\phi$ is a homomorphism}\\
&=& a \cdot \phi(e)                                                       &\quad \mathrel{\#} a = \phi(e)  \text{ (by assumption)}   \\
&=& a \cdot a                                                               &\quad \mathrel{\#} \hdots \\
&\implies& a \cdot a^{-1} =  (a \cdot a)  \cdot a^{-1}    &\quad \mathrel{\#} \text{multiply on the right by  $a^{-1}$} \\
&\implies& a \cdot a^{-1} =  a \cdot (a \cdot a^{-1})     &\quad \mathrel{\#} \text{multiplication is associative} \\
&\implies& a \cdot a^{-1} =  a \cdot  f                           &\quad \mathrel{\#} a \cdot a^{-1} = f, \text{ where $f$ is the identity in $H$} \\
&\implies& a \cdot a^{-1} =  a                                       &\quad \mathrel{\#} a \cdot  f  = a \\
&\implies& f = a                                                            &\quad \mathrel{\#} a \cdot a^{-1} = f  \text{ by definition of inverse} \\
&\implies& \phi(e) =  f                                                   &\quad \mathrel{\#} \text{$\phi$ maps the identity in $G$ to the identity in $H$}
\end{array}
\end{equation*}

\bigskip
\noindent
So $\phi$ maps the identity in $G$ to the identity in $H$ which shows 1. above.

\bigskip
\noindent
Another approach is to recognize that if the group operation is $+$, then $0$ is the (additive) identity, that is, $0 + x = x + 0 = x$. Then

\begin{equation*}
\begin{array}{rcll}
0 + x 
&=& x                                                                              &\quad \mathrel{\#} \text{$0$ is the additive identity}   \\
&\implies& \phi(0  + x) = \phi(x)                                      &\quad \mathrel{\#} \text{apply $\phi$}          \\
&\implies& \phi(0) +  \phi(x) = \phi(x)                              &\quad \mathrel{\#} \phi \text{ is a homomorphism}          \\
&\implies& (\phi(0) + \phi(x)) - \phi(x) = \phi(x) -\phi(x)   &\quad \mathrel{\#} \text{subtract $\phi(x)$ from both sides} \\
&\implies& (\phi(0) + \phi(x)) - \phi(x) = 0                       &\quad \mathrel{\#} \phi(x) -\phi(x) = 0 \\
&\implies& \phi(0) + (\phi(x) - \phi(x)) = 0                       &\quad \mathrel{\#} \text{addition is associative} \\
&\implies& \phi(0) + 0  = 0                                              &\quad \mathrel{\#} \phi(x) -\phi(x) = 0 \\
&\implies& \phi(0) = 0                                                     &\quad \mathrel{\#} \phi(x) \text{ maps $0$ to $0$}
\end{array}
\end{equation*}

\bigskip
\noindent
So $\phi(0) = 0$. A similar argument can be used where the group operation is multiplication:

\begin{equation}
\begin{array}{rcll}
1 \cdot x 
&=& x                                                                                                                         &\quad \mathrel{\#} \text{$1$ is the multiplicative identity} \\
&\implies& \phi(1 \cdot x) = \phi(x)                                                                             &\quad \mathrel{\#} \text{apply $\phi$} \\
&\implies& \phi(1) \cdot \phi(x) = \phi(x)                                                                     &\quad \mathrel{\#} \phi \text{ is a homomorphism}          \\
&\implies& (\phi(1) \cdot \phi(x)) \cdot  (\phi(x))^{-1} = \phi(x) \cdot  (\phi(x))^{-1}      &\quad \mathrel{\#} \text{multiply on the right by $ (\phi(x))^{-1}$} \\
&\implies& (\phi(1) \cdot \phi(x)) \cdot  (\phi(x))^{-1} = 1                                             &\quad \mathrel{\#}\phi(x) \cdot  (\phi(x))^{-1}  = 1 \\
&\implies& \phi(1) \cdot (\phi(x) \cdot  (\phi(x))^{-1}) = 1                                             &\quad \mathrel{\#} \text{multiplication is associative} \\
&\implies& \phi(1) \cdot 1 = 1                                                                                      &\quad \mathrel{\#} \phi(x) \cdot  (\phi(x))^{-1} = 1 \\
&\implies&  \phi(1) = 1                                                                                                 &\quad \mathrel{\#}  x \cdot 1 = x
\end{array}
\label{eqn:1}
\end{equation}

\bigskip
\noindent
So we have $\phi(0) = 0$ and  $\phi(1) = 1$. 

\bigskip
\noindent
To show 2., note that since $G$ is a group, $x^{-1} \in G$ and  $x \cdot x^{-1} = 1$ for $x, x^{-1}  \in G$. Then

\begin{equation}
\begin{array}{rcll}
 x \cdot x^{-1}
&=& 1                                                                              &\quad \mathrel{\#} \text{definition of inverse}   \\
&\implies& \phi(x \cdot x^{-1}) = \phi(1)                           &\quad \mathrel{\#} \text{apply $\phi$}          \\
&\implies& \phi(x \cdot x^{-1}) = 1                                   &\quad \mathrel{\#} \phi(1) =1 \text{ (see Equation \ref{eqn:1} above)}  \\
&\implies& \phi(x) \cdot \phi(x^{-1}) = 1                           &\quad \mathrel{\#} \phi \text{ is a homomorphism}    \\
&\implies&  \phi(x^{-1}) = 1/\phi(x)                                   &\quad \mathrel{\#}  \text{divide both sides by } \phi(x) \\
&\implies&  \phi(x^{-1}) = (\phi(x))^{-1}                             &\quad \mathrel{\#} \text{hence $\phi$ maps inverses to inverses}
\end{array}
\label{eqn:inverse}
\end{equation}

\bigskip
\noindent
Finally, to show 3., let $X = \phi(K)$. Then it suffices to show that $X$ is non-empty and closed under products and inverses. $X$ contains $f$, the identity of $H$, by 1.
above. We also know that  $X$ is closed under inverses by 2. above. Finally, we know that $X$  and is closed under products (almost) by definition. Thus $X$ is a subgroup. 
$\square$

% \newpage
\begin{thm}
Let $G$ and $H$ be groups and let $f: G \rightarrow H$ be a group homomorphism. Then $f$ is one-to-one iff the kernel of $f$
is trivial, that is, $\ker f = \{e\}$ where $e$ is the identity element of $G$.
\end{thm}

\noindent
\textbf{Proof: }  Here we'll show that $f$ is one-to-one  $\iff \ker f =\{e\}$: 
\begin{itemize}
\item $f$ is one-to-one  $\implies \ker f =\{e\}$ 

Suppose the homomorphism $f : G \rightarrow H$ is one-to-one. Then since $f$ is a group homomorphism, the identity element $e$ of $G$ is mapped 
to the identity element $e^\prime$ of $H$. That is,  $f(e ) = e^\prime$ (see Equation \ref{eqn:1} above).

Now let $g \in \ker f$, so $f(g) = e^\prime$, recalling that $\ker f = \{g \in G \mid f(g) = e^\prime\}$. So now $f(e) = e^\prime$ and  $f(g) = e^\prime$ which 
implies that $f(g)=f(e)$. 
 
Since $f$ is one-to-one\footnote{Recall that saying $f$ is one-to-one means that $f(x) = f(y) \implies x = y$.} we know that  $(g) = f(e) \implies g = e$, so $g = e$. But $g$ was an 
arbitrary element of $\ker f$  so $f$ maps every $g \in \ker f$ to $e$. Hence $\ker f = \{e\}$. 

\item $\ker f = \{e\} \implies f$ is one-to-one

On the other hand, suppose that $\ker f = \{e\}$ and that there exists $g_1, g_2 \in G$ such that 

\begin{equation}
f(g_1) = f(g_2) 
\label{eqn:g}
\end{equation}

Now consider an element $g_1g_2^{-1} \in G$. Then

\begin{equation}
\begin{array}{rcll}
f(g_1g_2^{-1})
&=& f(g_1) f(g_2^{-1})                                                                                           &\quad \mathrel{\#} \text{$f$ is a homomorphism}   \\
&=& f(g_1) (f(g_2))^{-1}                                                                                           &\quad \mathrel{\#} f(g_2^{-1})   = (f(g_2))^{-1}  \text{ (Equation \ref{eqn:inverse})}  \\
&=& f(g_1) (f(g_1))^{-1}                                                                                           &\quad \mathrel{\#} f(g_2) = f(g_1)  \text{ (Equation \ref{eqn:g})} \\
&=& e^\prime                     &\quad \mathrel{\#} f(g_1) (f(g_1))^{-1}  = e^\prime
\end{array}
\label{eqn:eprime}
\end{equation}

So $f(g_1g_2^{-1}) = e^\prime$ which implies that  $g_1g_2^{-1} \in \ker f$. But by assumption $\ker f = \{e\}$ so $g_1g_2^{-1}  = e$. 

If you multiply $g_1g_2^{-1}  = e$  on the right by $g_2$, you notice that 

\begin{equation*}
\begin{array}{rcll}
g_1g_2^{-1}
&=& e                                                                                     &\quad \mathrel{\#} \text{Equation \ref{eqn:eprime}}   \\
&\implies&  (g_1 \cdot g_2^{-1}) \cdot g_2 = e \cdot g_2         &\quad \mathrel{\#} \text{multiply on the right by $g_2$} \\
&\implies&  g_1 \cdot  (g_2^{-1} \cdot g_2) = e \cdot g_2      &\quad \mathrel{\#} \text{multiplication is associative} \\
&\implies&  g_1 \cdot e = e \cdot g_2                                    &\quad \mathrel{\#} g_2^{-1} \cdot g_2 = e \\
&\implies&  g_1 =  g_2                                                           &\quad \mathrel{\#} e \cdot g = g \cdot e = g
\end{array}
\end{equation*}


So $f(g_1) = f(g_2)$ (Equation \ref{eqn:g}) implies that $g_1 = g_2$ and so $f$ is one-to-one. 
\end{itemize}

This shows that $f$ is one-to-one  $\iff \ker f =\{e\}$.

\section{The First Isomorphism Theorem}
The First Isomorphism Theorem (FIT) is a handy piece of machinery for many problems in group theory. Before getting to all of that, recall the following:

\begin{itemize}
\item A mapping $\phi: G \rightarrow G^\prime$ is called a \textbf{group homomorphism} if it preserves the group operation: $\phi(ab) = \phi(a)\phi(b)$.
\item The \textbf{image of $\mathbf{G}$:}  $\phi(G) = \{\phi(g) \mid g \in G\}$.
\item The \textbf{kernel of $\boldsymbol{\phi}$}: $\ker \phi = \{g \in G \mid \phi(g) = e^\prime\}$  ($e^\prime$ is the identity in $G^\prime$).
\item $\phi(a) = \phi(b)$ iff $a \ker \phi = b \ker \phi$.
\item If $\phi(g) = g^\prime$ then $\phi^{-1}(g^{\prime}) = g \ker \phi$.
\end{itemize}

\bigskip
\noindent
So here's a theorem:  Let $\phi: G \rightarrow G$ be a  group homomorphism. Then $\ker \phi  \lhd  G$.


\bigskip
\noindent 
In other words:  The kernel of $\phi$ is a normal subgroup of $G$. 

\bigskip
\noindent
To show this, first let $k \in \ker \phi$ and let $g \in G$. Then we 
want to show\footnote{Recall that for normal subgroups $xN = Nx$ which implies that $xnx^{-1} \in N$.}
that $gkg^{-1} \in  \ker \phi$ or equivalently that $g \ker \phi g^{-1} = \ker \phi$. So now consider $\phi(gkg^{-1})$:

\begin{equation*}
\begin{array}{rcll}
\phi(gkg^{-1})
&=& \phi(g)\phi(k)\phi(g^{-1})         &\qquad \mathrel{\#} \text{$\phi$ is a homomorphism}   \\
&=& \phi(g)\phi(k)(\phi(g))^{-1}       &\qquad \mathrel{\#} \phi(g^{-1})  = (\phi(g))^{-1} \text{ (see (\ref{eqn:inverse}) above)} \\
&=& \phi(g) e (\phi(g))^{-1}             &\qquad \mathrel{\#} k \in \ker \phi \implies \phi(k) = e  \text{ ($e$ is the identity in $G$)} \\
&=& \phi(g) (\phi(g))^{-1}                &\qquad \mathrel{\#} \phi(g) e = \phi(g) \text{ and } e   \phi(g)^{-1}  = \phi(g)^{-1} \\
&=& e                                             &\qquad \mathrel{\#} xx^{-1} = x^{-1} x = e \text{ (where $x \neq 0$)}  \\
\phi(gkg^{-1}) &=& e                      &\qquad \mathrel{\#} \implies gkg^{-1} \in  \ker \phi
\end{array}
\end{equation*}

\bigskip
\noindent
So $gkg^{-1} \in  \ker \phi \implies g \ker \phi g^{-1} = \ker \phi \implies \ker \phi \lhd G$. 

\bigskip
\noindent
\textbf{Example:} Consider the group $(\mathbb{Z}_{12},+)$ and let $\phi: \mathbb{Z}_{12} \rightarrow \mathbb{Z}_{12}$ by  $\phi(x) = 3x$. 
First, we need to show that $\phi$ is a homomorphism. But this is an easy one:  $\phi(x+y) = 3 (x + y) = 3x + 3y = \phi(x) + \phi(y)$. 
So $\phi$ is a homomorphism. Next consider the mapping $\phi$ on $\mathbb{Z}_{12}$:

\bigskip
\begin{equation*}
\begin{array}{l c c c c c c c c c c c c}
\mathbb{Z}_{12}:  & 0      & 1      & 2      & 3      & 4      & 5      & 6      & 7      & 8      & 9      & 10    & 11 \\
                             & \bda & \bda & \bda & \bda & \bda & \bda & \bda & \bda & \bda & \bda & \bda & \bda \\
\phi(x):                  & 0      & 3      & 6      & 9      & 0      & 3      & 6      & 9      & 0      & 3      & 6      & 9 \\
\end{array}
\end{equation*}

\bigskip
\noindent
We can see that the image of $\mathbb{Z}_{12}$, $\phi(\mathbb{Z}_{12}) = \{0,3,6,9\}$ and that $\ker \phi = \{0,4,8\}$. 

\bigskip
\noindent
Grouping the  elements of $\mathbb{Z}_{12}$ with their images we see that

\bigskip
\begin{equation*}
\begin{array}{l c c c c}
\mathbb{Z}_{12}:  & \{0,4,8\}  & \{1,5,9\}     & \{2,6,10\}   & \{3,7,11\} \\
                             & \bda       & \bda          & \bda           & \bda  \\
\phi(x):                  & 0            & 3               & 6                & 9     \\
                             & \bda       & \bda          & \bda           & \bda  \\
\phi(x)^{-1}:          & \{0,4,8\}  & \{1,5,9\}     & \{2,6,10\}    & \{3,7,11\}  \\
                             & \veq       & \veq           & \veq           & \veq \\
                             & 0 + \ker \phi & 1+\ker \phi & 2+\ker \phi & 3+\ker \phi

\end{array}
\end{equation*}

\bigskip
\noindent
This is a nice example of the property  that $\phi(a) = x \implies \phi^{-1}(x) = a \ker \phi$. To see this, choose, for example, $a = 5$. Then 
$\phi(5) = 3\cdot 5 = 15$ and $15 \mod 12 = 3$, so $x = 3$. This implies that  
$\phi(3)^{-1} = 5 + \ker \phi = \{5+0, 5+4, 5+8\} = \{5, 9, 13\} = \{1,5,9\}$, noting that $13 \mod 12 = 1$. 

\bigskip
\noindent
Now we can state the \textbf{First Isomorphism Theorem:} Let $\phi: G \rightarrow H$ be an onto homomorphism.  Then $H \simeq G/N$, where $N = \ker \phi$.

\bigskip
\noindent
Though technical this is really an amazing theorem. So here's one way to prove this.  We need to construct  an isomorphism 
$\widetilde{\phi}: G/N \rightarrow H$ by $\widetilde{\phi}(gN) = \phi(g)$. To show this, we need to show three things:  that $\widetilde{\phi}$ is well-defined, 
that $\widetilde{\phi}$ is a homomorphism, and that $\widetilde{\phi}$ is one-to-one and onto.

\begin{itemize}
\item To show that $\widetilde{\phi}$ is well-defined\footnote{Well-defined is kind of the opposite of one-to-one where we show that if $f(x) = f(y)$ then $x = y$.}. 
That is, if $gN = hN$ then $\widetilde{\phi}(gN) = \widetilde{\phi}(hN)$. Now, if $gN = hN$ then since $N$ is the the identity in $G/N$, $g = hn$ for some $n \in N$. This means

\begin{equation*}
\begin{array}{rcll}
\widetilde{\phi}(gN)
&=& \phi(g)                                                    &\qquad \mathrel{\#} \text{definition of $\widetilde{\phi}(gN)$}   \\
&=& \phi(hn)                                                  &\qquad \mathrel{\#}  g = hn \text{ since } gN = hN \text{ by assumption}  \\
&=& \phi(h)\phi(n)                                            &\qquad \mathrel{\#}  \text{$\phi$ is a homomorphism} \\
&=& \phi(h)e                                                  &\qquad \mathrel{\#}  n \in N = \ker \phi = \{x \in G  \mid \phi(x) = e\} \\
&=&  \phi(h)                                                   &\qquad \mathrel{\#}  x \cdot e = x  \\
&=& \widetilde{\phi}(hN)                                &\qquad \mathrel{\#}  \text{definition of $\widetilde{\phi}$} \\
\widetilde{\phi}(gN) &=& \widetilde{\phi}(hN)   &\qquad \mathrel{\#} \text{so $\widetilde{\phi}$ is well-defined}
\end{array}
\end{equation*}

\item Next, we need to show that $\widetilde{\phi}$ is a homomorphism. To see this, consider 

\begin{equation*}
\begin{array}{rcll}
\widetilde{\phi}(gN hN)
&=& \widetilde{\phi}(ghNN)                                                              &\qquad \mathrel{\#} \text{definition of group multiplication}   \\
&=& \widetilde{\phi}(ghN)                                                                 &\qquad \mathrel{\#} NN = N \text{($N$ a subgroup)}  \\
&=& \phi(gh)                                                                                     &\qquad \mathrel{\#} \text{definition of $\widetilde{\phi}$}  \\
&=& \phi(g)\phi(h)                                                                               &\qquad \mathrel{\#} \phi \text{ is a homomorphism} \\
&=& \widetilde{\phi}(gN) \widetilde{\phi}(hN)                                      &\qquad \mathrel{\#} \text{definition of $\widetilde{\phi}$} \\
\widetilde{\phi}(gN hN) &=& \widetilde{\phi}(gN) \widetilde{\phi}(hN)   &\qquad \mathrel{\#} \text{therefore $\widetilde{\phi}$ is a homomorphism}
\end{array}
\end{equation*}

So $\widetilde{\phi}$ is a homomorphism.

\item  For one-to-one,  we need to show that $\widetilde{\phi}(gN) =  \widetilde{\phi}(hN)  \implies gN = hN$. We know that 
 if $\widetilde{\phi}(gN) = e_H$, then $\phi(g) = e_H$. So $g \in \ker \phi$ (recalling that $N = \ker \phi$). This implies that $gN \in N$ ($N$ is the 
identity in $G/N$). Similarly for $\widetilde{\phi}(hN)$, so $\widetilde{\phi}$ is one-to-one. [ed: this  isn't complete]

\item Finally,  to show onto,  choose a $h$ in $H$. We want to show that $\exists gN \in G/N \text{ with } \widetilde{\phi}(gN)  = h$. Well, we know that $\phi$ is onto
so $\exists g \in G$ with $\phi(g) = h$. This means that $\widetilde{\phi}(gN) = \phi(g) = h$, so $\widetilde{\phi}$ is onto.
\end{itemize}


\bigskip
\noindent
This shows that $H \simeq G/\ker \phi$.

\section{Rings, Ideals and Homomorphisms}
This section provides a few notes on Ring Theory. To start, a few definitions

\noindent
\begin{definition}
A ring $R$ is an abelian group with a multiplication operation  
\begin{equation*}
(a, b) \mapsto ab
\end{equation*}
which is associative and satisfies the distributive laws $a(b+c) = ab+ac$ and $(a+b)c = ac+bc$ and has identity element 1.
\end{definition}

\noindent
Here there is a group structure with the addition operation but not necessarily with the multiplication operation. Thus an element of a ring may or may not 
be invertible with respect to the multiplication operation. 

\begin{definition}
Let $a,b$ be in a ring $R$. If $a \neq 0$ and $b \neq 0$ but $ab = 0$, then we say that $a$ and $b$ are 
\emph{zero divisors}. If $ab = ba = 1$, we say that $a$ is a \emph{unit} or that $a$ is invertible.
\end{definition}

\noindent
For example, consider the ring $\mathbb{Z}_n$. Notice that if $n$ is not prime then $\mathbb{Z}_n$ has zero divisors. For a concrete example, consider 
the ring  $\mathbb{Z}_{10}$ and let $a = 2$ and $b = 5$. Then  $ab = 2 \cdot 5 = 10$ and $10 \mod 10 = 0$, so 2 and 5 are zero divisors in 
$\mathbb{Z}_{10}$. On the other hand, if $n = p$ is prime then the only factors of $p$ are 1 and $p$, so $\mathbb{Z}_p$ has no zero divisors.

\begin{definition}
Let $R$ be a ring. Then if $ab = ba$ for any $a,b \in R$ then $R$ is said to be \emph{commutative}.
\end{definition}

\noindent
In general, for a given ring $R$ the addition operation is commutative,  but the the multiplication operation may or may not be commutative. 
There are two particular kinds of rings where the multiplication operation is well-behaved:

\begin{definition}
 An \emph{integral domain} is a commutative ring with no zero divisors. A \emph{division ring} or skew field is a ring in which every non-zero element $a$
 has an inverse $a^{-1}$.
\end{definition}

\noindent
For example, the integers $\mathbb{Z}$ form an integral domain. The quaternions (more on this later) form a division ring.

\begin{definition}
The characteristic of a ring $R$, denoted by $\text{char}(R)$, is the smallest positive integer $n$ such that

\begin{equation*}
n \cdot 1 = \underbrace{1 + 1 + \cdots + 1}_{n \text{ times}} = 0
\end{equation*}
\end{definition}


\bigskip
\noindent
One more definition then we'll look at a few examples.

\begin{definition}
A subring of a ring $R$ is a subset $S$ of $R$ that forms a ring under the operations of addition and multiplication defined in $R$.
\end{definition}

\subsection{Examples}
Before diving into examples, it is worthwhile to notice that the concepts of subrings and ideals, while related, have subtle and important differences. 
Both an ideal $I$ and a subring $S$ of a ring $R$ are subsets of $R$ which are subgroups under addition and are closed under multiplication. However, 
each has an  additional property: 

\begin{itemize}
\item An ideal $I$ has the absorption property ($\forall r \in R$ $rI \subset I$), while a subring is only required to be closed 
under multiplication.
\item A subring $S$ is usually required to contain the multiplicative identity 1, while an ideal is not required to contain 1.
\end{itemize}

\noindent
For example, consider the integers $\mathbb{Z}$.  $\mathbb{Z}$ is a subring of the rational numbers $\mathbb{Q}$
but is not an ideal since $\frac{1}{2} \cdot 1 = \frac{1}{2}$ and $\frac{1}{2} \notin  \mathbb{Z}$ (that is, $\mathbb{Z}$ does not
\emph{absorb} $\mathbb{Q}$). On the other hand,  the subset $2\mathbb{Z} \subset \mathbb{Z}$, where $2\mathbb{Z} = \{2n \mid n \in Z\}$
is an ideal of $\mathbb{Z}$  but is not a subring (since $1 \notin 2\mathbb{Z}$). A few other examples:

\begin{itemize}
\item $\mathbb{Z}$ is an integral domain but not a field since only $\{-1,1\}  \in \mathbb{Z}$ have inverses in $\mathbb{Z}$.
\item As we saw above, the integers modulo $n$, $\mathbb{Z}_n$, form a ring which is an integral domain if and only if $n$ is prime.
\item The $n \times n$ matrices $\mathcal{M}_{n}(\mathbb{R})$ with coefficients in $\mathbb{R}$ are a ring, but not an
integral domain if $n \geq 2$.
\item The quaternions are the smallest and perhaps the most famous example of a division ring. To see this, first take $1, i, j, k$ to be basis vectors 
for a 4-dimensional vector space over $R$, and define multiplication by
\begin{flalign*}
i^2  &= j^2 = k^2 = -1, \\
ij     &= k,                     \\
jk    &= i,                      \\
ki    &= j,                      \\ 
ji     &= -ij,                    \\
kj    &= -jk,                   \\
ik    &= -ki
\end{flalign*}

\noindent
Then we can define the quaternions $\mathbb{H} = \{a + bi + cj + dk, \text{ for } a,b,c,d \in R \}$. $\mathbb{H}$ forms a division ring called the quaternions \cite{horn2001}. 
To show that $\mathbb{H}$ is a division ring, we need to show that every non-zero element is invertible. To do this, consider the \emph{conjugate} of an element 
$h = a + bi + cj +dk \in H$ to be $\bar{h} = a - bi - cj - dk$; note that this is analogous to the complex conjugates we saw for complex numbers in $\mathbb{C}$.
It is pretty easy to see (multiply it out) that 

\begin{equation*}
q\bar{q} = a^2 + b^2 + c^2 + d^2 
\end{equation*}

\noindent
Next,  take 
\begin{equation}
q^{-1} = \frac{q}{q \bar{q}}
\label{eqn:quat}
\end{equation}

\bigskip
\noindent
Then  $qq^{-1} = q^{-1}q = 1$ and the denominator in Equation \ref{eqn:quat} cannot be $0$ unless $a = b = c = d = 0$.
\end{itemize}

\bigskip
\noindent
As we saw with groups, we can define a map from a ring to another which has the property of carrying one ring structure to the other.

\bigskip
\begin{definition}
Let $R$ and $S$ be rings. A map $f : R \rightarrow S$  satisfying
\begin{enumerate}
\item $f(a + b) = f(a) + f(b)$  (so $f$ is a group homomorphism under +)
\item $f(ab) = f(a)f(b)$
\item $f(1_R) = 1_S$
\end{enumerate}
for $a, b \in R$ is called \emph{ring homomorphism}
\end{definition}

\subsection{Ideals}
The concept of an "ideal number" was introduced by the mathematician Kummer as being some special numbers (well, today we call them groups) having 
the property of unique factorization, even when considered over more general rings than $\mathbb{Z}$. Today's definition of \emph{ideals} looks more like:

\begin{definition}
Let $I$ be a subset of a ring $R$. Then an additive subgroup of $R$ having the property that
\begin{equation*}
ra \in I \text{ for }  a \in I, \: r \in R
\end{equation*}

\noindent
is called a left ideal of $R$.  Similarly

\begin{equation*}
ar \in I  \text{ for }  a \in I, \: r \in R
\end{equation*}

\noindent
is called a right ideal of $R$. If an ideal is both a right and a left ideal then we call it a two-sided ideal of $R$, or simply an ideal of $R$.
\end{definition}

\noindent
 We say that an ideal $I$ of $R$ is proper if $I \neq R$.  We say that is it non-trivial if $I \neq R$ and  $I \neq  0$.

\bigskip
\noindent
I've seen several different notations for ideals including, among others:  $rI = \{ri \mid r \in R, i \in I\}$ and 
$rI \subset I \;  \forall r \in R$. 

\bigskip
\noindent
So in words: An ideal $I$ is a subset of a ring $R$ such that

\begin{itemize}
\item $I$ is a subgroup of $R$ under addition (so $0 \in I$ and so $I \neq \emptyset$) 
\item $I$  is not only closed under multiplication but also satisfies the \emph{stronger property} that it  
"absorbs" all of the elements of $R$ under multiplication: $\forall r \in R$ $rI \subset I$
\end{itemize}

\bigskip
\noindent
In addition, if $f : R \rightarrow S$ is a ring homomorphism we define the kernel of $f$ in the expected way, namely,  $\ker f = \{r \in R \mid f(r) = 0\}$. Since 
a ring homomorphism is a group homomorphism, we already know that $f$ is one-to-one iff $\ker f = \{0\}$.  $\ker f$ is a proper two-sided ideal since

\begin{itemize}
\item $\ker f$ is an additive subgroup of $R$
\item Take $a \in \ker f$ and $r \in R$. Then $f(ra) = f(r)f(a) = 0$ and $f(ar) = f(a)f(r) = 0$ so both $ra$ and $ar$ are in $\ker f$.
\end{itemize}

\noindent
Quick proof:  Since $\ker f$ is a two-sided ideal of $R$, then either $\ker f = \{0\}$ or $\ker f = R$.
But $ker f \neq R$ since $f(1) = 1 $ by definition. In words, $\ker f$ is a proper ideal.

\bigskip
\noindent
It is worth noticing the analogy between rings and their two-sided ideals and groups and their normal subgroups:

\begin{itemize}
\item Two-sided ideals are stable when the ring acts on them by multiplication, either on the right or on the left, and so
\begin{equation*}
rar^{-1} \in I \text{ for }a \in I \text{ and } r \in R
\end{equation*}

\noindent
while normal subgroups are stable when the groups act on them on them by conjugation:
\begin{equation*}
ghg^{-1} \in H \text{ for } h \in H, g \in G
\end{equation*}

\noindent
That is  $H \lhd G$.
\item Groups with only trivial normal subgroups are called simple. We will not see it formally here, but rings with only trivial two-sided ideals 
as in the above lemma are called simple rings.
\item The kernel of a group homomorphism is a normal subgroup, while the kernel of a ring homomorphism is an ideal.
\item Normal subgroups allowed us to define quotient groups. We will see now that two-sided ideals will allow to define quotient rings.
\end{itemize}

\subsection{Quotient rings}
Let I be a proper two-sided ideal of $R$. Since $I$ is an additive subgroup of $R$ by definition, it makes sense to speak of cosets $r + I$ 
of $I$ where $r \in R$. Furthermore, a ring has a structure of abelian group for addition, so $I$ satisfies the definition of a normal subgroup. 
From group theory we know that of the quotient group is
\begin{equation*}
R/I = \{r + I \text{ for } r \in R\}
\end{equation*}


\bigskip
\section{Acknowledgements}
Thanks to Joel Bion for pointing out a typo in an early version of these notes.

\newpage
\bibliographystyle{plain}
\bibliography{/Users/dmm/papers/bib/qc}
\end{document} 
