\documentclass[11pt, oneside]{article}   	% use "amsart" instead of "article" for AMSLaTeX format


\usepackage{geometry}                		% See geometry.pdf to learn the layout options. There are lots.
\geometry{letterpaper}                   		% ... or a4paper or a5paper or ... 
%\geometry{landscape}                		% Activate for for rotated page geometryhttps://www.washingtonpost.com/world/europe/amid-impeachment-probe-gordon-sondland-is-overseeing-a-renovation-of-his-residence-that-has-cost-1-million-in-taxpayer-money/2019/10/16/d0eece92-ef86-11e9-bb7e-d2026ee0c199_story.html?tid=sm_tw
%\usepackage[parfill]{parskip}    		% Activate to begin paragraphs with an empty line rather than an indent
\usepackage{graphicx}				% Use pdf, png, jpg, or eps� with pdflatex; use eps in DVI mode
								% TeX will automatically convert eps --> pdf in pdflat						\label{thm:integral_domain}

								% TeX will automatically convert eps --> pdf in pdflatex		
\usepackage{amssymb}
\usepackage{amsmath}
\usepackage{amsthm}
\usepackage{mathrsfs}
\usepackage[hyphens,spaces,obeyspaces]{url}
\usepackage{url}
\usepackage{hyperref}
\usepackage{subcaption}
\usepackage{authblk}
\usepackage{mathtools}
\usepackage{graphicx}
\usepackage[export]{adjustbox}
\usepackage{fixltx2e}
\usepackage{hyperref}
\usepackage{alltt}
\usepackage{color}
\usepackage[utf8]{inputenc}
\usepackage[english]{babel}
\usepackage{float}
\usepackage{bigints}
\usepackage{braket}
\usepackage{siunitx}
\usepackage{mathtools}


\usepackage{tikz}
\usepackage{verbatim}
\usetikzlibrary{plotmarks}
\usepackage{pgfplots}
\usepackage{amsmath}
\usepackage{relsize}
 \usepackage[hyphenbreaks]{breakurl}
 
 \newtheorem{thm}{Theorem}[section]
% \newtheorem{defn}[thm]{Definition}
\theoremstyle{definition}
\newtheorem{definition}{Definition}[section]
\newtheorem{proposition}{Proposition}[section]
\newtheorem{lemma}{Lemma}[section]
\newtheorem{example}{Example}[section]

\newcommand{\argmax}{\operatornamewithlimits{argmax}}
\newcommand{\argmin}{\operatornamewithlimits{argmin}}





\title{A Few Notes On The Riemann Zeta Function}
\author{David Meyer \\ dmm@1-4-5.net}

\date{Last update: \today}							% Activate to display a given date or no date

\begin{document}
\maketitle

\section{Introduction}
TBD

\section{Euler's Product Formula}
Recall that the Riemann zeta function is defined to be

\medskip
\begin{equation*}
\zeta(s) = \sum^\infty_{n = 1} \frac{1}{n^s}
\end{equation*}

\bigskip
\noindent
In 1737, Leonhard Euler discovered the beautiful connection between the zeta function and the prime numbers and proved this identity:

\medskip
\begin{equation}
\sum^\infty_{n = 1} \frac{1}{n^s} = \prod_{\text{$p$ prime}} \!  \frac{1}{1 - p^{-s}}
\label{eqn:zeta}
\end{equation}

\bigskip
\noindent
The left side of Equation \ref{eqn:zeta} is by definition $\zeta(s)$. The infinite product on the right side of Equation \ref{eqn:zeta}  extends over all prime numbers $p$ and is called an Euler 
Product \cite{euler_product}, which is the expansion of a Dirichlet series into an infinite product indexed by prime numbers:

\begin{equation*}
\prod_{\text{$p$ prime}} \!  \frac{1}{1 - p^{-s}} = \frac{1}{1 - 2^{-s}} \cdot \frac{1}{1 - 3^{-s}} \cdot  \frac{1}{1 - 5^{-s}} \cdot  \frac{1}{1 - 7^{-s}} \cdot  \frac{1}{1 - 11^{-s}} \cdots  \frac{1}{1 - p^{-s}} \cdots
\end{equation*}

\bigskip
\noindent
Both sides of the Euler product formula converge for $\text{Re}(s) > 1$.


\bibliographystyle{plain}
\bibliography{/Users/dmm/papers/bib/qc}


\end{document} 

