\documentclass[11pt, oneside]{article}   	% use "amsart" instead of "article" for AMSLaTeX format


% \usepackage{draftwatermark}
% \SetWatermarkText{Draft}
% \SetWatermarkScale{5}
% \SetWatermarkLightness {0.9} 
% \SetWatermarkColor[rgb]{0.7,0,0}


\usepackage{geometry}                		% See geometry.pdf to learn the layout options. There are lots.
\geometry{letterpaper}                   		% ... or a4paper or a5paper or ... 
%\geometry{landscape}                		% Activate for for rotated page geometry
%\usepackage[parfill]{parskip}    		% Activate to begin paragraphs with an empty line rather than an indent
\usepackage{graphicx}				% Use pdf, png, jpg, or eps� with pdflatex; use eps in DVI mode
								% TeX will automatically convert eps --> pdf in pdflat						
								% TeX will automatically convert eps --> pdf in pdflatex		
\usepackage{amssymb}
\usepackage{mathrsfs}
\usepackage{hyperref}
\usepackage{url}
\usepackage{subcaption}
\usepackage{authblk}
\usepackage{amsmath}
\usepackage{mathtools}
\usepackage{graphicx}
\usepackage[export]{adjustbox}
\usepackage{fixltx2e}
\usepackage{hyperref}
\usepackage{alltt}
\usepackage{color}
\usepackage[utf8]{inputenc}
\usepackage[english]{babel}
\usepackage{float}
\usepackage{bigints}
\usepackage{braket}
\usepackage{siunitx}


\DeclareMathOperator{\E}{\mathbb{E}}
\newcommand{\argmax}{\operatornamewithlimits{argmax}}
\newcommand{\argmin}{\operatornamewithlimits{argmin}}

\title{A Few Notes on Density Operators, Expectation Values and Matrix Shapes}
\author{David Meyer \\ dmm@\{1-4-5.net,uoregon.edu\}}

\date{Last update: \today}							% Activate to display a given date or no date



\begin{document}
\maketitle

\section{Introduction}
These notes started life as an experiment in drawing matrices and their shapes (see Section \ref{sec:shapes}). However, it has evolved into a more ad-hoc collection of notes
covering a few topics in quantum mechanics. So its a WIP. We start with a review of Orthonormality, Completeness, and Projection...


\section{Orthonormality, Completeness, and Projection}
As we saw above, unitary matrices are matrices which satisfy 

\begin{equation}
\mathbf{U}^{-1} = \mathbf{U}^{\dagger}
\label{eqn:inverse}
\end{equation}

\bigskip
\noindent
Unitary matrices are ubiquitous and important in quantum mechanics, in 
particular because they have the following unique and useful properties: Orthonormality, Completeness, and Projection \cite{cresser2007}. 
We'll briefly look at each of these below\footnote{I will use the 
notation $\begin{pmatrix} x_1, \hdots, x_n \end{pmatrix}^{\text{T}}$ and $\begin{bmatrix} x_1, \hdots, x_n \end{bmatrix}^\text{T}$ interchangably in the following discussion.}.


\subsection{Orthornomality}
We can rewrite Equation \ref{eqn:inverse} as

\begin{equation}
\mathbf{U}^{\dagger} \mathbf{U} = \mathbf{I}
\label{eqn:dagger}
\end{equation}

\bigskip
\noindent
where \textbf{I} is the \emph{identity} matrix. What Equation \ref{eqn:dagger} is really telling us is that the columns of the matrix \textbf{U} form a set of orthnormal vectors.

\bigskip
\noindent
Note that we can interpret a matrix as a row vector where the entries are the columns  $\mathbf{v}_i$ of \textbf{U}. That is

\begin{equation*}
\mathbf{U} = \begin{bmatrix} \mathbf{v}_1 & \mathbf{v}_2 & \hdots & \mathbf{v}_{N} \end{bmatrix}
\end{equation*}

\bigskip
\noindent
Similarly, $\mathbf{U}^{-1}$ can be written as a column vector where the entries are the row vectors $ \mathbf{v}_{i}^\dagger$:

\begin{equation*}
\mathbf{U}^{-1} = \mathbf{U}^{\dagger} = \begin{bmatrix} \mathbf{v}_1^\dagger \\ \mathbf{v}_2^\dagger \\ \vdots  \\ \mathbf{v}_{N}^\dagger  \end{bmatrix}
\end{equation*}

\bigskip
\noindent
Now we can see that 

\begin{flalign*}
\mathbf{U}^{\dagger} \mathbf{U} &= \begin{bmatrix} \mathbf{v}_1^\dagger \\ \mathbf{v}_2^\dagger \\ \vdots  \\ \mathbf{v}_{N}^\dagger  \end{bmatrix}
 \begin{bmatrix} \mathbf{v}_1 & \mathbf{v}_2 & \hdots & \mathbf{v}_{N} \end{bmatrix}  \\
 &= 
 \begin{bmatrix}  
 \mathbf{v}_1^\dagger  \cdot  \mathbf{v}_1 & \mathbf{v}_1^\dagger  \cdot  \mathbf{v}_2 &  \mathbf{v}_1^\dagger  \cdot  \mathbf{v}_3 & \hdots & \mathbf{v}_1^\dagger  \cdot  \mathbf{v}_N \\
 \mathbf{v}_2^\dagger  \cdot  \mathbf{v}_1 & \mathbf{v}_2^\dagger  \cdot  \mathbf{v}_2 &  \mathbf{v}_2^\dagger  \cdot  \mathbf{v}_3 & \hdots & \mathbf{v}_2^\dagger  \cdot  \mathbf{v}_N \\
 \mathbf{v}_3^\dagger  \cdot  \mathbf{v}_1 & \mathbf{v}_3^\dagger  \cdot  \mathbf{v}_2 &  \mathbf{v}_3^\dagger  \cdot  \mathbf{v}_3 & \hdots & \mathbf{v}_3^\dagger  \cdot  \mathbf{v}_N \\
 \vdots & \vdots & \vdots &\ddots & \vdots \\
 \mathbf{v}_N^\dagger  \cdot  \mathbf{v}_1 & \mathbf{v}_N^\dagger  \cdot  \mathbf{v}_2 &  \mathbf{v}_N^\dagger  \cdot  \mathbf{v}_3 & \hdots & \mathbf{v}_N^\dagger  \cdot  \mathbf{v}_N 
 \end{bmatrix} \\
 &= \mathbf{I}
\end{flalign*}

\noindent
or in Dirac notation \cite{2000RPPh...63.1893G}

\begin{flalign*}
\mathbf{U}^{\dagger} \mathbf{U} &= \begin{bmatrix} \mathbf{v}_1^\dagger \\ \mathbf{v}_2^\dagger \\ \vdots  \\ \mathbf{v}_{N}^\dagger  \end{bmatrix}
 \begin{bmatrix} \mathbf{v}_1 & \mathbf{v}_2 & \hdots & \mathbf{v}_{N} \end{bmatrix} \\
 &= 
 \begin{bmatrix} 
 \bra{v_1} \\  \bra{v_2} \\ \vdots \\ \bra{v_N} 
\end{bmatrix}
 \begin{bmatrix} 
 \ket{v_1} &  \ket{v_2} & \hdots & \ket{v_N} 
\end{bmatrix} \\
&= 
\begin{bmatrix}  
\braket{v_1 | v_1} & \braket{v_1 | v_2} & \braket{v_1 | v_3} & \hdots & \braket{v_1 | v_N} \\
\braket{v_2 | v_1} & \braket{v_2 | v_2} & \braket{v_2 | v_3} & \hdots & \braket{v_2 | v_N} \\
\braket{v_3 | v_1} & \braket{v_3 | v_2} & \braket{v_3 | v_3} & \hdots & \braket{v_3 | v_N} \\
\vdots & \vdots & \vdots & \ddots &   \vdots \\
\braket{v_N | v_1} & \braket{v_N | v_2} & \braket{v_N | v_3} & \hdots & \braket{v_N | v_N} 
\end{bmatrix}  \\
&= \mathbf{I}
\end{flalign*}

\bigskip
\noindent
What we can notice\footnote{$\delta_{ij}$ is the Kronecker Delta function \cite{wiki:kronecker_delta}, $\delta_{ij} = \bigg \{
\begin{array}{ll}
1 & {\rm when  ~}  i = j \\
0 & {\rm when  ~}  i \ne j 
\end{array}$}
here is that since $(\mathbf{U}^{\dagger} \mathbf{U})_{ij} = (\mathbf{U}^{-1} \mathbf{U})_{ij} = \delta_{ij}$,  
the columns of \textbf{U} can be written as the inner product  $\braket{v_i | v_j} = \delta_{ij}$. Said another way,
the vectors $v_i$ form an orthonormal set. In particular, if $\mathbf{V} = \{v_j\}$ is an orthonormal set, then for  $v_i, v_j \in \mathbf{V}$,
the inner product $\braket{v_i | v_j} = \delta_{ij}$. See Section \ref{sec:shapes} for a brief discussion on matrix shapes.

\subsection{Completeness}
From $\mathbf{U}^\dagger \mathbf{U} = \mathbf{I}$ we saw that we could derive orthonormality. But we also expect that $\mathbf{U} \mathbf{U}^\dagger = \mathbf{I}$.
It turns out that we can get something interesting by observing this. In particular 

\begin{flalign*}
\mathbf{U} \mathbf{U}^\dagger &= \begin{bmatrix} \ket{v_1} &  \ket{v_2} & \ket{v_3} & \hdots & \ket{v_N} \end{bmatrix} 
\begin{bmatrix} \bra{v_1} \\ \bra{v_2} \\ \bra{v_3} \\ \vdots \\ \bra{v_N} \end{bmatrix}
\end{flalign*}

\bigskip
\noindent
If we multiply this out we find that

\begin{equation}
\ket{v_1}\bra{v_1} + \ket{v_2} \bra{v_2} + \cdots + \ket{v_N} \bra{v_N} = \sum\limits_{i = 1}^{N} \ket{v_i} \bra{v_i} = \mathbf{I}
\label{eqn:completeness}
\end{equation}

\noindent
Equation \ref{eqn:completeness} is known as the \emph{completeness} relation. 

\bigskip
\noindent
Completeness turns out to be useful and is a sort of a "dual" of orthonormality. While orthonormality is kind of an "inner product" 
($\mathbf{U}^{\dagger}\mathbf{U}$), completeness is like an outer product in that $\mathbf{U}\mathbf{U}^\dagger$ is a sum 
over $i$ of $\ket{v_i}\bra{v_i}$, although the shapes might be seen as reversed (see Section \ref{sec:shapes} on shapes). 

\subsection{Projection}
To get an idea of what projection is all about, consider the expansion of a vector into components in a basis:

\begin{equation}
\ket{w} =  \sum\limits_{i = 1}^{N} w_i \ket{v_i}
\label{eqn:ket_w}
\end{equation}

\bigskip
\noindent
Now, if the set of vectors basis vectors  $\{v_i\}$ are orthonormal, then we know that 

\begin{equation*}
w_i  =  \braket{v_i  | w}
\end{equation*}

\bigskip
\noindent
and substituting back into Equation \ref{eqn:ket_w} we get

\begin{equation*}
\ket{w} =  \sum\limits_{i = 1}^{N}  \braket{v_i  | w} \ket{v_i}
\end{equation*}

\bigskip
\noindent
Interestingly, there is another way to derive this result:  use the completeness relation, which is simply a fancy but useful way to write \textbf{I}:

\begin{equation*}
\ket{w} =  \mathbf{I} \cdot \ket{w} =  \Bigg (\sum\limits_{i = 1}^{N}  \ket{v_i} \bra{v_i} \Bigg ) \ket{w} = \sum\limits_{i = 1}^{N}  \ket{v_i} \braket{v_i | w}
\end{equation*} 

\bigskip
\noindent
In words, we were able to use the completeness relation to project a vector onto its components in a particular basis.

\bigskip
\noindent
For example, we know that for vectors $\ket{\alpha}$ and $\ket{\beta}$, we can take the inner product between them by using their components in a basis $\{v_i\}$:

\bigskip
\begin{equation*}
\braket{\alpha  | \beta} = \sum\limits_{i = 1}^{N} a_i^* b_i
\end{equation*} 

\bigskip
\noindent
where $a_i = \braket{v_i | \alpha}$ and $b_i = \braket{v_i | \beta}$. Interestingly, we can again derive this using the completeness relation:

%
% Thanks to Alan Barrett, Job Sniders, Pierre Francois and  Nikolai Matni for their suggestions to make 
% my "comment" code better (essentially wrap in in an array)
%
%  Sat Dec 15 10:41:23 PST 2018
%

\begin{equation*}
\begin{array}{rcll}
\braket{\alpha | \beta} 
&=& \bra{\alpha}  \mathbf{I} \ket{\beta}                                                                           & \quad \mathrel{\#} \braket{\alpha | \beta} = \bra{\alpha} \ket{\beta} = \bra{\alpha}  \mathbf{I} \ket{\beta} \\
&=& \bra{\alpha} \Bigg (\sum\limits_{i = 1}^{N}  \ket{v_i} \bra{v_i} \Bigg )  \ket{\beta}    & \quad \mathrel{\#} \sum\limits_{i = 1}^{N}  \ket{v_i} \bra{v_i}   = \mathbf{I}  \;\; (\text{Equation } \ref{eqn:completeness}) \\
&=& \sum\limits_{i = 1}^{N} \braket{\alpha | v_i} \braket{v_i | \beta}                               & \quad \mathrel{\#}  \text{rearrange} \\
&=& \sum\limits_{i = 1}^{N} \braket{v_i | \alpha}^* \braket{v_i | \beta}                            & \quad  \mathrel{\#} \braket{a | b} = \braket{ b | a}^* \text{so} \braket{\alpha | v_i}  = \braket{v_i | \alpha}^* \\
&=& \sum\limits_{i = 1}^{N} a_i^*b_i                                                                               & \quad \mathrel{\#} a_i^* =  \braket{v_i | \alpha}^*  \text{ and } b_i = \braket{v_i | \beta}
\end{array}
\end{equation*}


\bigskip
\section{Expectation Values}
\label{sec:ev}
Consider an observable \textbf{A}  in the pure state $\ket{\psi}$.  The expectation value $\braket{A}_{\psi}$ is given by

\begin{flalign}
\braket{A}_{\psi} &= \bra{\psi} A \ket{\psi} 
\label{eqn:exp}
\end{flalign}

\bigskip
\noindent
where $\dim (\bra{\psi}) = 1 \times n$, $\dim (A) = n \times n$, and $\dim (\ket{\psi}) = n \times 1$. 

\bigskip
\noindent
So why is $\braket{A}_{\psi}$ an expectation? 
Well, first, if $A$ is an observable for a system with a discrete set of values $\{a_1, a_2, \hdots, a_N\}$,
then this observable is represented by a Hermitean operator $\hat{A}$ that has these discrete values as its eigenvalues, and associated eigenstates 
$\{\ket{a_n}\}$, for $n = 1,2,3,\hdots$ satisfying the eigenvalue equation $\hat{A}\ket{a_n} = a_n \ket{a_n}$. I drop the "hat" in most of the below.

\bigskip
\noindent

First, observe that $ \bra{a_n} A = a_n \bra{a_n}$. Why? 

\begin{equation}
\begin{array}{rcllll}
A \ket{a_n}
&=& a_n \ket{a_n}                                                                                                       && \quad \mathrel{\#} \text{eigenvalue equation for $A$ } (A \textbf{v} = \lambda \mathbf{v})  \\
&\implies& (A \ket{a_n})^{\dagger}               &= (a_n \ket{a_n})^{\dagger}                    & \quad \mathrel{\#} \text{conjugate transpose both sides} \\
&\implies& \ket{a_n}^{\dagger} A^{\dagger} &= \ket{a_n}^{\dagger} a_n^{\dagger}       & \quad \mathrel{\#} (AB)^\dagger = B^\dagger A^\dagger \\
&\implies& \ket{a_n}^{\dagger} A^{\dagger} &= a_n^{\dagger} \ket{a_n}^{\dagger}       & \quad \mathrel{\#} \text{rearrange ($a_n^\dagger$ is a scalar)} \\
&\implies& \ket{a_n}^{\dagger} A                 &= a_n^{\dagger} \ket{a_n}^{\dagger}       & \quad \mathrel{\#} \text{$A$ is Hermitian so $A = A^\dagger$} \\
&\implies& \ket{a_n}^{\dagger} A                 & = a_n^{*} \ket{a_n}^{\dagger}                 & \quad \mathrel{\#} a_n^{\dagger} = a_n^* \; \text{($a_n$ is a scalar)} \\
&\implies& \bra{a_n} A                                 &= a_n^{*} \bra{a_n}                                  & \quad \mathrel{\#} \ket{a_n}^{\dagger} = \bra{a_n} \\
&\implies& \bra{a_n} A                                 &= a_n  \bra{a_n}                                      & \quad \mathrel{\#} a_n* = a_n
\label{eqn:aketan}
\end{array}
\end{equation}

\bigskip
\noindent
But why does $a_n^* = a_n$ (last line of (\ref{eqn:aketan}))? Well, consider

\bigskip
\begin{equation}
\begin{array}{lclll}
A X
&=& \lambda X                                                                                              && \quad \mathrel{\#} \text{eigenvalue equation}  \\
&\implies& X^{\dagger} A^\dagger   &= X^{\dagger} \lambda^{\dagger}           & \quad \mathrel{\#} (AB)^\dagger = B^\dagger  A^\dagger \\
&\implies& X^{\dagger} A^\dagger   &= \lambda^{\dagger} X^{\dagger}           & \quad \mathrel{\#} \text{rearrange ($\lambda^\dagger$ is a scalar)} \\
&\implies& X^{\dagger} A^\dagger   &= \lambda^{*} X^{\dagger}                      & \quad \mathrel{\#} \lambda^\dagger = \lambda^* \; \text{($\lambda$ is a scalar)} \\
&\implies& X^{\dagger} A                 &= \lambda^{*} X^{\dagger}                      & \quad \mathrel{\#} A^\dagger = A  \text{ since $A$ is Hermitian} \\
&\implies& X^{\dagger} A                 &= X^{\dagger}   \lambda^{*}                    & \quad \mathrel{\#} \text{rearrange} \\
&\implies& X^\dagger  A X               &= X^{\dagger}  \lambda^{*} X                  & \quad \mathrel{\#} \text{multiply both sides by $X$}
\end{array}
\label{eqn:ax}
\end{equation}

\bigskip
\noindent
Now notice that if we multiply both sides of the original eigenvalue equation ($A X = \lambda X$) by $X^{\dagger}$ we get 
$X^{\dagger} AX = X^{\dagger} \lambda X$.  We know from (\ref{eqn:ax}) that $X^\dagger  A X = X^{\dagger}  \lambda^{*} X$
and therefore that $X^{\dagger}  \lambda^{*} X = X^{\dagger}  \lambda X $.
This implies that $\lambda^{*} = \lambda$, so $\lambda \in \mathbb{R}$.
Similarly $a_n^* = a_n$ so $a_n \in \mathbb{R}$.

\bigskip
\noindent
Another way to look at this is to assume the computational basis\footnote{The approach taken in (\ref{eqn:aketan}) doesn't seem to require this assumption.}
and then

\begin{equation*}
\begin{array}{rcll}
 \bra{a_n} A
&=& a_n \bra{n} A  
    & \quad \mathrel{\#} \bra{a_n}   =  a_n \begin{bmatrix} 0 & \hdots & 1 & \hdots 0 \end{bmatrix} = a_n \bra{n}    \\
&=& a_n \bra{n} A^{\dagger}
    & \quad \mathrel{\#} \text{$A$ is Hermitian so $A = A^{\dagger}$}    \\
&=& a_n \bra{n} \begin{bmatrix} \bra{a_1}  \\  \vdots \\ \bra{a_n} \\ \vdots \\  \bra{a_N}  \end{bmatrix}      
    & \quad \mathrel{\#}  \ A^{\dagger} =  \begin{bmatrix} \bra{a_1}  \\  \vdots \\ \bra{a_n} \\ \vdots \\  \bra{a_N}  \end{bmatrix}  \\
&=& a_n \begin{bmatrix} 0 & \cdots & 1 & \cdots 0 \end{bmatrix}  \begin{bmatrix} \bra{a_1}  \\  \vdots \\ \bra{a_n} \\ \vdots \\  \bra{a_N}  \end{bmatrix}  
    & \quad \mathrel{\#}  \bra{n}  = \begin{bmatrix} 0 & \hdots & 1 & \hdots 0 \end{bmatrix}  \\
&=& a_n \bra{a_n} 
    & \quad \mathrel{\#}  \bra{n} \text{ selects the $n^{th}$ element of $A^{\dagger}, \bra{a_n}$}
\end{array}
\end{equation*}

\bigskip
\noindent
In any event, now we have $\bra{a_n} A = a_n \bra{a_n}$. So we can observe that 

\begin{equation*}
\begin{array}{rcll}
\braket{A}_{\psi} 
&=& \braket{\psi |A | \psi}                             
    & \qquad \mathrel{\#} \text{definition of $\braket{A}_{\psi}$ for \emph{pure} state $\ket{\psi}$} \\
&=& \braket{\psi  | IA | \psi}                             
    & \qquad \mathrel{\#} I \cdot A = A \\
&=& \braket{\psi | \bigg (\sum\limits_{n = 1}^{N} \ket{a_n} \bra{a_n} \bigg ) A | \psi}                      
    & \qquad \mathrel{\#} \sum\limits_{n = 1}^{N} \ket{a_n} \bra{a_n} = \mathbf{I} \;\;  \text{\big (Equation \ref{eqn:completeness}} \big ) \\
&=& \sum\limits_{n = 1}^{N} \braket{\psi | a_n} \braket{a_n | A | \psi}                 
     & \qquad \mathrel{\#} \text{rearrange} \\
&=& \sum\limits_{n = 1}^{N} \braket{\psi | a_n} a_n   \braket{a_n | \psi}
    & \qquad \mathrel{\#}  \bra{a_n} A = a_n \bra{a_n} \text{ (see above)} \\
&=& \sum\limits_{n = 1}^{N} \braket{\psi | a_n}  \braket{a_n | \psi} a_n
    & \qquad \mathrel{\#}  \text{rearrange} \\
&=& \sum\limits_{n = 1}^{N} | \braket{\psi | a_n}|^2  a_n
    & \qquad \mathrel{\#} |\braket{\psi | a_n}|^2  = \braket{\psi | a_n}  \braket{\psi | a_n}^*  = \braket{\psi | a_n}  \braket{a_n | \psi}  \\
&=& \sum\limits_{n = 1}^{N} p(a_n)  a_n
    & \qquad \mathrel{\#}  |\braket{\psi | a_n}|^2 = p(a_n),  \text{the probability of observing eigenvalue $a_n$} \\
&=& \sum\limits_{n = 1}^{N} \frac{N_n}{N}  a_n
    & \qquad \mathrel{\#}  \text{$N_n$ is the number of times $a_n$  has been measured} \\
&=& \E [A]
    & \qquad \mathrel{\#}  \E [X] = \sum\limits_{n = 1}^{N} p(X_n) X_n
\end{array}
\end{equation*}

\bigskip
\noindent
So the expectation value for the result of a measurement represented by a self-adjoint operator $A$, $\braket{A}_{\psi}$,
 is the weighted average of all possible outcomes under $A$, that is, $\E [A]$.

\section{Shapes}
\label{sec:shapes}
One way to visualize $\braket{A}_{\psi}$ is

\bigskip
\begin{flalign*}
\braket{A}_{\psi} & \rightarrow
\underbrace{\begin{bmatrix} \hdots \hdots  \hdots \end{bmatrix}}_{1 \times n} 
\underbrace{
\begin{bmatrix}
\hdots & \hdots  & \hdots \\
\vdots & \ddots  &  \vdots \\
 \hdots & \hdots  & \hdots 
\end{bmatrix}}_{n \times n} 
\underbrace{\begin{bmatrix} \hdots \\ \vdots \\  \hdots  \end{bmatrix}}_{n \times 1} \\
& \rightarrow c
\end{flalign*}

\noindent
where $c \in \mathbb{C}$.

\bigskip
\noindent
The \emph{density operator} $\rho$ for pure state $\ket{\psi}$ is given by $\rho = \ket{\psi} \bra{\psi}$. The shape of $\rho$ is 

\bigskip
\begin{flalign*}
\rho \rightarrow
\underbrace{\begin{bmatrix} \hdots \\ \vdots \\  \hdots  \end{bmatrix}}_{n \times 1} 
\underbrace{\begin{bmatrix} \hdots \hdots  \hdots \end{bmatrix}}_{1 \times n} \rightarrow
\underbrace{
\begin{bmatrix}
\hdots & \hdots  & \hdots \\
\vdots & \ddots  &  \vdots \\
 \hdots & \hdots  & \hdots 
\end{bmatrix}}_{n \times n} 
\end{flalign*}

\bigskip
\noindent
The shape of the inner product of two $n \times 1$ column vectors  $\braket{\mathbf{u}, \mathbf{v}} = \braket{u | v} =  \mathbf{u}^{\text{T}}\mathbf{v}$ is 

\bigskip
\begin{flalign*}
\mathbf{u}^{\text{T}}\mathbf{v}  & \rightarrow
\underbrace{\begin{bmatrix} \hdots \hdots  \hdots \end{bmatrix}}_{1 \times n}  \underbrace{\begin{bmatrix} \hdots \\ \vdots \\  \hdots  \end{bmatrix}}_{n \times 1}  \rightarrow c
\end{flalign*}

\bigskip
\noindent
where $c \in \mathbb{C}$. The shape of the outer product $\mathbf{u} \otimes \mathbf{v} = \ket{u} \bra{v} = \mathbf{u} \mathbf{v}^{\text{T}}$ is 

\bigskip
\begin{flalign*}
\mathbf{u} \mathbf{v}^{\text{T}}  \rightarrow
\underbrace{\begin{bmatrix} \hdots \\ \vdots \\  \hdots  \end{bmatrix}}_{n \times 1} 
\underbrace{\begin{bmatrix} \hdots \hdots  \hdots \end{bmatrix}}_{1 \times n} \rightarrow
\underbrace{
\begin{bmatrix}
\hdots & \hdots  & \hdots \\
\vdots & \ddots  &  \vdots \\
 \hdots & \hdots  & \hdots 
\end{bmatrix}}_{n \times n} 
\end{flalign*}


\bigskip
\section{The Density $\rho$ and the Trace of an Operator $D$}
So $\rho$ is an $n \times n$ \emph{linear operator} with $\text{Tr}(\rho)  = \text{Tr}(\ket{\psi} \bra{\psi}) = \braket{\psi | \psi}$. 
In addition, $\text{Tr}(\ket{\psi_i} \bra{\psi_i}) = \braket{\psi_i | \psi_i}  = \delta_{ii} = 1$, and if $\{\ket{\psi_i}\}$ is an
orthonormal basis then $\text{Tr}(\ket{\psi_i} \bra{\psi_j}) = \braket{\psi_i | \psi_j}  = \delta_{ij}$.

\bigskip
\noindent
The density matrix \cite{porter04052011} $\rho$ has the following important properties:

\begin{equation*}
\begin{array}{lll}
& \text{Projection:}            & \quad \rho^2 = \rho           \\
& \text{Hermiticity:}           & \quad \rho^\dagger = \rho \\
& \text{Normalization:}      & \quad \text{Tr}(\rho) = 1    \\
& \text{Positivity:}              & \quad \rho \geq 1
\end{array}
\end{equation*}

\noindent
The \emph{trace} of an operator $D$,  $\text{Tr}(D)$, is defined to be $\text{Tr}(D) = \sum\limits_{i = 1}^{n} \bra{n} D \ket{n}$.
Now, suppose $D = \ket{\psi} \bra{\phi}$. Then we can see that $\text{Tr}(D) = \text{Tr}(\ket{\psi} \bra{\phi}) = \braket{\phi | \psi}$ as follows:

\begin{equation*}
\begin{array}{rcll}
\text{Tr}(D) 
&=& \sum\limits_{n = 1}^{N} \bra{n} D \ket{n}                                                          & \quad  \mathrel{\#} \text{definition of $\text{Tr}(D)$}  \\
&=& \sum\limits_{n = 1}^{N} \bra{n} (\ket{\psi} \bra{\phi})  \ket{n}                            & \quad  \mathrel{\#} D = \ket{\psi} \bra{\phi} \\
&=& \sum\limits_{n = 1}^{N} \bra{n} \ket{\psi} \bra{\phi}\ket{n}                                & \quad  \mathrel{\#} \text{drop parens} \\
&=& \sum\limits_{n = 1}^{N} \braket{n | \psi} \braket{\phi | n}                                  & \quad  \mathrel{\#} \braket{a | b} = \bra{a} \ket{b} \\
&=& \sum\limits_{n = 1}^{N} \braket{\phi | n} \braket{n | \psi}                                  & \quad  \mathrel{\#} \text{rearrange} \\
&=& \bra{\phi}  \Bigg (  \sum\limits_{n = 1}^{N}  \ket{n} \bra{n} \Bigg )  \ket{\psi}    & \quad  \mathrel{\#} \text{neither $\phi$ nor $\psi$ depend on $n$}\\
&=& \braket{\phi | I | \psi}                                                                                        & \quad  \mathrel{\#} \sum\limits_{n = 1}^{N} \ket{n} \bra{n} = \mathbf{I}  \;\;  \text{\big (Equation \ref{eqn:completeness}} \big ) \\
&=& \bra{\phi } \ket{\psi}                                                                                          & \quad  \mathrel{\#} \bra{\phi } I = \bra{\phi} \text{ and } I \ket{\psi}  = \ket{\psi}  \\
&=& \braket{\phi | \psi}                                                                                            & \quad  \mathrel{\#} \braket{\phi | \psi} = \bra{\phi} \ket{\psi}
\end{array}
\end{equation*}


\bigskip
\noindent
So the trace of the outer product  $\ket{\psi} \bra{\phi}$,  $\text{Tr}(\ket{\psi} \bra{\phi})$, is the inner product  $\braket{\phi | \psi}$.

\bigskip
\noindent
A simple theorem relates the expectation value of an observable $A$ in a state represented by a density matrix $\rho$ to the trace of $A$:

\begin{equation}
\braket{A}_\rho = \text{Tr}(\rho A)
\label{eqn:braket_A}
\end{equation}

\bigskip
\noindent
The proof of Equation \ref{eqn:braket_A} is also pretty simple:

\begin{equation*}
\begin{array}{rcll}
\text{Tr}(\rho A) 
&=& \text{Tr}(\ket{\psi} \bra{\psi}  A)                                                                      & \quad \mathrel{\#} \rho \equiv \ket{\psi} \bra{\psi} \\
&=& \sum\limits_{n = 1}^{N} \bra{n} \ket{\psi} \bra{\psi} A \ket{n}                          & \quad \mathrel{\#} \text{definition of Tr($\cdot$)} \\
&=& \sum\limits_{n = 1}^{N} \braket{n | \psi} \bra{\psi} A \ket{n}                           & \quad \mathrel{\#}  \braket{n | \psi}  =  \bra{n} \ket{\psi}  \\
&=& \sum\limits_{n = 1}^{N} \bra{\psi} A \ket{n} \braket{n | \psi}                           & \quad \mathrel{\#} \text{rearrange} \\
&=& \bra{\psi}  A \Bigg (\sum\limits_{n = 1}^{N} \ket{n} \bra{n} \Bigg ) \ket{\psi}   & \quad \mathrel{\#} \text{neither $A$ nor $\psi$ depend on $n$} \\
&=& \bra{\psi} A  \cdot I  \ket{\psi}                                                                         & \quad \mathrel{\#} \sum\limits_{n = 1}^{N} \ket{n} \bra{n} = \mathbf{I} \;\; \text{\big (Equation \ref{eqn:completeness}} \big )  \\
&=& \bra{\psi} A \ket{\psi}                                                                                      & \quad  \mathrel{\#} \mathbf{A} \cdot \mathbf{I} = \mathbf{A} \\
&=& \braket{A}                                                                                                      & \quad \mathrel{\#} \braket{A}_{\psi} = \bra{\psi} A \ket{\psi}   \big (\text{Equation } \ref{eqn:exp} \big )
\end{array}
\end{equation*}

\bigskip
\section{A More General View of the Density Operator}
Consider an ensemble of identical quantum systems. The system has probability $w_i$ to be in quantum state $\ket{\psi_i}$.  Here $\braket{\psi_i | \psi_i} = 1$,  but
the states $\ket{\psi_i}$ aren't necessarily orthogonal to one another.   That means that out of all the examples in the ensemble, a fraction $w_i$  are in state $\ket{\psi_i}$, with
$w_i  >  0$ and $\sum\limits_{i} w_i = 1$.
\bigskip
\noindent
The expectation value for the result of a measurement represented by a self-adjoint operator $A$ is

\begin{equation}
\braket{A}_{\psi} = \sum\limits_{i} w_i \braket{\psi_i | A | \psi_i}
\label{eqn:braketA}
\end{equation}

\bigskip
\noindent
We can write the expectation value in a different way using a basis $\ket{K}$ as 

\begin{equation*}
\begin{array}{rcll}
\braket{A}_{\psi}
&=& \sum\limits_{i} w_i \braket{\psi_i | A | \psi_i}                                                                          
     & \quad \mathrel{\#} \text{defintion of } \braket{A}_{\psi} , \text{Equation } \ref{eqn:braketA} \\
&=& \sum\limits_{i} w_i \braket{\psi_i | I A I | \psi_i}                                                                      
     & \quad \mathrel{\#} \mathbf{A} = \mathbf{I} \cdot \mathbf{A} \cdot \mathbf{I} \\
&=& \sum\limits_{i} w_i \braket{\psi_i | \Big (  \sum\limits_{J} \ket{J} \bra{J} \Big ) |A | \Big ( \sum\limits_{K} \ket{K} \bra{K} \Big ) | \psi_i}   
     & \quad \mathrel{\#} \sum\limits_{J} \ket{J}\bra{J} = \mathbf{I}, \sum\limits_{K} \ket{K}\bra{K} = \mathbf{I} \\
&=& \sum\limits_{i} w_i \sum\limits_{J,K} \braket{\psi_i | J} \braket{J |A | K} \braket{K | \psi_i}    
     & \quad \mathrel{\#} \text{rearrange} \\
&=& \sum\limits_{i} w_i \sum\limits_{J,K}  \braket{K | \psi_i}   \braket{\psi_i | J} \braket{J |A | K} 
     & \quad \mathrel{\#} \text{rearrange} \\
&=&  \sum\limits_{J,K} \sum\limits_{i} w_i \braket{K | \psi_i}   \braket{\psi_i | J} \braket{J |A | K} 
     & \quad \mathrel{\#} \text{none of $A$, $J$, or $K$ depend on $i$} \\
&=&  \sum\limits_{J,K}  \braket{K | \Big ( \sum\limits_{i} w_i  \ket{\psi_i}\bra{\psi_i}  \Big) | J} \braket{J |A | K} 
     & \quad \mathrel{\#} \text{rearrange} \\
&=& \sum\limits_{J,K} \braket{K | \rho | J} \braket{J | A | K}
     & \quad \mathrel{\#} \rho \equiv \sum\limits_{i} w_i \ket{\psi_i} \bra{\psi_i} \\\
&=& \sum\limits_{K} \braket{K | \rho  I A | K}
     & \quad \mathrel{\#} \sum\limits_{J} \ket{J} \bra{J} = \mathbf{I} \\
&=& \sum\limits_{K} \braket{K | \rho A | K}
     & \quad \mathrel{\#}  \mathbf{I} \cdot \mathbf{A} = \mathbf{A} \\
&=&  \text{Tr}(\rho A)
     & \quad \mathrel{\#} \text{Tr}(D) = \sum\limits_{n} \braket{n | D | n}
\end{array}
\end{equation*}

\subsection{Properties of the Density Operator}
As mentioned above, there are several important properties of the density operator $\rho$.  The first of these is that $\text{Tr}(\rho) = 1$.
This follows from $w_i$ has $w_i  >  0$ and $\sum\limits_{i} w_i = 1$. 

\bigskip
\noindent
Next, $\rho$ is self-adjoint: $\rho^{\dagger} = \rho$. Because it is self-adjoint, 
$\rho$ has eigenvectors $\ket{J}$ with eigenvalues $\lambda_{J}$ and the eigenvectors form a basis for vector space. Thus $\rho$ has a standard spectral
representation

\begin{equation*}
\rho = \sum\limits_{J} \lambda_{J} \ket{J} \bra{J}
\end{equation*}

\bigskip
\noindent
We can express $\lambda_{J}$ as $\lambda_{J}  = \braket{J | \rho | J}$. Then

\bigskip
\begin{equation*}
\begin{array}{rcll}
\lambda_{J} 
&=& \braket{J | \rho | J}                                                                                   & \quad \mathrel{\#}  \\
&=& \braket{J | \Big ( \sum\limits_{i} w_i \ket{\psi_i} \bra{\psi_i} \Big ) | J}      & \quad \mathrel{\#}  \rho = \sum\limits_{i} w_i \ket{w_i} \bra{w_i}  \\
&=& \sum\limits_{i} w_i  \braket{J | \psi_i} \braket{\psi_i | J}                           & \quad \mathrel{\#}  \text{rearrange} \\
&=& \sum\limits_{i} w_i \braket{J | \psi_i} \braket{J | \psi_i}^*                         & \quad \mathrel{\#} \braket{J | \psi_i}^* = \braket{\psi_i | J} \\
&=& \sum\limits_{i} w_i |\braket{J | \psi_i}|^2                                                  & \quad \mathrel{\#} \braket{J | \psi_i} \braket{J | \psi_i}^* =  |\braket{J | \psi_i}|^2 
\end{array}
\end{equation*}

\bigskip
\noindent
Since $w_i > 0$ and $ |\braket{J | \psi_i}|^2 > 0$, each eigenvalue must be non-negative, that is, $\lambda_{J} \geq 0$. In addition, the trace of $\rho$ is the sum of its 
eigenvalues, so $\sum\limits_{J} \lambda_{J} = 1$. Since each eigenvalue is non-negative, $\lambda_{J} \leq 1$.



\bigskip
\noindent
Another way to see why $|\braket{a_n | \psi}|^2 = p(a_n)$:


\begin{equation*}
\begin{array}{rcll}
\ket{\psi}
&=& I \ket{\psi}                                                                                                   & \quad \mathrel{\#}  \mathbf{I} \cdot \mathbf{X} = \mathbf{X} \\
&=& \sum\limits_{n} \ket{a_n} \bra{a_n} \ket{\psi}                                              & \quad \mathrel{\#} \sum\limits_{n} \ket{a_n} \bra{a_n} = I \\
&=& \sum\limits_{n} \ket{a_n} \braket{a_n | \psi}                                               & \quad \mathrel{\#} \bra{a_n} \ket{\psi} = \braket{a_n | \psi} 
\end{array}
\end{equation*}

\bigskip
\noindent
So $\braket{a_n | \psi}$ is the amplitude of $\ket{a_n}$, making $| \braket{a_n | \psi} |^2 = p(a_n)$.

\bigskip
\section{Acknowledgements}

\newpage
\bibliographystyle{plain}
\bibliography{/Users/dmm/papers/bib/qc}
\end{document} 
