\documentclass[11pt, oneside]{article}   	% use "amsart" instead of "article" for AMSLaTeX format


% \usepackage{draftwatermark}
% \SetWatermarkText{Draft}
% \SetWatermarkScale{5}
% \SetWatermarkLightness {0.9} 
% \SetWatermarkColor[rgb]{0.7,0,0}


\usepackage{geometry}                		% See geometry.pdf to learn the layout options. There are lots.
\geometry{letterpaper}                   		% ... or a4paper or a5paper or ... 
%\geometry{landscape}                		% Activate for for rotated page geometry
%\usepackage[parfill]{parskip}    		% Activate to begin paragraphs with an empty line rather than an indent
\usepackage{graphicx}				% Use pdf, png, jpg, or eps� with pdflatex; use eps in DVI mode
								% TeX will automatically convert eps --> pdf in pdflat						
								% TeX will automatically convert eps --> pdf in pdflatex		
\usepackage{amssymb}
\usepackage{mathrsfs}
\usepackage{hyperref}
\usepackage{url}
\usepackage{subcaption}
\usepackage{authblk}
\usepackage{amsmath}
\usepackage{mathtools}
\usepackage{graphicx}
\usepackage[export]{adjustbox}
\usepackage{fixltx2e}
\usepackage{hyperref}
\usepackage{alltt}
\usepackage{color}
\usepackage[utf8]{inputenc}
\usepackage[english]{babel}
\usepackage{float}
\usepackage{bigints}
\usepackage{braket}
\usepackage{siunitx}

%
% so you can do e.g., \begin{bmatrix}[r] (or [c] or [l])
%

\makeatletter
\renewcommand*\env@matrix[1][c]{\hskip -\arraycolsep
  \let\@ifnextchar\new@ifnextchar
  \array{*\c@MaxMatrixCols #1}}
\makeatother

\newcommand{\argmax}{\operatornamewithlimits{argmax}}
\newcommand{\argmin}{\operatornamewithlimits{argmin}}

\begin{document}

\begin{equation*}
\begin{array}{lllll}
e^{i\phi}
&=&                   \cos \phi + i \sin \phi                    &\quad  \mathrel{\#} \text{Euler's Formula}                            \\
&\Rightarrow&  e^{i\pi} = \cos \pi + i \sin \pi          &\quad  \mathrel{\#} \text{set $\phi = \pi$}                              \\
&\Rightarrow&  e^{i\pi} = -1 + i \cdot 0                  &\quad  \mathrel{\#} \text{$\cos \pi = -1$ and $\sin \pi = 0$}  \\
&\Rightarrow&  e^{i\pi} = -1 + 0                            &\quad  \mathrel{\#} i \cdot 0 = 0                                            \\
&\Rightarrow&  e^{i\pi} = -1                                  &\quad  \mathrel{\#} \text{simplify}                                          \\
&\Rightarrow&  e^{i\pi} +1 = 0                              &\quad  \mathrel{\#} \text{Euler's Identity} 
\end{array}
\end{equation*}



\end{document} 

