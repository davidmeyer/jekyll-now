\documentclass[11pt, oneside]{article}   	% use "amsart" instead of "article" for AMSLaTeX format


% \usepackage{draftwatermark}
% \SetWatermarkText{Draft}
% \SetWatermarkScale{5}
% \SetWatermarkLightness {0.9} 
% \SetWatermarkColor[rgb]{0.7,0,0}


\usepackage{geometry}                		% See geometry.pdf to learn the layout options. There are lots.
\geometry{letterpaper}                   		% ... or a4paper or a5paper or ... 
%\geometry{landscape}                		% Activate for for rotated page geometry
%\usepackage[parfill]{parskip}    		% Activate to begin paragraphs with an empty line rather than an indent
\usepackage{graphicx}				% Use pdf, png, jpg, or eps� with pdflatex; use eps in DVI mode
								% TeX will automatically convert eps --> pdf in pdflat						
								% TeX will automatically convert eps --> pdf in pdflatex		
\usepackage{amssymb}
\usepackage{amsmath}
\usepackage{amsthm}
\usepackage{mathrsfs}
\usepackage{hyperref}
\usepackage{url}
\usepackage{subcaption}
\usepackage{authblk}
\usepackage{mathtools}
\usepackage{graphicx}
\usepackage[export]{adjustbox}
\usepackage{fixltx2e}
\usepackage{hyperref}
\usepackage{alltt}
\usepackage{color}
\usepackage[utf8]{inputenc}
\usepackage[english]{babel}
\usepackage{float}
\usepackage{bigints}
\usepackage{braket}
\usepackage{siunitx}

\theoremstyle{definition}
\newtheorem{thm}{Theorem}[section]
% \newtheorem{defn}[thm]{Definition}
\newtheorem{definition}{Definition}[section]
\newtheorem{example}{Example}[section]
% \newtheorem[thm]{exmp}
\newtheorem{proposition}{Proposition}[section]




\newcommand{\veq}{\mathrel{\rotatebox{90}{$=$}}}
\DeclareMathOperator{\bda}{\Big \downarrow}

\DeclareMathOperator{\mymod}{\text{mod}}

\DeclareMathOperator{\E}{\mathbb{E}}
\newcommand{\argmax}{\operatornamewithlimits{argmax}}
\newcommand{\argmin}{\operatornamewithlimits{argmin}}


\title{A Few Notes on Groups, Rings, and Fields}
\author{David Meyer \\ dmm@\{1-4-5.net,uoregon.edu\}}

\date{Last update: September 10, 2017}							% Activate to display a given date or no date



\begin{document}
\maketitle

\section{Introduction}
Suppose we want to solve an equation of the form

\begin{equation}
f(x) = x^{n-1} + a_{n-2}x^{n-2} + a_{n-3}x^{n-3} + \cdots + a_{1}x + a_0 = 0
\label{eqn:f(x)}
\end{equation}

\bigskip
\noindent
where the coefficients\footnote{Note that the largest degree term ($x^{n-1}$) has coefficient 1. This is called a \emph{monic} polynomial.}
$a_i \in \mathbb{Q}$.  We can notice quite a few interesting things about $f(x)$.  For example, if $R$ is a ring then ring of polynomials in $x$ 
with coefficients in $R$, denoted $R[x$],  consists of all formal sums

\begin{equation*}
f(x) = \sum\limits_{i = 0}^{\infty} a_ix^i
\end{equation*}

\bigskip
\noindent
where $a_i = 0$ for all but finitely many values of $i$.


\bigskip
\noindent
The fundamental theorem of algebra \cite{steed2014} tells us that for any $n > 0$ 
and arbitrary complex coefficients $a_{n-1}, \hdots, a_0 \in \mathbb{C}$ there is a complex solution
$x = \lambda \in \mathbb{C}$. If we iterate the process we find that 

\begin{equation}
f(x) = (x - \lambda_0)(x - \lambda_2) \cdot \hdots \cdot (x - \lambda_{n - 1}) = 0
\label{eqn:factorization}
\end{equation}

\bigskip
\noindent
for $\lambda_0, \lambda_2, \hdots \lambda_{n-1} \in \mathbb{C}$.  Here $f(x) = 0$ iff $x = \lambda_j$ for some
$j \in \{0,1,\hdots, n-1\}$.

\bigskip
\noindent
\textbf{Aside:} What is being assumed here? Well, we are assuming that if $r \cdot s = 0$ then either $r$ or $s$ (or both) equal zero. 
If $r \neq 0$ and $s \neq 0$ but $r \cdot  s = 0$ we call $r$ and $s$ \emph{zero divisors}. A commutative ring with no zero divisors
is called an \emph{intergral domain}\footnote{Saying that $F$ has no zero divisors is equivalent to saying that $F$ has a cancellation law.}. 
The canonical example of an integral domain is the integers $\mathbb{Z}$. 

\bigskip
\noindent
BTW,  why is $\mathbb{Z}$ not a field? Well, consider for example that $2 \in \mathbb{Z}$ but $\frac{1}{2} \notin \mathbb{Z}$ so not every 
non-zero $n \in \mathbb{Z}$ has an inverse in $\mathbb{Z}$ and so $\mathbb{Z}$  is not a field. Every \emph{finite} 
integral domain is a field however (Theorem \ref{thm:finite_id_is_a_field}).

\bigskip
\noindent
Note that if we have zero divisors then the factorization  shown in Equation \ref{eqn:factorization} might not find all of
the roots of $f(x)$ (values of $x$ for which $f(x) = 0$). Why? Consider the following example:

\begin{equation}
\begin{array}{rcll} 
x^2 + 5x + 6 \equiv 0 \text{ mod } 12
&\Rightarrow& (x + 2) \cdot (x + 3) \equiv 0 \text{ mod } 12 
\end{array}
\label{eqn:roots}
\end{equation}

\bigskip
\noindent
Here we can read off the roots $x \equiv -2 \text{ mod } 12 \Rightarrow x  = 10 \text{ mod } 12$ and 
$x \equiv -3 \text{ mod } 12 \Rightarrow x  = 9 \text{ mod } 12$. So we have two roots (mod 12) 
at $x = 9$ and $x = 10$. But are these all of the roots? Well, the answer is no. Consider
$f(1) \text{ mod } 12 \equiv (1^2 + 5 + 6) \text{ mod } 12 \equiv 0 \text{ mod } 12$. In addition,
$f(6) \text{ mod } 12 \equiv (36 + 30 +6) \text{ mod } 12 \equiv 72 \text{ mod } 12 \equiv 0 \text{ mod } 12$.

\bigskip
\noindent
So the roots of Equation \ref{eqn:roots} are $\{1,6,9,10\}$. Why were we only able to find
two of the roots (9 and 10) by factoring? It is because the ring $\mathbb{Z}_{12}$ has 
zero divisors. What are the zero divisors in $\mathbb{Z}_{12}$? Well


\begin{equation}
\begin{array}{lrcll} 
\,\; 2   \cdot 6     \equiv     12 \text{ mod } 12 \equiv 0 \text{ mod } 12 \\
\,\; 3   \cdot 4     \equiv     12 \text{ mod } 12 \equiv 0 \text{ mod } 12 \\
\,\; 4   \cdot 3     \equiv    12 \text{ mod } 12 \equiv 0 \text{ mod } 12 \\
\,\; 6   \cdot 2     \equiv    12 \text{ mod } 12 \equiv 0  \text{ mod } 12 \\
\,\; 8   \cdot 3     \equiv     24 \text{ mod } 12 \equiv 0  \text{ mod } 12 \\
\,\; 9   \cdot 8     \equiv     72 \text{ mod } 12 \equiv 0  \text{ mod } 12 \\
10      \cdot 6     \equiv     60 \text{ mod } 12 \equiv 0  \text{ mod } 12 \\
\end{array}
\end{equation}


\bigskip
\noindent
Note that if $p$ is a prime then $\mathbb{Z}_{p}$ is an integral domain (has no zero divisors).

\bigskip
\noindent
So the condition we need is that the set of coefficients are drawn from an integral domain. 

\begin{thm}
Every field $F$ is an integral domain.
\label{thm:integral_domain}
\end{thm}

\noindent
\textbf{Proof:} Recall that if $F$ is a field then each non-zero $r \in F$ has an inverse $r^{-1}$. 
So suppose $r,s \in F$ and $r \neq 0$ such that $r \cdot s = 0$. Then the claim is that $s = 0$.
Why? Consider


\begin{equation}
\begin{array}{rcll} 
r \cdot s 
&=& 0                                                                                &\quad  \mathrel{\#} \text{assumption with $r \neq 0$}         \\
&\Rightarrow& r^{-1} \cdot (r \cdot s) = r^{-1} \cdot 0        &\quad  \mathrel{\#} \text{multiply both sides by $r^{-1}$}    \\
&\Rightarrow&  r^{-1} \cdot (r \cdot s) =  0                        &\quad  \mathrel{\#} x \cdot 0 = 0                                          \\
&\Rightarrow&  (r^{-1} \cdot r) \cdot s =  0                        &\quad  \mathrel{\#} \text{multiplication is associative}         \\
&\Rightarrow&  s = 0                                                         &\quad  \mathrel{\#} r^{-1} \cdot r = 1
\end{array}
\end{equation}

\bigskip
\noindent
So $r$ is not a zero divisor. But every non-zero element $r$ of the field $F$ has an inverse ($r$ is a "unit")
so $F$ has no zero divisors and is by definition an integral 
domain. $\square$

\bigskip
\noindent
Theorem \ref{thm:finite_id_is_a_field} below shows a limited version of this theorem in the other direction: 
Every finite integral domain is a field.

%\newpage
\section{Splitting Fields}
Recall that the ring of polynomials over a field $F$, denoted $F[x]$, is defined as follows\footnote{I reversed the order
of Equation \ref{eqn:f(x)} since its an easier form to work with. In addition, we can assume $a_{n-1} = 1$ since $f(x)$ 
is monic.}


\begin{definition}
\textbf{Polynomial Ring over $\mathbf{F}$:} The polynomial ring over $F$ is defined as 

\begin{equation*}
F[x] = \{f(x) \mid f(x) = a_0x^0 + a_1x^1 +a_2x^2 + \hdots + a_{n-1}x^{n-1} \}
\end{equation*}

\bigskip
\noindent
with $a_i \in F$ and with the usual ring properties.

\bigskip
\noindent
Aside on notation: while $F[x]$ is defined as above, $F(x)$ is defined differently. 

\begin{equation*}
F(x) = \Bigg  \{\frac{p(x)}{q(x)} \; \bigg \lvert  \; p(x),q(x) \in F[x]  \Bigg \}
\end{equation*}
\label{def:polynomials}
\end{definition}

\noindent
There doesn't seem to be any standard convention as to the definitions of  $F[x]$ vs. $F(x)$. I've seen $F(x)$ used to mean what I 
defined as $F[x]$ above.

\begin{definition}
\textbf{Splitting Field:} Let $f \in F[x]$. An extension field\footnote{$E$ is an extension field of $F$ if $F$ is a subfield of $E$.} 
$E$ of $F$, written $E/F$,  is called a \emph{splitting field} for $f$ over $F$ if the following two 
conditions are satisfied:

\begin{enumerate}
\item $f$ factors into linear polynomials ("splits" or "splits completely") in $E [x]$
\item $f$ does not split completely in $K[x]$ for any $F \subsetneq K \subsetneq E$
\end{enumerate}
\label{def:splitting_field}
\end{definition}

\subsection{The Evaluation Homomorphism: $\boldsymbol{e: F[x] \rightarrow F[\alpha]}$}
TBD

%\newpage
\subsection{Examples}
\begin{example}
$\mathbb{Q}[\sqrt{2}]$ is a splitting field for $x^2 - 2$ over $\mathbb{Q}$.  \\

\noindent
Why? Consider the conditions in Definition \ref{def:splitting_field}: First, 
the polynomial $x^2 -2$ factors into linear polynomials ("splits")  in $\mathbb{Q}[\sqrt{2}][x]$: $x^2 -2 = (x - \sqrt{2})(x + \sqrt{2})$. 
To see this, consider

\begin{equation*}
\begin{array}{rcll}
\mathbb{Q}[\sqrt{2}]
&=& a_0(\sqrt{2})^0 +  a_1(\sqrt{2})^1 + a_2(\sqrt{2})^2 +  a_3(\sqrt{2})^3  + a_4(\sqrt{2})^4 +  \cdots + a_{n-1}(\sqrt{2})^{n-1}  
     &\mathrel{\#} \text{defn $\mathbb{Q}[\sqrt{2}]$} \\
&=& a_0 + a_1 \sqrt{2}+ a_2 2 + a_3 2 \sqrt{2} + a_4 4 + a_5 4 \sqrt{2}     +  \cdots + a_{n-1}2^{\frac{n-1}{2}}                              
     &\mathrel{\#} \text{simplify} \\
&=& (a_0 + a_2 2 + a_4 4 + \cdots) + (a_1 + a_3 2 + a_5 4  + \cdots) \sqrt{2}                                    
     &\mathrel{\#} \text{group terms} \\
&=& a + b \sqrt{2}                                                                                                                                                
     &\mathrel{\#} a +b \sqrt{2}  \in \mathbb{Q} [\sqrt{2}]\\
\end{array}
\end{equation*}

\bigskip
\noindent
Note that here $a =  a_0 + a_2 2 + a_4 4 + \cdots$ and $b = a_1 + a_3 2 + a_5 4  + \cdots$ and 
 that $a,b \in \mathbb{Q}$ since $\mathbb{Q}$ is closed under addition and multiplication.

\bigskip
\noindent
Next we need to see what $\mathbb{Q}[\sqrt{2}][x]$ looks like. We saw above that the elements of $\mathbb{Q}[\sqrt{2}]$ have the 
form $a +b \sqrt{2}$ for $a,b \in \mathbb{Q}$. So an element $p(x) \in \mathbb{Q}[\sqrt{2}][x]$ looks like (Definition \ref{def:polynomials})


\begin{equation*}
\begin{array}{rlll} 
p(x)
&=& \sum\limits_{i =  0}^{n-1} (a_i+ b_i \sqrt{2}) x^i                                                           \\
&=& (a_0 +b_0 \sqrt{2})x^0 + (a_1 +b_1 \sqrt{2})x^1 + (a_2 +b_2 \sqrt{2})x^2 + \cdots  + (a_{n-1} + b_{n-1}\sqrt{2}) x^{n-1}
\end{array}
\end{equation*}

\bigskip
\noindent
for some $a_i, b_i \in \mathbb{Q}$.

\bigskip
\noindent
Now, if we consider the case in which $a_0 = 0, b_0 = 1, a_1 = 1, b_1 = 0$ and 
$a_i = b_i = 0$ for $1 < i \leq n - 1$ we get an element $p(x) \in \mathbb{Q}[\sqrt{2}][x]$ 
that looks like

\begin{equation*}
\begin{array}{rlll} 
p(x)   
&=& (a_0 + b_0 \sqrt{2}) x^0 + (a_1 + b_1 \sqrt{2}) x^1 + \sum\limits_{i = 2}^{n-1} (a_i+ b_i \sqrt{2}) x^i    \\    
&=& (0 + 1 \sqrt{2}) 1 + (1 + 0 \sqrt{2}) x + \sum\limits_{i = 2}^{n-1} 0                                   \\    
&=&   \sqrt{2} + x                                                                                                                                      \\
&=&  x +  \sqrt{2}   
\end{array}
\end{equation*}

\bigskip
\noindent
so we can see that $x^2 - 2$ splits in $\mathbb{Q}[\sqrt{2}][x]$  since $x^2 -2 = (x - \sqrt{2})(x + \sqrt{2})$ (let $b_0 = -1$ to get the $(x - \sqrt{2})$ factor).

\bigskip
\noindent
So the first criteria of Definition \ref{def:splitting_field} is satisfied, but is there a field $K$ that splits $x^2 - 2$ such that 
$\mathbb{Q} \subsetneq  K \subsetneq \mathbb{Q}[\sqrt{2}]$ (the second criteria in Definition \ref{def:splitting_field})? Well, if we consider 
$\mathbb{Q}[\sqrt{2}][x]$ as a vector space over $\mathbb{Q}[\sqrt{2}]$ we see that it is of order 2 (written $[\mathbb{Q}[\sqrt{2}]: \mathbb{Q}] = 2$), 
so there is no field $K$ such that $\mathbb{Q} \subsetneq  K \subsetneq \mathbb{Q}[\sqrt{2}]$. So the second criteria is true and so  
$\mathbb{Q}[\sqrt{2}]$ is a splitting field for $f(x) = x^2 - 2$.
\end{example}



\begin{example}
$\mathbb{Q}[\sqrt[3]{2}]$ is \emph{not} a splitting field for $x^3 - 2$ over $\mathbb{Q}$. 

\bigskip
\noindent
Why? Well, it is because the polynomial $x^3 - 2$ does not split in $\mathbb{Q}[\sqrt[3]{2}][x]$. But still why? After all
$x^3 - 2$ does have a root at $\sqrt[3]{2}$ in $\mathbb{Q}[\sqrt[3]{2}] [x]$ . However,
if we divide $x^3 - 2$ by $x - \sqrt[3]{2}$ we see that

\begin{equation}
x^3 - 2 = (x - \sqrt[3]{2})(x^2 + \sqrt[3]{2} x + (\sqrt[3]{2})^2)
\label{eqn:x^3-2}
\end{equation}

\bigskip
\noindent
and it turns out that $h(x) = x^2 + \sqrt[3]{2} x + (\sqrt[3]{2})^2$ is \emph{irreducible}\footnote{A polynomial $p(x)$ is
irreducible if no polynomials $g(x)$ and $h(x)$ exist such that $p(x) = g(x) \cdot h(x)$.} in $\mathbb{Q}[\sqrt[3]{2}]$.
This is because the roots of $h(x)$ are complex and but everything in $\mathbb{Q}[\sqrt[3]{2}]$ is real.
\end{example}

\bigskip
\noindent
So what is a splitting field for $x^2 + \sqrt[3]{2} x + (\sqrt[3]{2})^2$ over $\mathbb{Q}$? Well, we know $x^3-2$ splits into the factors
shown in Equation \ref{eqn:x^3-2} in $\mathbb{Q}[\sqrt[3]{2}]$, so one approach would be to adjoin the (complex) roots of $x^2 + \sqrt[3]{2} x + (\sqrt[3]{2})^2$
to $\mathbb{Q}[\sqrt[3]{2}]$. 

\bigskip
\noindent
The idea to "keep adding roots of irreducible factors" is the core idea in the proof that every polynomial has a splitting field. This 
observation leads to the following proposition:

\bigskip
\begin{proposition}
Let $f \in F[x]$ and $E$ be an extension field of $F$. If $E$ contains the roots $\alpha_1, \cdots, \alpha_n$ of $f$ and $f$ splits 
in $F[\alpha_1, \cdots, \alpha_n][x]$ then $F[\alpha_1, \cdots, \alpha_n]$ is a splitting field for $f$ over $F$. 
\end{proposition}

\noindent
\textbf{Proof: } Because $f$ splits in $F[\alpha_1, \hdots, \alpha_n]$  we only need to show that $f$ doesn't split
in a proper subfield containing $F$. Suppose $E$ is such a proper subfield.  Then there is at least one root 
$\alpha_i$ such that $\alpha_i \notin E$. But this would mean that $f$ would not split in $E$ because
if it did then $\alpha_i$  would be a root of one of the linear factors in $E[x]$; this would contradict our assumption 
that $\alpha_i \notin E$. So such an $E$ does not exist.

\bigskip
\noindent
This result guarantees that if you can find all the roots of a polynomial in \emph{some} extension field, then you can construct a splitting field easily. 
This is great for polynomials that are in, say,  $\mathbb{Q}[x]$  because it is often easy to find roots in $\mathbb{C}$. But what about more obscure fields like 
$\mathbb{Z}_7$, where we don't have a good understanding of its extension fields? It is not obvious (at least to me) that polynomials 
over these fields have splitting fields, but luckily it turns out they do.

\bigskip
\noindent
\textbf{Aside: } We saw that every field is an integral domain (Theorem \ref{thm:integral_domain}). Here we 
observe that any finite integral domain (like $\mathbb{Z}_7$) is a field.

\bigskip
\begin{thm}
Every finite integral domain is a field.
\label{thm:finite_id_is_a_field}
\end{thm}

\bigskip
\noindent
\textbf{Proof:}  The proof is based on the fact that since $R$ is an integral domain it has a cancellation law (or equivalently, $R$ has no zero divisors). 
Having a cancellation law means that

\begin{equation}
ab = ac \implies b = c
\label{eqn:cancellation_law}
\end{equation}


\bigskip
\noindent
To see why any finite integral domain $R$ is a field, consider $R = \{r, r^2, r^3, \hdots, r^n\}$  where 
$r^k \neq 0$ for $1 \leq k \leq n$.  Since $R$ is finite we will have $r^k = r^l$ for some $k$ and $l$ such that $k > l$. Then

\bigskip
\begin{equation*}
\begin{array}{rlll} 
r^k 
&=& r^l                                                                                        &    \qquad \mathrel{\#} \text{$R$ is a finite integral domain}        \\
&\Rightarrow& r \cdot r^{k-1} = r \cdot r^{l -1}                             &    \qquad \mathrel{\#} \text{factor out $r$}                                  \\
&\Rightarrow& r^{k-1} = r^{l -1}                                                   &    \qquad\mathrel{\#} \text{use cancellation law (cancel $r$, Equation \ref{eqn:cancellation_law}})    \\
&\Rightarrow& r \cdot r^{k-2} = r \cdot r^{l -2}                             &    \qquad\mathrel{\#} \text{factor out $r$}                                   \\
&\Rightarrow& r^{k-2} = r^{l -2}                                                   &    \qquad\mathrel{\#} \text{use cancellation law (cancel $r$, Equation \ref{eqn:cancellation_law})}  \\
&\vdots                                                                                        && \qquad\mathrel{\#} \text{iterate $l - 1$ times}                          \\    
&\Rightarrow& r^{k - l + 1} = r^1                                                 &     \qquad\mathrel{\#} \text{...}                                                    \\       
&\Rightarrow& r \cdot r^{k - l} = r \cdot r^0                                 &     \qquad\mathrel{\#} \text{factor out $r$}                                   \\    
&\Rightarrow& r^{k-l} = r^{0}                                                       &    \qquad\mathrel{\#} \text{use cancellation law (cancel $r$, Equation \ref{eqn:cancellation_law})}   \\   
&\Rightarrow& r^{k-l} = 1                                                            &     \qquad\mathrel{\#} r^0 = 1
\end{array}
\end{equation*}

\bigskip
\noindent
So $ r^{k-l} = 1 $. If $k-l = 1$ then $r$ is a unit since $r^{k-l} = r^{1} = 1$ so $r^{-1}$ is $\frac{1}{r}$. Otherwise
$k-l > 1$ and $r^{k-l} = 1 \Rightarrow r^{k-l-1} = \frac{1}{r}$.   So $r^{-1} = r^{k-l -1}$ and 
every $r \neq 0 \in R$ has an inverse. Thus every non-zero  $r \in R$ is a unit and so $R$ is a field. $\square$


\section{Note: Gauss and the Gaussian Integers $\boldsymbol{\mathbb{Z}[i]}$}
First, recall that the Gaussian Integers  $\mathbb{Z}[i] = \{a + bi \mid  a,b \in \mathbb{Z} \text{ and } i = \sqrt{-1} \}$. 
Gauss found that the polynomial $a^2 + b^2$ had a unique factorization (would "split") in $\mathbb{Z}[i]$: 


\begin{equation*}
\begin{array}{rlll} 
a^2 + b^2 
&=&  (a - bi)(a + bi)
\end{array}
\end{equation*}

\bigskip
\noindent
The natural question was are there other values that could be adjoined to $\mathbb{Z}$ to form a new number 
system in which some polynomial would split. For example

\begin{equation*}
\begin{array}{rlll} 
\mathbb{Z}[\sqrt{-5}]
&=& \{a+b\sqrt{-5}  \mid a,b \in \mathbb{Z}\}
\end{array}
\end{equation*}

\bigskip
\noindent 
Here we can factor say $6$ in $\mathbb{Z}[\sqrt{-5}]$ as $6 = 2 \cdot 3 = (1 - \sqrt{-5}) \cdot (1 + \sqrt{-5})$. So the natural question
is there other values in which we can factor polynomials into irreducible factors? It turns out there 
is are precisely nine such numbers, $\{1,2,3,7,11,19,43,67,163\}$ (Gauss discovered this sequence but couldn't prove that these were 
the only such numbers). That is, only the negative square root of these numbers can be adjoined to  $\mathbb{Z}$ to get a ring with 
unique factorization. This is the set

\begin{equation*}
\begin{array}{rlll} 
\{\sqrt{-1}, \sqrt{-2}, \sqrt{-3},  \sqrt{-7}, \sqrt{-11}, \sqrt{-19}, \sqrt{-43}, \sqrt{-67}, \sqrt{-163}\}
\end{array}
\end{equation*}

\bigskip
\noindent
Interestingly,  a Heegner number (so named for the amateur mathematician that proved Gauss's conjecture) is a square-free positive integer $d$ 
such that the imaginary quadratic field $\mathbb {Q} [\sqrt {-d}]$  has unique factorization.

\bigskip
\noindent
These numbers turn up in all kinds of interesting places, including Ramanujan's constant $e^{{\pi {\sqrt {163}}}}$.  For example


\begin{center}
\begin{equation*}
\begin{array}{llll} 
e^{{\pi {\sqrt  {19}}}}   &\approx  96^{3} + 744 - 0.22 \\
e^{{\pi {\sqrt  {43}}}}   &\approx 960^{3}+744-0.000\,22\\
e^{{\pi {\sqrt  {67}}}}   &\approx 5\,280^{3}+744-0.000\,0013\\
e^{{\pi {\sqrt  {163}}}} &\approx 640\,320^{3}+744-0.000\,000\,000\,000\,75
\end{array}
\end{equation*}
\end{center}

\noindent
or alternatively

\begin{equation*}
\begin{array}{llll} 
e^{{\pi {\sqrt  {19}}}}   &\approx 12^{3}(3^{2}-1)^{3}+744-0.22    \\
e^{{\pi {\sqrt  {43}}}}   &\approx 12^{3}(9^{2}-1)^{3}+744-0.000\,22\\
e^{{\pi {\sqrt  {67}}}}   &\approx 12^{3}(21^{2}-1)^{3}+744-0.000\,0013\\
e^{{\pi {\sqrt  {163}}}} &\approx 12^{3}(231^{2}-1)^{3}+744-0.000\,000\,000\,000\,75
\end{array}
\end{equation*}


\bigskip
\begin{thm}
If $m$ is an integer then either $m^2 \equiv 0 \; (\mymod 4)$ or $m^2 \equiv 1 \; (\mymod 4)$.
\end{thm}

\bigskip
\noindent \textbf{Proof:} Let $m \in \mathbb{Z}$. Then $m$ is either even or $m$ is odd.

\begin{equation*}
\begin{array}{llll} 
\textbf{Case I:}    
& \text{Assume $m$ is even.} \\
& \text{If $m$ is even then there exists $k \in \mathbb{Z}$ such that $m = 2k$.} \\
& \text{Then $m^2 = 4k^2$, and so $4 \lvert m^2$ and hence $m^2 \equiv 0 \; (\mymod 4)$.} \\ \\
\textbf{Case II:}    
& \text{Assume $m$ is odd.} \\
& \text{If $m$ is odd then there exists $k \in \mathbb{Z}$ such that $m = 2k + 1$.} \\
& \text{Then $m^2 = 4k^2 +4k + 1 \Rightarrow m^2 - 1 = 4(k^2 + k)$ so}                       \\
& \text{$4 \lvert (m^2 - 1)$.  Therefore $(m^2 - 1)  \equiv 0 \; (\mymod 4)$ and}      \\
&m^2 \equiv 1 \; (\mymod 4).
\end{array}
\end{equation*}


\bigskip
\noindent
Thus if $m$ is an integer then either $m^2 \equiv 0 \; (\mymod 4)$ or $m^2 \equiv 1 \; (\mymod 4)$. $\square$

\bigskip
\noindent
Recall that a \emph{unit} in a ring $R$ is an element which has a multiplicative inverse. 

\begin{proposition}
Let $F$ be a field and let $F[x]$ be the polynomial ring over $F$. Then units in $F[x]$ are exactly the nonzero elements of $F$.
\end{proposition}

\noindent 
\textbf{Proof:}  First, observe that the nonzero elements of $F$ are invertible in $F$ since $F$ is a field.  These elements are 
also invertible in $F[x]$ since, as we just saw,  they are invertible in $F$. 

\bigskip
\noindent
Suppose, OTOH  that $f(x) \in F[x] $ is invertible. That is, $f(x)g(x) = 1$ 
for some $g(x) \in F [x]$. Then $\deg f \cdot g = \deg f + \deg g = \deg 1 = 0$,
which requires that both $f$ and $g$ to have degree 0. In particular, $f$ 
must have degree 0. So $f$ is a nonzero constant, i.e. $f$ is an element of $F$. $\square$


\begin{proposition}
Let $R$ be a commutative ring and let $a$ be a unit in $R$. Then $a$ divides $r$  for all $r \in R$.
\end{proposition}

\noindent 
\textbf{Proof:} First assume $1 \in R$ ($R$ is a ring rather than a rng). Then $a$ a unit in $R$ means that there exists $b \in R$ such
that $ab = 1$. Note that $ab \in R$ since $R$ is closed under multiplication. 

\bigskip
\noindent
Now let $r$ be an arbitrary element of $R$. Then 

\begin{equation*}
\begin{array}{rlll} 
r
&=&  1 \cdot r                         &\qquad\qquad\mathrel{\#} \text{$1$ is the multiplicative identity}\\
&=& (ab) \cdot r                      &\qquad\qquad\mathrel{\#} \text{$a$ a unit $\Rightarrow$ $1 = ab$ with $ab \in R$}           \\
&=& a \cdot (br)                      &\qquad\qquad\mathrel{\#} \text{multiplication is associative}       \\
&\Rightarrow& a|r                   &\qquad\qquad\mathrel{\#}\text{$a|r \Rightarrow r = a \cdot m$. Here  $m = br$. $\square$}
\end{array}
\end{equation*}

\bigskip


\begin{proposition}
Let $R$ be a commutative ring and let $a$ and $b$ be units in $R$. Then $ab$ is a unit in $R$.
\end{proposition}

\noindent
\textbf{Proof:}  Let $a, b \in R$ be units. Then  there exists $c, d \in R$ such that $ac = 1$ and
$bd = 1$. To show that $ab$ is a unit in $R$ consider 


\begin{equation*}
\begin{array}{rlll} 
ac
&=&  a(1c)                 &\qquad\qquad\qquad \mathrel{\#} c = 1 c            \\
&=&  a(1)c                 &\qquad\qquad\qquad \mathrel{\#} \text{multiplication is associative} \\
&=&  a(bd)c               &\qquad\qquad\qquad \mathrel{\#} \text{$b$ a unit so $1 = bd$} \\
&=&  abdc                 &\qquad\qquad\qquad \mathrel{\#} \text{multiplication is still associative}       \\
&=&  (ab)(dc)            &\qquad\qquad\qquad \mathrel{\#} \text{multiplication is associative}       \\
&=&  1                      &\qquad\qquad\qquad \mathrel{\#} ac = 1      \\
\end{array}
\end{equation*}

\bigskip
\noindent
So $(ab)(dc) = 1$ which implies that $ab$ is a unit in $R$ with inverse $dc$. $\square$


\newpage
\section{Acknowledgements}

\bibliographystyle{plain}
\bibliography{/Users/dmm/papers/bib/qc}
\end{document} 
