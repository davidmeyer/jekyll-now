\documentclass[11pt, oneside]{article}   	% use "amsart" instead of "article" for AMSLaTeX format


% \usepackage{draftwatermark}
% \SetWatermarkText{Draft}
% \SetWatermarkScale{5}
% \SetWatermarkLightness {0.9} 
% \SetWatermarkColor[rgb]{0.7,0,0}


\usepackage{geometry}                		% See geometry.pdf to learn the layout options. There are lots.
\geometry{letterpaper}                   		% ... or a4paper or a5paper or ... 
%\geometry{landscape}                		% Activate for for rotated page geometryhttps://www.washingtonpost.com/world/europe/amid-impeachment-probe-gordon-sondland-is-overseeing-a-renovation-of-his-residence-that-has-cost-1-million-in-taxpayer-money/2019/10/16/d0eece92-ef86-11e9-bb7e-d2026ee0c199_story.html?tid=sm_tw
%\usepackage[parfill]{parskip}    		% Activate to begin paragraphs with an empty line rather than an indent
\usepackage{graphicx}				% Use pdf, png, jpg, or eps� with pdflatex; use eps in DVI mode
								% TeX will automatically convert eps --> pdf in pdflat						\label{thm:integral_domain}

								% TeX will automatically convert eps --> pdf in pdflatex		
\usepackage{amssymb}
\usepackage{amsmath}
\usepackage{amsthm}
\usepackage{mathrsfs}
\usepackage[hyphens,spaces,obeyspaces]{url}
\usepackage{url}
\usepackage{hyperref}
\usepackage{subcaption}
\usepackage{authblk}
\usepackage{mathtools}
\usepackage{graphicx}
\usepackage[export]{adjustbox}
\usepackage{fixltx2e}
\usepackage{hyperref}
\usepackage{alltt}
\usepackage{color}
\usepackage[utf8]{inputenc}
\usepackage[english]{babel}
\usepackage{float}
\usepackage{bigints}
\usepackage{braket}
\usepackage{siunitx}
\usepackage{mathtools}
\usepackage{xcolor}



\usepackage[hyphenbreaks]{breakurl}

\newtheorem{thm}{Theorem}[section]
% \newtheorem{defn}[thm]{Definition}
\theoremstyle{definition}
\newtheorem{definition}{Definition}[section]
\newtheorem{proposition}{Proposition}[section]
\newtheorem{lemma}{Lemma}[section]
\newtheorem{example}{Example}[section]




\newcommand{\veq}{\mathrel{\rotatebox{90}{$=$}}}
\DeclareMathOperator{\bda}{\Big \downarrow}


\DeclareMathOperator{\E}{\mathbb{E}}
\newcommand{\argmax}{\operatornamewithlimits{argmax}}
\newcommand{\argmin}{\operatornamewithlimits{argmin}}

\title{Merry X-mas!}
\author{David Meyer \\ dmm@\{1-4-5.net,uoregon.edu\}}

\date{Last update: December 25, 2019}							% Activate to display a given date or no date



\begin{document}
\maketitle


\begin{equation*}
\begin{array}{rcll} 
y
&=& \frac{\ln \big (\frac{x}{m} -  sa  \big ) }{r^2}                                        &\qquad \qquad \mathrel{\#} \text{define $y$}                                                                   \\
&\Rightarrow& r^2 y                          = \ln \big (\frac{x}{m} -  sa  \big )    &\qquad \qquad  \mathrel{\#} \text{multiply both sides by $r^2$}                                       \\
&\Rightarrow& e^{r^2y}                     = \frac{x}{m} -  sa                           &\qquad \qquad \mathrel{\#} \text{exponentiate both sides, noting that $e^{\ln(x)} = x$}  \\
&\Rightarrow& m e^{r^2y}                 = x - msa                                        &\qquad \qquad \mathrel{\#} \text{multiply both sides on the left by $m$}                         \\
&\Rightarrow& m e^{r^2y}                 = x - mas                                        &\qquad \qquad \mathrel{\#} \text{$sa = as$ (assume multiplication is commutative)}      \\
&\Rightarrow& \bf{\color{red}{m e^{rry}  = x - mas}}                                &\qquad \qquad \mathrel{\#} \text{$r^2 = rr \rightarrow$ \bf{Merry X-mas!}}
\end{array}
\end{equation*}


\end{document} 
