\documentclass[11pt, oneside]{article}   	% use "amsart" instead of "article" for AMSLaTeX format


% \usepackage{draftwatermark}
% \SetWatermarkText{Draft}
% \SetWatermarkScale{5}
% \SetWatermarkLightness {0.9} 
% \SetWatermarkColor[rgb]{0.7,0,0}


\usepackage{geometry}                		% See geometry.pdf to learn the layout options. There are lots.
\geometry{letterpaper}                   		% ... or a4paper or a5paper or ... 
%\geometry{landscape}                		% Activate for for rotated page geometryhttps://www.washingtonpost.com/world/europe/amid-impeachment-probe-gordon-sondland-is-overseeing-a-renovation-of-his-residence-that-has-cost-1-million-in-taxpayer-money/2019/10/16/d0eece92-ef86-11e9-bb7e-d2026ee0c199_story.html?tid=sm_tw
%\usepackage[parfill]{parskip}    		% Activate to begin paragraphs with an empty line rather than an indent
\usepackage{graphicx}				% Use pdf, png, jpg, or eps� with pdflatex; use eps in DVI mode
								% TeX will automatically convert eps --> pdf in pdflat						\label{thm:integral_domain}

								% TeX will automatically convert eps --> pdf in pdflatex		
\usepackage{amssymb}
\usepackage{amsmath}
\usepackage{amsthm}
\usepackage{mathrsfs}
\usepackage[hyphens,spaces,obeyspaces]{url}
\usepackage{url}
\usepackage{hyperref}
\usepackage{subcaption}
\usepackage{authblk}
\usepackage{mathtools}
\usepackage{graphicx}
\usepackage[export]{adjustbox}
\usepackage{fixltx2e}
\usepackage{hyperref}
\usepackage{alltt}
\usepackage{color}
\usepackage[utf8]{inputenc}
\usepackage[english]{babel}
\usepackage{float}
\usepackage{bigints}
\usepackage{braket}
\usepackage{siunitx}
\usepackage{mathtools}



\usepackage[hyphenbreaks]{breakurl}

\newtheorem{thm}{Theorem}[section]
% \newtheorem{defn}[thm]{Definition}
\theoremstyle{definition}
\newtheorem{definition}{Definition}[section]
\newtheorem{proposition}{Proposition}[section]
\newtheorem{lemma}{Lemma}[section]
\newtheorem{example}{Example}[section]




\newcommand{\veq}{\mathrel{\rotatebox{90}{$=$}}}
\DeclareMathOperator{\bda}{\Big \downarrow}


\DeclareMathOperator{\E}{\mathbb{E}}
\newcommand{\argmax}{\operatornamewithlimits{argmax}}
\newcommand{\argmin}{\operatornamewithlimits{argmin}}



\begin{document}

\noindent
Let $P$ represent the President, and let $A$ be the statement "Is above the law". Let $E$ be the statement "There are Elections" and $L$ be the statement "There are laws".
Then

\begin{equation*}
\begin{array}{rcll}
P
&\Rightarrow& A                                              &\qquad \mathrel{\#} \text{If you are president ($P$) then you are above the law ($A$)}   \\
A &\Rightarrow& \neg L                                   &\qquad \mathrel{\#} \text{If you are above the law ($A$) then laws don't apply ($\neg L)$}  \\
P &\Rightarrow& \neg L                                   &\qquad \mathrel{\#} A \Rightarrow B \text{ and } B \Rightarrow C \text{ then } A \Rightarrow C \text { (transitivity of $\Rightarrow$)}  \\
E &\Rightarrow& L                                           &\qquad \mathrel{\#} \text{If there are elections ($E$) then there is a law creating that election ($L$)} \\
&\Rightarrow& \neg E \lor L                             &\qquad \mathrel{\#} \text{Definition of $\Rightarrow$}  \\
&\Rightarrow&  L \lor \neg E                            &\qquad \mathrel{\#} \text{$\lor$ is commuatative}  \\
&\Rightarrow&  \neg L \Rightarrow \neg E      &\qquad \mathrel{\#} \text{Definition of $\Rightarrow$}  
\end{array}
\end{equation*}


\bigskip
\noindent
So we have $\mathbf{(P \Rightarrow \neg L) \land ( \neg L \Rightarrow \neg E)}$  and so  $\mathbf{(P \Rightarrow \neg E)}$  by the transitivity of $\Rightarrow$.
That is,

\bigskip
\begin{center}
 {\bf \large If you are President ($P$) then Elections ($E$) don't apply to you}.
 \end{center}

\end{document} 
