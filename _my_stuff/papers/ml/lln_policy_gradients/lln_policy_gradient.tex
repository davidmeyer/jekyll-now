\documentclass[11pt, oneside]{article}   	% use "amsart" instead of "article" for AMSLaTex format


%\usepackage{draftwatermark}
% \SetWatermarkText{Confidential}
% \SetWatermarkScale{5}
% \SetWatermarkLightness {0.85} 
% \SetWatermarkColor[rgb]{0.7,0,0}


\usepackage{geometry}                		% See geometry.pdf to learn the layout options. There are lots.
\geometry{letterpaper}                   		% ... or a4paper or a5paper or ... 
%\geometry{landscape}                		% Activate for for rotated page geometryA. G. Barto, R. S. Sutton, and C. W. Anderson
%\usepackage[parfill]{parskip}    		% Activate to begin paragraphs with an empty line rather than an indent
\usepackage{graphicx}				% Use pdf, png, jpg, or eps� with pdflatex; use eps in DVI mode
								% Tex will automatically convert eps --> pdf in pdflatex		
\usepackage{amssymb}
\usepackage{mathrsfs}
\usepackage{hyperref}
\usepackage{url}
\usepackage{authblk}
\usepackage{amsmath}
\usepackage{mathtools}
\usepackage{graphicx}
\usepackage{fixltx2e}
\usepackage{hyperref}
\usepackage{alltt}
\usepackage{color}

\newcommand{\argmax}{\operatornamewithlimits{argmax}}
\newcommand{\argmin}{\operatornamewithlimits{argmin}}

\DeclareMathOperator{\E}{\mathbb{E}}
\newcommand{\Var}{\mathrm{Var}}
\newcommand{\Cov}{\mathrm{Cov}}



\title{The Law of Large Numbers and Policy Gradients}
\author{David Meyer \\ dmm@\{1-4-5.net,uoregon.edu,...\}}

\date{Last update: \today}							% Activate to display a given date or no date


\begin{document}
\maketitle

\section{Introduction} 

The \emph{Strong Law of Large Numbers} (LLN) is usually stated as follows: 

\bigskip
\noindent
Let $x_{1}, x_{2}, \hdots, x_{M}$ be a sequence of independent and identically distributed (i.i.d) random variables, each having a finite mean $\mu_i = E[x_{i}]$. 

\bigskip
\noindent
Then with probability one
\begin{equation}
\frac{1}{M}\sum\limits_{i=1}^{M} x_i \rightarrow E[x]
\end{equation}
as  $M \rightarrow \infty$.

\bigskip
\noindent
A complementary theorem, Ergodic Theorem, is stated as follows:
\bigskip
\noindent
Let $\theta^{(1)}, \theta^{(2)}, \hdots, \theta^{(M)}$ be $M$ samples from a Markov chain that is \emph{aperiodic}, \emph{irreducible}, and \emph{positive recurrent}\footnote{In this case, the chain is said to be \emph{ergodic}.}, and $E[g(\theta)] < \infty$.

\bigskip
\noindent
Then with probability one
\begin{equation}
\frac{1}{M}\sum\limits_{i = 1}^{M} g(\theta_{i}) \rightarrow E[g(\theta)]  = \int_{\Theta}^{}g(\theta) \: \pi(\theta) \:d\theta
\end{equation}
as $M \rightarrow \infty$ and where $\pi$ is the stationary distribution of the Markov chain.

\section{The LLN and Likelihood Ratio Policy Gradients} 
Suppose that $r(x)$ is a performance measure that depends on some random variable $X$, and 
$q(x; \theta)$ is the is the probability that $X = x$, 
parameterized by $\theta \in \mathbb{R}^K$. Under mild regularity conditions, the gradient with respect to $\theta$ 
of the expected performance $\eta(\theta)$ can be seen to be the following:

\begin{flalign}
\eta(\theta) &= \E_{x \sim q(xl \theta)} [r(x)] 
 \: \quad \qquad \qquad \qquad \qquad \qquad \mathrel{\#} \text{definition of }  \eta(\theta) \\
&= \sum\limits_{x} r(x) \cdot q(x; \theta) 
 \: \; \qquad \qquad \qquad \qquad \qquad \mathrel{\#} \text{definition of expectation}  \\
\nabla \eta(\theta)  &= \sum_{x} r(x) \nabla_{\theta} q(x; \theta) 
\qquad \qquad \qquad \qquad \qquad \mathrel{\#} \text{take the derivative of both sides}  \\
&= \sum_{x} r(x) \frac{\nabla_{\theta} q(x; \theta)}{q(x; \theta)} q(x; \theta) 
\quad \qquad \qquad \qquad \mathrel{\#} \text{multiply by } 1 = \frac{q(x; \theta)} {q(x; \theta)} \\
&= \E_{x \sim q(x;\theta)} r(x)  \frac{\nabla_{\theta} q(x; \theta)}{q(x; \theta)}
 \; \quad \qquad \qquad \qquad \mathrel{\#} \text{definition of expectation} 
\end{flalign}

\bigskip
\noindent
So our gradient 
$\nabla_{\theta} \eta(\theta) = \E_{x \sim q(x;\theta)} r(x)  \frac{\nabla_{\theta} q(x; \theta)}{q(x; \theta)}$, which 
means we can estimate the expectation (gradient) with 

\begin{equation*}
\hat{\eta}(\theta) = \frac{1}{N} \sum\limits_{i = 1}^N  r(x)  \frac{\nabla_{\theta} q(x; \theta)}{q(x; \theta)}
\end{equation*}

\bigskip
\noindent
Now, given the law of large numbers we know 

\begin{flalign*}
\hat{\eta}(\theta) \rightarrow \eta(\theta) \text{ with probability one}
\end{flalign*}

\bigskip
\noindent
This means our gradient estimator ($\hat{\eta}(\theta)$) is \emph{unbiased} since its expected value 
equals the true gradient. Specifically:

\begin{flalign}
\E[\hat{\eta}(\theta)] &= \nabla \eta(\theta) 
\end{flalign}



\newpage
\bibliographystyle{ieeetr}
\bibliography{/Users/dmm/papers/bib/ml}



\end{document} 
