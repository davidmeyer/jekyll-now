\documentclass[11pt, oneside]{article}   	% use "amsart" instead of "article" for AMSLaTeX format


% \usepackage{draftwatermark}
% \SetWatermarkText{Draft}
 % \SetWatermarkScale{5}
% \SetWatermarkLightness {0.85} 
% \SetWatermarkColor[rgb]{0.7,0,0}


\usepackage{geometry}                		% See geometry.pdf to learn the layout options. There are lots.
\geometry{letterpaper}                   		% ... or a4paper or a5paper or ... 
%\geometry{landscape}                		% Activate for for rotated page geometry
%\usepackage[parfill]{parskip}    		% Activate to begin paragraphs with an empty line rather than an indent
\usepackage{graphicx}				% Use pdf, png, jpg, or eps� with pdflatex; use eps in DVI mode
								% TeX will automatically convert eps --> pdf in pdflatex		
\usepackage{amssymb}
% \usepackage{mathrsfs}
% \usepackage{hyperref}
% \usepackage{url}
% \usepackage{authblk}
% \usepackage{amsmath}
% \usepackage{graphicx}
% \usepackage{fixltx2e}
% \usepackage{hyperref}
% \usepackage{alltt}
\usepackage{color}
\usepackage{bigints}

\newcommand{\argmax}{\operatornamewithlimits{argmax}}
\newcommand{\argmin}{\operatornamewithlimits{argmin}}

\title{Notes on policy gradients and the log derivative trick for reinforcement learning}
\author{David Meyer \\ dmm@\{1-4-5.net,uoregon.edu\}}
\date{Last update: \today}							% Activate to display a given date or no date



\begin{document}
\maketitle

\section{Introduction}
\noindent
The \emph{log derivative trick}\footnote{http://blog.shakirm.com/2015/11/machine-learning-trick-of-the-day-5-log-derivative-trick} is a widely used identity that allows us to find various gradients required for policy learning.  For policy-based reinforcement learning, we directly parameterize the policy. In value-based learning, we imagine we have value function approximator (either state-value or action-value) parameterized by $\theta$:

\begin{flalign}
V_\theta(s) & \approx V^\pi(s) \qquad \qquad \qquad  \; \; \;   \mathbin{\#} \text{state-value approximation}\\
Q_\theta(s,a) & \approx Q^\pi(s,a) \qquad \qquad  \; \;  \; \; \; \; \; \mathbin{\#} \text{action-value approximation}
\end{flalign}

\bigskip
\noindent
Here our goal is to directly parameterize the policy (i.e., \emph{model-free reinforcement learning}):
\begin{flalign}
\pi_\theta(s,a) & = \mathbb{P}[a | s, \theta] \qquad \qquad   \mathbin{\#} \text{parameterized policy}
\end{flalign}

\bigskip
\noindent
\section{Policy Objective Functions}

There are three basic policy objective functions, each of which has the goal of given a policy $\pi_\theta(s,a)$ with paramters $\theta$,  find the best $\theta$.
\begin{itemize}
\item In episodic environments we can use the \emph{start value}: $J_1(\theta) = V^{\pi_\theta}(s_1) = \mathbb{E}_{\pi_\theta}[v_1]$
\item In continuing environments we can use the \emph{average value}: $J_{aaV}(\theta) = \sum \limits_s^{} d^{\pi_\theta}(s)  V^{\pi_\theta}(s)$
\item Or we can use the \emph{average reward per time step}: $J_{avR}(\theta) =  \sum \limits_s^{} d^{\pi_\theta}(s)  \sum \limits_a^{} \pi_\theta(s,a) \mathcal{R}_s^a $
\end{itemize}

\bigskip
\noindent
where $d^{\pi_\theta}(s)$ is the \emph{stationary distribution} of a Markov chain for $\pi_\theta$.

\bigskip
\noindent
So now we're casting policy based reinforcement learning as an optimization problem (e.g., there is a neural network that we want to learn the policy, e.g., via gradient ascent). Now, let $J(\theta)$ be a \emph{policy objective function}. A policy gradient algorithm searches for a local maximum\footnote{in the case that $J(\theta)$ is \emph{non-convex}.} in $J(\theta)$ by ascending the gradient of the policy with respect to the parameters $\theta$. The update rule for $\theta$ is 

\begin{flalign}
\theta \leftarrow \theta + \alpha \nabla_{\theta} J(\theta)  
\end{flalign}

\bigskip
\noindent
where $\nabla_{\theta} J(\theta)$ is the \emph{policy gradient} and $\alpha$ is the \emph{learning rate} (sometimes step-size parameter). In particular

\begin{flalign}
\nabla_{\theta}J(\theta)           & = \begin{bmatrix}  \frac{ \partial J(\theta)}{\partial \theta_{1}}   \\  
\frac{ \partial J(\theta)}{\partial \theta_{2}}      \\ 
\vdots \\ 
 \frac{ \partial J(\theta)}{\partial \theta_{n}}         
 \end{bmatrix} 
\end{flalign}

% \newpage

\section{Computing the gradient analytically}

First, we assume that the policy $\pi_\theta$ is differentiable wherever it is non-zero (this is a softer requirement than requiring $\pi_\theta$ be differentiable \emph{everywhere}). In addition, we know the gradient: $\nabla_\theta J(\theta)$. In this case, let  $p(\mathbf{x};\theta)$ be the likelihood parametrized by $\theta$ and let $\log p(\mathbf{x}; \theta)$ be the \emph{log likelihood}.  Define

\begin{equation*}
\begin{array}{lllll}
y 
&=&  p(\mathbf{x}; \theta)                                                                                        & \qquad \qquad \mathrel{\#} \text{define $y$ to be the likelihood parametrized by $\theta$} \\  
\vspace{0.80mm}
z 
&=&                  \log y = \log p(\mathbf{x}; \theta)                                                    & \qquad \qquad \mathrel{\#}  \text{$z$ is the  log likelihood}\\  
\end{array}
\end{equation*}

\bigskip
\noindent
Then 

\begin{equation*}
\begin{array}{lllll}
\vspace{0.80mm}
\dfrac{dz}{d \theta} 
&=&                   \dfrac{dz}{dy} \cdot \dfrac{dy}{d\theta}                                               & \qquad \qquad \mathrel{\#}  \text{chain rule} \\
\vspace{0.80mm}
\dfrac{dz}{dy} 
&=&                     \dfrac{1}{p(\mathbf{x};\theta)}                                                        & \qquad \qquad \mathbin{\#} \dfrac{d}{dx} \log x \approx \frac{1}{x} \\
\vspace{0.80mm}
 \dfrac{dy}{d\theta} 
 &=& \dfrac{d}{d\theta}  \:p(\mathbf{x};\theta) = \nabla_\theta p(\mathbf{x};\theta)   & \qquad \qquad  \mathbin{\#} \text{ definition (chain rule, again)}\\
 \vspace{0.80mm}
 \dfrac{dz}{d \theta} 
 &=&          \dfrac{dz}{dy} \cdot \dfrac{dy}{d\theta} = \dfrac{\nabla_\theta \; p(\mathbf{x};\theta)}{p(\mathbf{x};\theta)} & \qquad \qquad  \mathbin{\#} \text{ apply chain rule} \\
 \vspace{0.80mm}
 \dfrac{dz}{d \theta} 
&=&  \nabla_\theta \log p(\mathbf{x};\theta)                                           & \qquad \qquad\mathbin{\#} \text{ set $w = p(\mathbf{x};\theta)$ and note that } \frac{1}{w}\nabla_\theta w = \nabla_{\theta} \log w
\end{array}
\end{equation*}

\bigskip
\noindent
Here $\nabla_\theta \log p(\mathbf{x};\theta)$ is known as the score or sometimes the \emph{Fischer} information. So the \emph{log derivative trick} (sometimes \emph{likelihood ratio}) is 
\begin{align*}
\nabla_\theta \log p(\mathbf{x};\theta)   = \frac{\nabla_\theta \; p(\mathbf{x};\theta)}{p(\mathbf{x};\theta)}
\end{align*}

\bigskip
\noindent
Setting  $\pi_\theta(s,a) = p(\mathbf{x};\theta)$ we see that


\begin{equation*}
\begin{array}{lllll}
\nabla_\theta \pi_\theta(s,a) 
&=&  \nabla_\theta \pi_\theta(s,a) \cdot \Big [\frac{\pi_\theta(s,a)}{\pi_\theta(s,a)} \Big]   & \qquad \qquad\mathbin{\#} \text{ multiply by $1 = \frac{\pi_\theta(s,a)}{\pi_\theta(s,a)} $}   \\
&=& \Big [\frac{\nabla_\theta \pi_\theta(s,a)}{\pi_\theta(s,a)} \Big ] \cdot \pi_\theta(s,a)   & \qquad \qquad\mathbin{\#} \text{ multiplication is associative}                                            \\
&=& \nabla_\theta \log \pi_\theta(s,a) \cdot \pi_\theta(s,a)                                               & \qquad \qquad\mathbin{\#} \text{ log derivative trick: } \frac{\nabla_\theta \pi_\theta(s,a)}{\pi_\theta(s,a)} = \nabla_\theta \log \pi_\theta(s,a) 
\end{array}
\end{equation*}

\bigskip
\noindent
and the score function is $\nabla_\theta \log \pi_\theta(s,a)$.
 

\bigskip
\noindent 
Another yet similar way to look policy gradients is as follows:\footnote{h/t Andrej  Karpathy http://karpathy.github.io/2016/05/31/rl/}
First, as Andrej points out policy gradients are a special case of a more general \emph{score function gradient estimator}. Here we have an expectation of the form $\mathrm{E}_{x \sim{p(x|\theta)}} \big[f(x)\big ]$. This is the expectation of a scalar valued function $f(x)$ under some probability distribution $p(x|\theta)$. Here $f(x)$ can be thought of as a \emph{reward function} and $p(x|\theta)$ is the \emph{policy} network. 

\bigskip
\noindent
The problem we want to solve is how we should shift the distribution (though its parameters $\theta$) to increase its score as judged by $f$. Since the general gradient decent update rule is 

\begin{flalign}
\theta \leftarrow \theta + \alpha \nabla_{\theta} J(\theta) 
\label{eqn:grad_update}
\end{flalign}

\noindent
our goal is to find $\nabla_{\theta}E_{x \sim p(x|\theta)}\big[f(x)\big]$ and update $\theta$ in the direction indicated by the gradient. However, what we know is $f$ and $p$. 

\begin{flalign*}
\nabla_{\theta}E_{x \sim p(x|\theta)} \big [f(x)\big] &= \nabla_{\theta} \sum \limits_{x}p(x) f(x) \qquad \qquad    \qquad  \mathbin{\#} \text{defn expectation} \\
& = \sum \limits_{x} \nabla_{\theta} p(x) f(x) \; \qquad \qquad  \qquad  \mathbin{\#} \text{swap sum and gradient}  \\
&= \sum \limits_{x}  p(x) \frac{\nabla_{\theta} p(x)}{p(x)} f(x) \; \qquad \qquad  \mathbin{\#} \text{multiply/divide by } p(x) \\
&=  \sum \limits_{x}  p(x) \nabla_{\theta} \log p(x) f(x) \qquad \quad  \mathbin{\#}  \frac{1}{z} \nabla_{\theta} z = \nabla_{\theta} \log (z)  \\
&= E_{x \sim p(x|\theta)} \big [ f(x) \nabla_{\theta} \log p(x) \big ] \quad \quad \mathbin{\#} \text{defn expectation again} 
\end{flalign*}

\bigskip
\noindent
The basic idea here is that we have some distribution $p(x|\theta)$ which we can sample from,\footnote{for example,  $p(x|\theta)$ could be a Gaussian} and for each sample we evaluate its score $f(x)$; then the gradient 

\begin{flalign}
\nabla_{\theta}E_{x \sim p(x|\theta)} \big [f(x)\big]  = E_{x \sim p(x|\theta)} \big [ f(x) \nabla_{\theta} \log p(x) \big ]
\end{flalign}

\bigskip
\noindent
is telling us how we should shift the distribution (through its parameters $\theta$) if we wanted its samples to achieve higher scores (as judged by $f$). The second term $\nabla_{\theta} \log p(x)$ is telling us which direction in parameter space would lead to increase of the probability assigned to a given $x$.  That is,  if we were to move $\theta$  in the direction $\nabla_{\theta} \log p(x)$ we would see the new probability assigned to some $x$ slightly increase (see Equation \ref{eqn:grad_update}).

\section{The Policy Gradient Theorem}
Recall that the the start value policy for episodic environments was

\begin{flalign}
J_1(\theta) = V^{\pi_\theta}(s_1) =  \mathbb{E}_{\pi_\theta}[v_1] = \sum \limits_{s \in \mathcal{S}} d(s) \sum \limits_{a \in \mathcal{A}} \pi_{\theta}(s,a) \mathcal{R}_{s}^{a}
\end{flalign}

\bigskip
\noindent
and thus
\begin{flalign}
\nabla_\theta J(\theta) & = \sum \limits_{s \in \mathcal{S}} d(s) \sum \limits_{a \in \mathcal{A}} \pi_{\theta}(s,a) \nabla_\theta \log \pi_\theta(s,a) \mathcal{R}_{s}^{a}  \\
&= \mathbb{E}_{\pi_\theta} [\nabla_\theta \log \pi_\theta(s,a) r]
\end{flalign}

\bigskip
\noindent
A few things to notice:
\begin{itemize}
\item The policy gradient theorem generalizes the likelihood ratio
\item In the policy gradient theorem we replace $r$ with the long-term value $Q^{\pi}(s,a)$ (our estimate of $r$).
\item The policy gradient theorem applies to the start-state, average reward and average value objectives
\end{itemize}

\bigskip
\noindent
For any differentiable policy $\pi_\theta(s,a)$ and for any policy objective $J_1$, $J_{avR}$ or $\frac{1}{1-\gamma}J_{avV}$, the \textbf{policy gradient} is:
\begin{flalign}
\nabla_\theta J(\theta) & =  \mathbb{E}_{\pi_\theta} \Big[\nabla_\theta \log \pi_\theta(s,a)  Q^{\pi_\theta}(s,a)\Big]
\end{flalign}

\bigskip
\noindent
Now that we have $\nabla_\theta J(\theta)$, we can use this gradient to train a neural network (e.g., the policy networks of AlphaGo).
\end{document} 

