\documentclass[11pt, oneside]{article}   	% use "amsart" instead of "article" for AMSLaTeX format


%\usepackage{draftwatermark}
% \SetWatermarkText{Confidential}
% \SetWatermarkScale{5}
% \SetWatermarkLightness {0.85} 
% \SetWatermarkColor[rgb]{0.7,0,0}


\usepackage{geometry}                		% See geometry.pdf to learn the layout options. There are lots.
\geometry{letterpaper}                   		% ... or a4paper or a5paper or ... 
%\geometry{landscape}                		% Activate for for rotated page geometryA. G. Barto, R. S. Sutton, and C. W. Anderson
%\usepackage[parfill]{parskip}    		% Activate to begin paragraphs with an empty line rather than an indent
\usepackage{graphicx}				% Use pdf, png, jpg, or eps� with pdflatex; use eps in DVI mode
								% TeX will automatically convert eps --> pdf in pdflatex		
\usepackage{amssymb}
\usepackage{mathrsfs}
\usepackage{hyperref}
\usepackage{url}
\usepackage{authblk}
\usepackage{amsmath}
\usepackage{mathtools}
\usepackage{graphicx}
\usepackage{fixltx2e}
\usepackage{hyperref}
\usepackage{alltt}
\usepackage{color}
\usepackage{stix}
\usepackage{caption}


\newcommand{\argmax}{\operatornamewithlimits{argmax}}
\newcommand{\argmin}{\operatornamewithlimits{argmin}}
\newcommand\independent{\protect\mathpalette{\protect\independenT}{\perp}}
\def\independenT#1#2{\mathrel{\rlap{$#1#2$}\mkern2mu{#1#2}}}
\DeclareMathOperator{\E}{\mathbb{E}}
\newcommand{\Var}{\mathrm{Var}}
\newcommand{\Cov}{\mathrm{Cov}}



\title{Some Notes on Kolmogorov Complexity, \\  the Algorithmic Markov Condition, and Causality}
\author{David Meyer \\ dmm@\{1-4-5.net,uoregon.edu,...\}}

\date{Last update: \today}							% Activate to display a given date or no date


\begin{document}
\maketitle


\section{Introduction} 
\label{sec:intro}
The Kolmogorov complexity of a string $s$, $K(x)$,  is generally be defined as a function from finite binary strings of arbitrary length to the natural numbers 
$\mathbb{N}$ \cite{Janzing:2010:CIU:1870245.1870271}.
That is,  $K: \{0,1\}^* \rightarrow  \mathbb{N}$  is a function defined on \emph{objects} represented by binary strings. As we shall see, this definition is can be extended to 
other types of objects such as numbers, sets, functions and probability distributions.

\bigskip
\noindent
As a first approximation, $K(x)$ may be thought of as the length of the shortest computer program that prints $x$ and then halts. This computer program may be written in C, Java, Python or any other universal programming language. These are programming languages in which a universal Turing Machine can be implemented. Many if not most  programming languages have this property. Here we will just fix some universal language (say, python) and define Kolmogorov complexity with respect to it. The invariance theorem (Section \ref{sec:invariance_theorem}) tells us that it does not really matter which language we pick. 

\bigskip
\noindent
Note that computer programs often make use of data, and frequently such data are encoded inside the program itself. An example is the bitstring "010110..." in the program

\begin{equation}
\label{eqn:print_string}
\text{print }  \text{"}\underbrace{010101\hdots01}_{\text{n times "01"}} \text{"}
\end{equation}

\bigskip
\noindent
So what is the complexity of the program shown in  \ref{eqn:print_string}? We think of the "interpreter" for this program as a \emph{Universal Machine} (in particular, a Universal Turing Machine or UTM), 
so all programs have a constant (call it $C$) complexity that accounts for the UTM. What is the Kolmogorov complexity of such a program? Well, since a program to print this string and halt, 
call it $p$, might be (in this case, python):


\begin{equation*}
\label{eqn:python_print_string}
\text{print "01" * n} 
\end{equation*}

\bigskip
\noindent
Here we notice that we need only $O(\log n)$ bits to represent the number $n$, the number of 
times the string "01" is repeated.  So for this program $K(p) = O(\log n)$. Other programs that have 
$K(p) = O(\log n)$ include printing out the first $n$ digits of $\pi$ (or $e$). So while the decimal expansion
of $\pi$ might appear to be random, it has low Kolmogorov complexity (in fact,  $O(\log n)$) since the digits of $\pi$ can be 
generated by a simple program where the only quantity that
is a function of $n$ is $n$ itself, and we need $O(\log n)$ bits to represent $n$.

\bigskip
\noindent
A bit of notation: We use $x$ to denote finite bitstrings $x \in \mathbb{B}^*$.  Also, let $l(x)$, the length of a given bitstring $x$, be $n$. 
That is, $n \coloneqq\ l(x)$.  Boldface $\mathbf{x}$ denotes a (possibly infinite) binary string and $x_{[1:n]}$ is the $n$-bit prefix 
of $\mathbf{x}$. 

\section{Properties of $K(x)$} 
\label{sec:properties}


\subsection{Very Simple Programs}
For very simple objects,  $K(x) = O(\log n)$.  Here $K(x)$ must be small for 'simple' or 'regular' objects $x$. For example, there exists a fixed-size program that, 
when input $n$, outputs the first n bits of $\pi$ and then halts (see above). It is easy to see that this specification of $n$ takes $O(\log n)$ bits. Thus, as we saw above,
when $x$ consists of the first $n$ bits of $\pi$, i$K(x) = O(\log n)$. 

\bigskip
\noindent
Note values of $n$, $K(x)$ may be quite a bit smaller than $O(\log n)$. 
For example, suppose $n = 2^m$ for some $m \in \mathbb{N}$, or said differently, $m = \log n$.  Then we can describe $n$ by first describing $m$.
To do so we need to describe a program implementing the function $f(z) = 2^z$.  Note that the description of the program that computes $f(z)$ takes a constant 
number of bits which does not depend on $n$. Now, we know that the description of $m$ takes $O(\log m)$ bits, and we also know that 
$m = \log n$,  so if $n$ has this form we get $K(x) = O(\log m) = O(\log \log n)$.

\subsection{Completely Random Objects}
Here $K(x) = n + O(\log n)$.  Here we can think of a code or description method as a binary relation between source words, namely, strings to be encoded and code words 
(the encoded versions of these strings). Without loss of generality, we can take the set of code words to be finite binary strings \cite{Cover2006}.  In these notes we'll only consider uniquely 
decodable codes where the relation is one-to-one or one-to- many\footnote{We'll need this 1:1 or many:1 property counting arguments.}, indicating that given an encoding 
$E(x)$ of string $x$, we can always reconstruct the original $x$. The Kolmogorov complexity of $x$, $K(x)$, can be viewed as the code length of $x$ that results from using the 
Kolmogorov code $E^*(x)$: this is the code that encodes $x$ by the shortest program that prints $x$ and halts. Another way to think about this: if $K(x) \geq n$ then $x$ is "Kolmogorov random";
we in this case we might also refer to $x$ as "incompressible". See also \cite{2009arXiv0910.1691J}.

\bigskip
\noindent
One of the key insights relating to Kolmogorov codes is as follows: for any uniquely decodable code, there are no more than $2^n$ strings $x$ which can be described by $n$ bits. 
The reason for this is that, as we know, there are no more than $2^n$ binary strings of length $n$. However,  the number of strings that can be described by less than $n$ bits,
that is, $2^0 + 2^1 + 2^2 + \hdots + 2^{n -1} = \sum\limits_{i = 0}^{n -1} 2^i = 2^n - 1$, and $2^n - 1 < 2^n$; thus by the Pigeon Hole Principle there is at least one string of length 
$n$ which can not be mapped to a shorter string\footnote{Such a string has $K(X) \geq |x|$, that is, $x$ is not compressible.}. 

\bigskip
\noindent
We can use the same argument to observe that  the fraction of strings $x$ of length $n$ with $K(x) < n - k$ is less than $2^{-k}$. Why? We know that
$2^0 + 2^1 + 2^2 + \hdots + 2^{{n -1}} < 2^n$.  Now, suppose that the $2^n$ strings are uniformly distributed (that is, all strings have equal probability,
namely, $2^{-n}$).  Then there are $2^n$ strings of length $n$, and $2^{(n-k)}/2^n = 2^{(n-k-n)} = 2^{-k}$. So no more than $2^{-k}$
of the  $2^n$ strings are compressible.  On the other hand, most of the strings of length $n$, notably the fraction $1-2^{-k}$, aren't compressible 
by more than a constant.  Somewhat surprising. Said another way, we know that  $K(x) \geq n - k $ with probability at least $1 - 2^{-k}$.



\bigskip
\noindent
To avoid variability due to encoding, we encode $x$ in a prefix-free manner (see Section \ref{sec:prefix_codes} on prefix codes), which requires 
$n + O(\log n)$ bits, rather than $n + O(1)$. Therefore, for all $x$ of length $n$, $K(x) \leq n+O(\log n)$. Now, as we saw 
above, except for a fraction of $2^{-k}$ of these $K(x) \geq  n - k$, so we see that for the overwhelming majority of $x$, $K(x) = n + O(\log n)$.

\subsection{Invariance}
\label{sec:invariance_theorem}
It would seem that $K(x)$ depends strongly on what programming language we used in our definition of $K$. However, it turns out that for any two
universal languages $L_1$ and $L_2$

\begin{equation}
|K_1(x) - K_2(x)| \leq C
\end{equation}

\bigskip
\noindent
where $K_1$ and $K_2$ denote the respective complexities of $L_1$ and $L_2$ and $C$ is a constant that depends on 
$L_1$ and $L_2$ but not on $x$ or its length. So any two universal machines differ by only a constant.

\subsection{Prefix Codes}
\label{sec:prefix_codes}

Let's start with a bit of notation:  Let $\mathcal{X}$  be some finite or countable set\footnote{Sometimes the symbol $\mathbb{B}$, for "binary" I assume, is used instead of $\mathcal{X}$.}. 
As usual, the notation $\mathcal{X}^*$ (or $\mathbb{B}^*$) is used to denote the set of finite 
strings or sequences over  $\mathcal{X}$ . For example, we may take $\mathcal{X}^* = \{0,1\}^* = \{\epsilon,0,1,00,01,10,11,000, \hdots\}$, where $\epsilon$ represents the 
empty string, that is, the string with no characters. The length $l(x)$ of $x$ is the number of bits in the binary string $x$. For example, $l(010) = 3$ and $l(\epsilon) = 0$. If $x$ is interpreted 
as an integer, we get $l(x) = \lfloor log(x + 1) \rfloor$, and for $x \geq 2$, 

\begin{equation}
\lfloor \log x \rfloor  \leq l(x) \geq \lceil \log x \rceil
\end{equation}

\bigskip
\noindent
where $\lceil x \rceil$ is the smallest integer larger than or equal to $x$ and $\lfloor x \rfloor$ is the largest integer smaller than or equal to $x$, and $\log$ denotes logarithm to base two. 
Kolmogorov theory is  typically be concerned with encoding finite-length binary strings by other finite-length binary strings; binary strings are only used as a convenience since observations in 
any alphabet can be so encoded in binary.

\bigskip
\noindent
One of the problems we face with many codes is that in general,  we cannot uniquely recover $x$ and $y$ from their concatenated encoding $E(xy)$. To see this, let $E$ be the identity 
mapping ($E(X) = x$). Then $E(11)E(11) = 1111 = E(111)E(1)$.  As we will see below, prefix-free codes are one way to address this problem. First, a bit of notation.

\bigskip
\noindent 
A binary string $x$ is a proper prefix of a  binary string $y$ if we can write $y = xz$ for  $z \neq \epsilon$.  
 A set $\{x,y, \hdots\}  \subseteq \{0,1\}^*$ is prefix-free if for any pair of distinct elements in the set neither is a 
proper prefix of the other. A function $D : \{0,1\}^* \rightarrow N$ 
defines a prefix-free code if its domain is prefix-free. Decoding a prefix-free code is relatively simple: start at the beginning and decode one code word at a 
time. When we come to the end of a code word, we know it is the end, since no code word is the prefix of any other code word in a prefix-free code. Clearly, prefix-free codes 
are uniquely decodable: we can always unambiguously reconstruct an outcome from its encoding.  Note that prefix codes are not the only codes with this property; there are uniquely 
decodable codes which are not prefix-free. In the next section, we will define Kolmogorov complexity in terms of prefix-free codes. 

\subsection{Prefix-free Codes for $\mathbb{N}$}
Suppose we encode each binary string $x = x_1x_2 \hdots x_n$ as 
 
\begin{equation}
\label{eqn:pairing}
\bar{x} = \underbrace{1111 \hdots 1}_{\text{n times}}0 x_1x_2 \hdots x_n
\end{equation}

\bigskip
\noindent
We can see that $\bar{x}$ is prefix-free:  we can determine where the code word $\bar{x}$ ends by reading it from left to right without the need to back up. 
$\bar{x}$ has some interesting properties. First,
note $l(\bar{x}) = 2n + 1$; here we have encoded strings in $\mathbb{B}^*$ in a relatively simple prefix-free manner; the cost is doubling the length of the encoding. 
There are, however, approaches that give us a much more efficient code. One such approach is to apply the $\bar{x}$ encoding (Equation \ref{eqn:pairing})
to $l(x)$ rather than to $x$. To do so  define $x^\prime = \overline{l(x)}x$, where $l(x)$ is interpreted as a binary string of length 
$\log l(x)$, where $\log l(x) = \log n$. Then the code that maps $x$ to $x^\prime$ is a prefix-free code satisfying $\forall x \in \{0,1\}^*, \; l(x^\prime) = n + 2 \log n + 1$ (ignoring rounding errors). 
This is because the length of the prefix-free code described  in Equation \ref{eqn:pairing} is $2n + 1$, here $n$ represents the length of $l(x)$, which is $\log n$. So we have the length of 
$\overline{l(x)}  = 2 \log n + 1$  so that $l(x^\prime) = l(\overline{l(x)}x) = 2 \log n + 1 + |x| =  n + 2 \log n + 1$.
This code is frequently called the \emph{standard prefix-free code for the natural numbers} and $L_{\mathbb{N}}(x)$ represents the code length of $x$ under this code. That is,
$L_{\mathbb{N}}(x) = l(x^\prime) = n + 2 \log n + 1$. 


\newpage
\bibliographystyle{ieeetr}
\bibliography{/Users/dmm/papers/bib/ml}

\end{document} 
