
\documentclass[11pt, oneside]{article}   	% use "amsart" instead of "article" for AMSLaTeX format


% \usepackage{draftwatermark}
% \SetWatermarkText{Confidential}
% \SetWatermarkScale{5}
% ��\SetWatermarkColor[rgb]{0.7,0,0}


\usepackage{geometry}                		% See geometry.pdf to learn the layout options. There are lots.
\geometry{letterpaper}                   		% ... or a4paper or a5paper or ... 
%\geometry{landscape}                		% Activate for for rotated page geometry
%\usepackage[parfill]{parskip}    		% Activate to begin paragraphs with an empty line rather than an indent
\usepackage{graphicx}				% Use pdf, png, jpg, or eps� with pdflatex; use eps in DVI mode
								% TeX will automatically convert eps --> pdf in pdflatex		
\usepackage{amssymb}
\usepackage{hyperref}
\usepackage{url}
\usepackage{authblk}
\usepackage{amsmath}
\usepackage{graphicx}
\usepackage{fixltx2e}
\usepackage{hyperref}
\usepackage{alltt}


% \documentclass[11pt]{amsart}

\title{On Virtualization}
\author{David Meyer}
\date{\today}                                           % Activate to display a given date or no date

\begin{document}
\maketitle

\begin{abstract}
A precise definition of \emph{virtualization} has been elusive. Given the advent of virtualization as a major compute, storage and networking building block a working definition of virtualization would be helpful....blah blah blah
\end{abstract}

\section{Introduction}
\label{sec:intro}
Virtualization is a relationship between producers and consumers of a given resource. For example, a physical server may provide sharing of cores (virtualization) to some number of consumers of those cores. Let $P_r$ be the set of producers of a resource $r$ such that

\begin{align} % requires amsmath; align* for no eq. number
   P_r = \sum\limits_{i = 1}^n P_{i,r} 
\end{align}

and let $C_r$ be the set of consumers of a resource $r$ such that 

\begin{align} 
  C_r = \sum\limits_{j = 1}^m C_{j,r}
\end{align}

Then the $virtualization$ of a resource $r$, $V_r$ is defined to be

\begin{align}
V_r = (P_r, C_r, F_r(P_r,C_r)) 
\end{align}

where 

\begin{align}
F_r = \iint \limits_{i,j}   f_{r_{x,y}}(P_{x,r},C_{y,r}) \,dx \,dy
\end{align}

and 
\begin{align}
f_{r_{x,y}}: C_{y,r} \rightarrow P_{x,r}
\end{align}

That is, a resource $r$ is "virtualized" by a function $F_r$. That is,  $F_r:  C_r \rightarrow P_r$. Note that $V_r \approx L(r)$, i.e., \emph{virtualization is a layering}.

\subsection{Interesting Boundary Conditions}
There are several interesting cases for $P:C$, namely

\begin{itemize}
\item $1:1$  -- This is the "no virtualization" case
\item $1:M$  -- This is the hypervisor case
\item $N:1$  -- This is the "big switch" case
\item $N:M$  -- This is the "scale out" case
\end{itemize}
%\subsection{}



\end{document}  