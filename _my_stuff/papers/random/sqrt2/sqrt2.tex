\documentclass[11pt, oneside]{article}   	% use "amsart" instead of "article" for AMSLaTeX format


% \usepackage{draftwatermark}
% \SetWatermarkText{Draft}
% \SetWatermarkScale{5}
% \SetWatermarkLightness {0.9} 
% \SetWatermarkColor[rgb]{0.7,0,0}


\usepackage{geometry}                		% See geometry.pdf to learn the layout options. There are lots.
\geometry{letterpaper}                   		% ... or a4paper or a5paper or ... 
%\geometry{landscape}                		% Activate for for rotated page geometry
%\usepackage[parfill]{parskip}    		% Activate to begin paragraphs with an empty line rather than an indent
\usepackage{graphicx}				% Use pdf, png, jpg, or eps� with pdflatex; use eps in DVI mode
								% TeX will automatically convert eps --> pdf in pdflat						
								% TeX will automatically convert eps --> pdf in pdflatex		
\usepackage{amssymb}
\usepackage{amsmath}
\usepackage{amsthm}
\usepackage{mathrsfs}
\usepackage{hyperref}
\usepackage{url}
\usepackage{subcaption}
\usepackage{authblk}
\usepackage{mathtools}
\usepackage{graphicx}
\usepackage[export]{adjustbox}
\usepackage{fixltx2e}
\usepackage{hyperref}
\usepackage{alltt}
\usepackage{color}
\usepackage[utf8]{inputenc}
\usepackage[english]{babel}
\usepackage{float}
\usepackage{bigints}
\usepackage{braket}
\usepackage{siunitx}

\theoremstyle{definition}
\newtheorem{thm}{Theorem}[section]
% \newtheorem{defn}[thm]{Definition}
\newtheorem{definition}{Definition}[section]
\newtheorem{example}{Example}[section]
% \newtheorem[thm]{exmp}
\newtheorem{proposition}{Proposition}[section]




\newcommand{\veq}{\mathrel{\rotatebox{90}{$=$}}}
\DeclareMathOperator{\bda}{\Big \downarrow}

\DeclareMathOperator{\mymod}{\text{mod}}

\DeclareMathOperator{\E}{\mathbb{E}}
\newcommand{\argmax}{\operatornamewithlimits{argmax}}
\newcommand{\argmin}{\operatornamewithlimits{argmin}}


\title{Proof that the square root of 2 is irrational}
\author{David Meyer \\ dmm@\{1-4-5.net,uoregon.edu\}}

\date{Last update: January 31, 2016}							% Activate to display a given date or no date



\begin{document}
\maketitle

\section{Introduction}
Here we prove by contradiction that $\sqrt{2} \notin \mathbb{Q}$. This approach to proving the irrationality of $\sqrt{2}$ is sometimes called
"Proof by infinite descent, not involving factoring" \cite{wiki:sqrt2}.

\begin{thm}
$\sqrt{2} \notin \mathbb{Q}$
\end{thm}

\noindent
\textbf{Proof:} As mentioned above, in this proof we assume that, for contradiction,  $\sqrt{2} \in \mathbb{Q}$. Then $\exists a,b  \in \mathbb{Z}$
such that $\frac{a}{b} = \sqrt{2}$ and  we assume that $\frac{a}{b}$ is simplified to \emph{lowest terms}. The contradiction we will find is that 
$\frac{a}{b}$ cannot be in lowest term and in fact, our assumptions imply that 2 divides $a$ (written $2|a$) \emph{and}  2 divides $b$ 
(and hence $\frac{a}{b}$ is not in lowest terms). Ok, but why?


\bigskip
\noindent
Well, consider our assumption $\sqrt{2} \in \mathbb{Q}$, which implies that for some $a,b \in \mathbb{Z}, b \neq 0$

\bigskip
\begin{equation*}
\frac{a}{b} = \sqrt{2}
\end{equation*}


\bigskip
\noindent
Then 

\begin{equation*}
a = \sqrt{2} b
\end{equation*}

\bigskip
\noindent
and therefore 

\begin{equation}
a^2 = 2b^2
\label{eqn:squared}
\end{equation}

\bigskip
\noindent
So $2|a^2$. So $a^2$ is even which implies that $a$ is even\footnote{Note that if $a$ is not even, then $a^2$ is not even.}. Now, since $a$ is even we can
write 

\begin{equation}
a = 2n
\label{eqn:a}
\end{equation}

\bigskip
\noindent
for some $n \in \mathbb{Z}$. If we plug this into Equation \ref{eqn:squared} we get

\begin{equation*}
4 n^2 = 2 b^2
\end{equation*}

\bigskip
\noindent
and therefore $2n^2 = b^2$. So $b^2$ and hence $b$ is even. So we can write

\begin{equation}
 b = 2 m
 \label{eqn:b}
 \end{equation}

 \bigskip
 \noindent
 for some $m \in \mathbb{Z}$. So for some $a,b \in \mathbb{Z}$ and  $b \neq 0$ our assumption implies


\begin{equation*}
\begin{array}{llll}
\sqrt{2} \in \mathbb{Q} 
&\Rightarrow& \frac{a}{b} = \sqrt{2}                               &\mathrel{\#} \text{assume $\frac{a}{b}$ is in lowest terms} \\
&\Rightarrow& a = 2n                                                     &\mathrel{\#} \text{Equation } \ref{eqn:a} \\
&\Rightarrow& b = 2m                                                  &\mathrel{\#} \text{Equation } \ref{eqn:b} \\
&\Rightarrow& \text{$2|a$ \emph{and} $2|b$}               &\mathrel{\#} \text{the contradiction ($\frac{a}{b}$ not in lowest terms)}
\end{array}
\end{equation*}

\bigskip
\noindent
So we assumed that $\frac{a}{b}$ was in lowest terms and we showed that this leads to a contradiction,  namely that 2 divides both $a$ and $b$
and thus $\frac{a}{b}$ is not in lowest terms (the contradiction). Hence our original assumption that $\sqrt{2} \in \mathbb{Q}$ is false and
$\sqrt{2} \notin \mathbb{Q}$. $\square$

\newpage
\section{Acknowledgements}

\bibliographystyle{plain}
\bibliography{/Users/dmm/papers/bib/qc}
\end{document} 
