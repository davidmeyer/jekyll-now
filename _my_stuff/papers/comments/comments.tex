\documentclass[11pt, oneside]{article}   	% use "amsart" instead of "article" for AMSLaTeX format


% \usepackage{draftwatermark}
% \SetWatermarkText{Draft}
% \SetWatermarkScale{5}
% \SetWatermarkLightness {0.9} 
% \SetWatermarkColor[rgb]{0.7,0,0}

\usepackage{geometry}                		% See geometry.pdf to learn the layout options. There are lots.
\geometry{letterpaper}                   		% ... or a4paper or a5paper or ... 
%\geometry{landscape}                		% Activate for for rotated page geometry
%\usepackage[parfill]{parskip}    		% Activate to begin paragraphs with an empty line rather than an indent
\usepackage{graphicx}				% Use pdf, png, jpg, or eps� with pdflatex; use eps in DVI mode
								% TeX will automatically convert eps --> pdf in pdflat						
								% TeX will automatically convert eps --> pdf in pdflatex		
\usepackage{amssymb}
\usepackage{mathrsfs}
\usepackage{hyperref}
\usepackage{url}
\usepackage{subcaption}
\usepackage{authblk}
\usepackage{amsmath}
\usepackage{mathtools}
\usepackage{graphicx}
\usepackage[export]{adjustbox}
\usepackage{fixltx2e}
\usepackage{hyperref}
\usepackage{alltt}
\usepackage{color}
\usepackage[utf8]{inputenc}
\usepackage[english]{babel}
\usepackage{float}
\usepackage{bigints}
\usepackage{braket}
\usepackage{siunitx}

%
% so you can do e.g., \begin{bmatrix}[r] (or [c] or [l])
%

\makeatletter
\renewcommand*\env@matrix[1][c]{\hskip -\arraycolsep
  \let\@ifnextchar\new@ifnextchar
  \array{*\c@MaxMatrixCols #1}}
\makeatother

\newcommand{\argmax}{\operatornamewithlimits{argmax}}
\newcommand{\argmin}{\operatornamewithlimits{argmin}}

\title{Comments}
\author{David Meyer \\ dmm@\{1-4-5.net,uoregon.edu\}}

\date{Last update: \today}							% Activate to display a given date or no date



\begin{document}
\maketitle

\bigskip
\noindent
If you think about it, suppose your age is $x$ and mine is $y$. Then

\begin{equation*}
\lim_{n \rightarrow \infty} \frac{x + n}{y + n} = 1
\end{equation*}

\bigskip
\noindent
Why? Suppose that you are 10 years old ($x = 10$) and I am 20 years old ($y = 20$). Then at year $0$ (now, $n = 0$), then  
$\frac{x + n}{y + n} =  \frac{10 + 0}{20 + 0}  = \frac{1}{2}$ = 0.50.  Now,  in 10 years ($n = 10$) $\frac{x + n}{y + n} =  
\frac{10 + 10}{20 + 10}  = \frac{20}{30} = \frac{2}{3}$ = 0.67. In 100 years ($n = 100$)  $\frac{x + n}{y + n} =  \frac{10 + 100}{20 + 100}  
= \frac{110}{120} = \frac{11}{12} = 0.92$.  In 1000 years, $\frac{1010}{1020} = 0.99$. When $n = \infty$ (asymptotically or "in the limit"), 
$\frac{x + \infty}{y + \infty} = 1$.
 
\bigskip
\noindent
{\Large Long story short:  Given enough time we're all the same age.}

\newpage
\bibliographystyle{plain}
\bibliography{/Users/dmm/papers/bib/qc}



\end{document} 
